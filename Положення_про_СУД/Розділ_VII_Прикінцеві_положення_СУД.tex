\section*{Розділ VII. Прикінцеві та перехідні положення}

\subsection*{7.1. Порядок внесення змін та доповнень}
    7.1.1. Зміни та доповнення до цього Положення вносяться виключно Конференцією студентів Університету (КСУ) більшістю голосів від загального складу її делегатів.

    7.1.2. Право ініціювати внесення змін та доповнень до цього Положення мають Студентська уповноважена делегація (СУД) або група делегатів КСУ кількістю не менше однієї п'ятої (1/5) від загального складу КСУ.

    7.1.3. Проєкт змін та доповнень до цього Положення підлягає попередньому оприлюдненню та обговоренню в порядку, встановленому Регламентом КСУ.

\subsection*{7.2. Тлумачення норм Положення}
    7.2.1. У разі виникнення потреби в роз'ясненні окремих норм цього Положення, офіційне тлумачення надається Студентською уповноваженою делегацією (СУД) за власною ініціативою або за запитом інших ОСС чи групи студентів.

    7.2.2. Тлумачення, надане СУД, набуває чинності з моменту його оприлюднення. Конференція студентів Університету (КСУ) на своєму найближчому засіданні може переглянути, змінити або скасувати таке тлумачення за власною ініціативою або за зверненням зацікавлених осіб/органів. Після розгляду КСУ рішення щодо тлумачення є остаточним.

\subsection*{7.3. Набуття чинності}
    7.3.1. Це Положення затверджується Конференцією студентів Університету (КСУ) більшістю голосів від загального складу її делегатів.

    7.3.2. Це Положення набуває чинності з дня його офіційного оприлюднення на інформаційних ресурсах ОСС Університету після його затвердження КСУ.

\subsection*{7.4. Перехідні положення}
    7.4.1. Питання, пов'язані з першим формуванням складу СУД після затвердження цього Положення (строки проведення виборів, організаційні аспекти тощо), вирішуються окремими рішеннями КСУ. 