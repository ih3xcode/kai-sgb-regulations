\section*{Розділ III. Структура та керівні органи СУД}
\addcontentsline{toc}{section}{Розділ III. Структура та керівні органи СУД}

\subsection*{3.1. Керівні посади СУД}
\addcontentsline{toc}{subsection}{3.1. Керівні посади СУД}
    3.1.1. Для організації своєї роботи Студентська уповноважена делегація (СУД) обирає зі свого складу керівні посади: Голову СУД, Заступника Голови СУД та Секретаря СУД.

    3.1.2. Голова, Заступник Голови та Секретар СУД обираються на першому засіданні новообраного складу СУД більшістю голосів від загального складу членів СУД шляхом таємного або відкритого голосування за рішенням СУД.

    3.1.3. Строк повноважень Голови, Заступника Голови та Секретаря СУД відповідає строку їхніх повноважень як членів СУД, але вони можуть бути достроково переобрані за рішенням СУД, прийнятим не менш як двома третинами голосів від загального складу СУД.

\subsection*{3.2. Голова СУД}
\addcontentsline{toc}{subsection}{3.2. Голова СУД}
    3.2.1. Голова СУД здійснює загальне керівництво діяльністю СУД, організовує її роботу та представляє СУД у відносинах з іншими органами ОСС, адміністрацією Університету та зовнішніми організаціями.

    3.2.2. До повноважень Голови СУД належить:

        \begin{enumerate}[label=\alph*)]
            \item Скликання та ведення засідань СУД;
            \item Формування проєкту порядку денного засідань;
            \item Підписання протоколів засідань та офіційних документів СУД (рішень, висновків, запитів тощо);
            \item Розподіл обов'язків між членами СУД;
            \item Організація виконання рішень СУД та контроль за їх виконанням;
            \item Звітування про діяльність СУД перед КСУ (разом з колегіальним звітом);
            \item Здійснення інших повноважень, передбачених цим Положенням та рішеннями СУД.
        \end{enumerate}

    3.2.3. Голова СУД може приймати операційні рішення та робити офіційні заяви від імені СУД між засіданнями з питань, що не потребують колегіального вирішення або є невідкладними, з подальшим інформуванням членів СУД на найближчому засіданні.

\subsection*{3.3. Заступник Голови СУД}
\addcontentsline{toc}{subsection}{3.3. Заступник Голови СУД}
    3.3.1. Заступник Голови СУД обирається для допомоги Голові у виконанні його функцій та для забезпечення безперервності роботи керівництва СУД.

    3.3.2. Заступник Голови СУД виконує обов'язки Голови СУД у разі його відсутності (відпустка, хвороба, відрядження тощо) або за його дорученням.

    3.3.3. Заступник Голови СУД виконує інші обов'язки, покладені на нього Головою СУД або рішенням СУД.

\subsection*{3.4. Секретар СУД}
\addcontentsline{toc}{subsection}{3.4. Секретар СУД}
    3.4.1. Секретар СУД відповідає за організаційно-технічне забезпечення діяльності СУД та ведення документації.

    3.4.2. До функцій Секретаря СУД належить:

        \begin{enumerate}[label=\alph*)]
            \item Ведення та оформлення протоколів засідань СУД;
            \item Забезпечення ведення діловодства СУД, облік та зберігання документів;
            \item Інформування членів СУД про час, місце та порядок денний засідань;
            \item Підготовка матеріалів до засідань за дорученням Голови СУД;
            \item Забезпечення оприлюднення рішень СУД у встановленому порядку;
            \item Виконання інших організаційних доручень Голови СУД.
        \end{enumerate}

\subsection*{3.5. Колегіальний орган та внутрішні структури СУД}
\addcontentsline{toc}{subsection}{3.5. Колегіальний орган та внутрішні структури СУД}
    3.5.1. Вищим керівним органом СУД є загальні збори її членів, які проводяться у формі засідань. Саме на засіданнях приймаються колегіальні рішення СУД з усіх питань, віднесених до її компетенції.

    3.5.2. Для детального опрацювання окремих напрямків своєї діяльності (наприклад, фінансовий контроль, розгляд скарг, медіація, моніторинг нормативної бази тощо) СУД має право створювати зі свого складу постійні або тимчасові комісії чи робочі групи. Порядок їх створення та діяльності визначається окремими рішеннями СУД.