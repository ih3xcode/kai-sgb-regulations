\section*{Розділ VI. Взаємодія СУД з іншими органами та підзвітність}

\subsection*{6.1. Взаємодія з Конференцією студентів Університету (КСУ)}
    6.1.1. СУД є підзвітною та підконтрольною виключно КСУ.

    6.1.2. СУД звітує про свою діяльність перед КСУ за вимогою КСУ, але не рідше одного разу на рік. Форма та зміст звіту визначаються СУД за погодженням з КСУ (або її робочим органом).

    6.1.3. КСУ має право заслуховувати позачергові звіти або інформацію Голови чи членів СУД з окремих питань діяльності.

    6.1.4. КСУ може надавати СУД рекомендації або запити щодо напрямків її роботи чи необхідності розгляду певних питань. Такі рекомендації/запити не є обов'язковими дорученнями і приймаються СУД до розгляду в межах її незалежності та компетенції.

    6.1.5. Рішення КСУ, що стосуються діяльності СУД (затвердження Положення, обрання/припинення повноважень членів, затвердження звітів тощо), є обов'язковими для СУД.

\subsection*{6.2. Взаємодія з іншими органами студентського самоврядування (ОСС)}
    6.2.1. СУД взаємодіє з СР КАІ, СР СМ, ЦВКс, СРФ та іншими ОСС з метою здійснення своїх контрольних та наглядових функцій.

    6.2.2. СУД має право запитувати будь-яку інформацію та документацію (протоколи, рішення, звіти, фінансові документи тощо), необхідну для виконання своїх повноважень, від будь-якого ОСС та його посадових осіб.

    6.2.3. Відповідний ОСС та його посадові особи зобов'язані надати запитувану СУД інформацію та/або документацію у повному обсязі та у строк, встановлений СУД, який не може бути меншим за 3 (три) робочі дні, якщо інший термін не обґрунтований невідкладністю питання.

    6.2.4. СУД надає ОСС обов'язкові до розгляду рекомендації щодо усунення виявлених порушень, а також може надавати висновки та роз'яснення щодо застосування нормативних актів.

    6.2.5. СУД інформує відповідні ОСС про результати розгляду скарг на їхню діяльність та прийняті рішення.

\subsection*{6.3. Взаємодія з Адміністрацією Університету}
    6.3.1. СУД взаємодіє з посадовими особами та структурними підрозділами Адміністрації Університету переважно в рамках розгляду скарг та звернень студентів на дії/бездіяльність Адміністрації, а також при здійсненні контролю за дотриманням прав студентів.

    6.3.2. СУД має право направляти офіційні запити до посадових осіб та структурних підрозділів Адміністрації Університету для отримання інформації, документів та пояснень, необхідних для розгляду скарг або здійснення своїх функцій. Адміністрація надає відповідь у строки, встановлені законодавством України про звернення громадян.

    6.3.3. За результатами розгляду скарг на Адміністрацію або виявлення системних проблем, СУД може надавати рекомендації відповідним посадовим особам або структурним підрозділам Адміністрації щодо усунення порушень або вдосконалення роботи.

    6.3.4. СУД може ініціювати проведення спільних засідань, консультацій або робочих зустрічей з представниками Адміністрації для обговорення питань, що належать до її компетенції.

    6.3.5. У разі систематичного невиконання рекомендацій СУД з боку посадових осіб чи структурних підрозділів Адміністрації або ненадання відповіді на запити СУД, Голова СУД має право звернутися з відповідним поданням до Ректора Університету або ініціювати розгляд цього питання представниками студентства у Вченій раді Університету.

\subsection*{6.4. Оскарження рішень СУД до Конференції студентів Університету}
    6.4.1. Рішення СУД з наступних питань можуть бути оскаржені до КСУ:
        \begin{enumerate}[label=\alph*)]
            \item Прийняття рішення про припинення повноважень члена ОСС (п. 4.6.1);
            \item Прийняття рішення про відсторонення члена ОСС від виконання повноважень (п. 4.6.2.в);
            \item Визнання неправомірним і скасування акту / рішення органу ОСС (п. 4.3.3.г);
            \item Винесення органу ОСС обов'язкового припису (як за результатами розгляду скарг (п. 4.3.3), так і за результатами контролю (п. 4.1.3));
            \item Надання офіційного тлумачення норм Положення про ОСС (п. 4.5);
            \item Прийняття обов'язкового рішення щодо розмежування компетенції у спорах між ОСС (п. 4.4.2).
        \end{enumerate}

    6.4.2. Право на оскарження рішення СУД до КСУ мають:
        \begin{enumerate}[label=\alph*)]
            \item Особа або орган ОСС, щодо якого СУД прийняла оскаржуване рішення;
            \item Скаржник, якщо він не погоджується з рішенням СУД за результатами розгляду його скарги (у випадках, коли таке рішення СУД підлягає оскарженню згідно п. 6.4.1);
            \item Інший орган ОСС, якщо оскаржуване рішення СУД безпосередньо порушує його права чи законні інтереси.
        \end{enumerate}

    6.4.3. Апеляція на рішення СУД подається у письмовій формі на ім'я Спікера КСУ протягом 1 (одного) місяця з дня офіційного оприлюднення або отримання копії відповідного рішення СУД.

    6.4.4. Подання апеляції не зупиняє дію оскаржуваного рішення СУД, якщо КСУ (або її Спікер до засідання КСУ) не прийме іншого рішення.

    6.4.5. Спікер КСУ передає апеляцію на розгляд тимчасової апеляційної комісії, що створюється КСУ. До складу цієї комісії не можуть входити члени СУД, особи, щодо яких було прийнято оскаржуване рішення, та особи, які подали скаргу/апеляцію.

    6.4.6. Апеляційна комісія вивчає матеріали справи, заслуховує пояснення сторін (апелянта та представника СУД) і надає свій висновок та проєкт рішення на розгляд пленарного засідання КСУ.

    6.4.7. За результатами розгляду апеляції КСУ більшістю голосів від загального складу делегатів може прийняти одне з таких рішень:
        \begin{enumerate}[label=\alph*)]
            \item Залишити рішення СУД без змін, а апеляцію без задоволення;
            \item Скасувати рішення СУД повністю або в частині;
            \item Змінити рішення СУД;
            \item Направити справу на новий розгляд до СУД з наданням відповідних вказівок.
        \end{enumerate}
        
    6.4.8. Рішення КСУ за результатами розгляду апеляції є остаточним в системі студентського самоврядування. 