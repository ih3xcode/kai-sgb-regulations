\section*{Розділ V. Порядок роботи СУД}

\subsection*{5.1. Засідання СУД}
    5.1.1. Основною формою роботи СУД є засідання, які проводяться за необхідності, але періодичність чергових засідань визначається самою СУД на початку строку повноважень (рекомендується проводити не рідше одного разу на два місяці протягом навчального року).

    5.1.2. Засідання СУД скликаються Головою СУД. Позачергове засідання може бути скликане за ініціативою Голови СУД або на вимогу не менше ніж третини (1/3) від загального складу членів СУД.

    5.1.3. Повідомлення про дату, час, місце та проєкт порядку денного засідання надсилається членам СУД Секретарем СУД не пізніше ніж за 3 (три) календарні дні до засідання. У невідкладних випадках цей строк може бути скорочений за рішенням Голови СУД.

    5.1.4. Засідання СУД є правомочним (має кворум), якщо на ньому присутні не менше двох третин (2/3) від загального (обраного) складу членів СУД.

    5.1.5. СУД може проводити засідання та голосування з використанням засобів дистанційного електронного зв'язку (відеоконференції, системи електронного голосування тощо) за умови забезпечення належної ідентифікації учасників, можливості вільного обговорення питань та достовірної фіксації результатів голосування. Порядок використання дистанційних технологій визначається внутрішнім регламентом СУД або окремим її рішенням.

\subsection*{5.2. Прийняття рішень СУД}
    5.2.1. Рішення СУД приймаються на її засіданнях шляхом відкритого або таємного голосування (спосіб голосування визначається СУД для кожного питання окремо, якщо інше не встановлено цим Положенням).

    5.2.2. Рішення СУД вважається прийнятим, якщо за нього проголосувала \textbf{більшість голосів від загального (обраного) складу членів СУД}, якщо цим Положенням не встановлено іншої необхідної кількості голосів.

    5.2.3. Рішення з наступних питань потребують \textbf{кваліфікованої більшості – не менше двох третин (2/3) голосів від загального (обраного) складу членів СУД}:
        \begin{enumerate}[label=\alph*)]
            \item Прийняття рішення про припинення повноважень члена ОСС (відповідно до п. 4.6.1);
            \item Прийняття рішення про відсторонення члена ОСС від виконання повноважень (відповідно до п. 4.6.2);
            \item Надання офіційного тлумачення норм Положення про ОСС (відповідно до п. 4.5);
            \item Прийняття рішення щодо розмежування компетенції у спорах між ОСС (відповідно до п. 4.4.2);
            \item Внесення пропозицій до КСУ про висловлення недовіри члену ОСС;
            \item Затвердження внутрішнього регламенту роботи СУД;
            \item Інші питання, визначені цим Положенням або рішенням КСУ.
        \end{enumerate}

    5.2.4. У разі рівного розподілу голосів при голосуванні з будь-якого питання, голос Голови СУД є вирішальним.

\subsection*{5.3. Процедури розгляду питань}
    5.3.1. \textbf{Розгляд скарг та звернень на ОСС:} Розгляд скарг відбувається на засіданнях СУД. Засідання є відкритими, якщо СУД не прийме іншого рішення або якщо цього не вимагає хоча б одна зі сторін (скаржник або представник органу/особа, на яку подано скаргу) з метою захисту конфіденційної інформації або персональних даних. Скаржник та представник органу/особа, дії якої оскаржуються, мають бути повідомлені про засідання та мають право бути присутніми, надавати пояснення та докази. Детальна процедура розгляду скарг, включаючи строки, порядок подання, фіксацію результатів, визначається внутрішнім регламентом СУД.

    5.3.2. \textbf{Розгляд спорів між ОСС:} Розгляд спорів здійснюється на засіданні СУД із запрошенням представників сторін спору. Сторони мають право представити свої позиції та аргументи. Рішення СУД оформлюється письмово та надсилається сторонам спору.

    5.3.3. \textbf{Дисциплінарні процедури щодо членів ОСС:} Рішення про застосування заходів впливу (попередження, визнання дій неправомірними, відсторонення, припинення повноважень) приймається за результатами розгляду справи про порушення, під час якого забезпечується право відповідного члена ОСС на захист (бути повідомленим про суть звинувачень, надавати пояснення, докази). Підставами для прийняття СУД рішення про \textbf{припинення повноважень} члена ОСС (відповідно до п. 4.6.1) є:
        \begin{enumerate}[label=\roman*)]
            \item Грубе порушення Статуту Університету, Положення про ОСС або положення про відповідний ОСС, що завдало істотної шкоди студентському самоврядуванню;
            \item Систематичне (більше 2 разів протягом семестру) невиконання обов'язкових рішень СУД без поважних причин;
            \item Вчинення дій, що дискредитують органи студентського самоврядування або підривають їх авторитет;
            \item Використання статусу члена ОСС у власних інтересах або в інтересах третіх осіб, що суперечить цілям ОСС (доведений конфлікт інтересів);
            \item Нецільове використання коштів або майна ОСС (підтверджене результатами перевірки СУД або іншого уповноваженого органу).
        \end{enumerate}

    \subsubsection*{5.3.4. Особливості розгляду скарг на дії/бездіяльність Адміністрації Університету}
        5.3.4.1. Скарги студентів на дії/бездіяльність посадових осіб або структурних підрозділів Адміністрації Університету подаються до СУД у письмовому вигляді з чітким викладенням суті проблеми та обґрунтуванням порушення прав чи інтересів студента.

        5.3.4.2. При розгляді такої скарги СУД має право:
            \begin{enumerate}[label=\alph*)]
                \item Направляти офіційні запити до відповідної посадової особи/структурного підрозділу Адміністрації для отримання пояснень, інформації та документів у строки, встановлені законодавством про звернення громадян;
                \item Запрошувати представників Адміністрації на свої засідання для надання пояснень та обговорення скарги (за згодою представника Адміністрації).
            \end{enumerate}

        5.3.4.3. За результатами розгляду скарги на Адміністрацію СУД може прийняти одне або декілька з таких рішень:
            \begin{enumerate}[label=\alph*)]
                \item Визнати скаргу обґрунтованою (повністю або частково) або необґрунтованою;
                \item Надати рекомендації відповідній посадовій особі/структурному підрозділу Адміністрації щодо усунення порушення прав чи інтересів студента та/або вдосконалення відповідних процедур;
                \item Звернутися з офіційним поданням або інформаційним листом до Ректора Університету щодо виявленого порушення або системної проблеми;
                \item Ініціювати розгляд питання представниками студентства у Вченій раді Університету;
                \item Оприлюднити знеособлену інформацію про типові порушення або результати розгляду суспільно значущих справ (за рішенням СУД та з дотриманням законодавства про захист персональних даних).
            \end{enumerate}

        5.3.4.4. СУД інформує скаржника та, за необхідності, відповідну посадову особу/структурний підрозділ Адміністрації про результати розгляду скарги.

\subsection*{5.4. Документообіг та оприлюднення}
    5.4.1. Хід засідань СУД фіксується у протоколі, який веде Секретар СУД. Протокол підписується Головою СУД та Секретарем СУД.

    5.4.2. Рішення СУД оформлюються окремими документами або як частина протоколу засідання.

    5.4.3. Протоколи засідань (за винятком частин, що містять інформацію з обмеженим доступом) та рішення СУД підлягають оприлюдненню на офіційних інформаційних ресурсах СУД (якщо такі створені) та/або ОСС Університету протягом 5 (п'яти) робочих днів з дня проведення засідання/прийняття рішення. 