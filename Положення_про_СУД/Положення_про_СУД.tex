\documentclass[12pt, a4paper]{extarticle}
\usepackage[ukrainian]{babel}
\usepackage{fontspec}
\usepackage{amsmath}
\usepackage{amssymb}
\usepackage{graphicx}
\usepackage{indentfirst}
\usepackage{enumitem}
\usepackage{geometry}
\geometry{a4paper, margin=2cm}
\usepackage[nottoc]{tocbibind}
\usepackage{hyperref}
\hypersetup{
    colorlinks=true,
    linkcolor=black,
    filecolor=blue,      
    urlcolor=blue,
    pdftitle={Положення про Студентську уповноважену делегацію},
    pdfauthor={ДУ ``КАІ''},
    pdfcreator={LaTeX},
    pdfproducer={LaTeX}
}

\setmainfont{Liberation Serif}

\setlength{\parskip}{1ex}
\setlength{\parindent}{1.25cm}

\title{Положення про Студентську уповноважену делегацію}
\author{Державний університет ``Київський авіаційний інститут''}
\date{\today}

\begin{document}

\begin{titlepage}
    \centering
    \vspace*{\fill}
    {\Huge\bfseries Положення}\par
    \vspace{1em}
    {\LARGE про Студентську уповноважену делегацію}\par
    \vspace{0.5em}
    {\large Державного некомерційного підприємства \\ ``Державний університет ``Київський авіаційний інститут''}\par
    \vspace*{\fill}
\end{titlepage}

\section*{Глосарій термінів та абревіатур}
\begin{description}[leftmargin=3cm,style=nextline]
    \item[СУД] Студентська уповноважена делегація -- постійно діючий орган, відповідальний за контрольну діяльність в системі ОСС.
    \item[ОСС] Органи студентського самоврядування.
    \item[КСУ] Конференція студентів Університету -- вищий представницький орган студентського самоврядування.
    \item[СР КАІ] Студентська рада Київського авіаційного інституту -- вищий виконавчий орган студентського самоврядування.
    \item[ЦВКс] Центральна виборча комісія студентів -- постійно діючий орган, відповідальний за організацію та проведення виборів до ОСС.
    \item[СР СМ] Студентська рада студмістечка -- орган, що координує діяльність ОСС гуртожитків.
    \item[СРФ/СРІ] Студентська рада факультету/інституту -- основний орган студентського самоврядування на рівні факультету/інституту.
\end{description}

% Додаємо зміст
\renewcommand{\contentsname}{Зміст}
\tableofcontents
\newpage

\section*{Розділ I. Загальні положення}

\subsection*{1.1. Статус та мета Студентської уповноваженої делегації}
    1.1.1. Студентська уповноважена делегація (далі – СУД) є постійно діючим колегіальним органом студентського самоврядування Університету, який здійснює контрольну та наглядову діяльність в системі органів студентського самоврядування (ОСС) і є підзвітним та підконтрольним виключно Конференції студентів Університету (КСУ).

    1.1.2. Головною метою діяльності СУД є забезпечення дотримання ОСС усіх рівнів вимог законодавства України, Статуту Університету, Положення про органи студентського самоврядування Університету (далі – Положення про ОСС), цього Положення та інших нормативних актів, що регулюють діяльність ОСС, а також захист прав та інтересів студентів в межах системи ОСС, сприяння покращенню ефективності роботи ОСС та утвердженню демократичних принципів у їхній діяльності.

\subsection*{1.2. Принципи діяльності СУД}
    1.2.1. Діяльність СУД ґрунтується на принципах:
        \begin{enumerate}[label=\alph*)]
            \item Законності;
            \item Незалежності від Адміністрації Університету та від інших органів студентського самоврядування (СР КАІ, СР СМ, ЦВКс, СРФ тощо) при здійсненні своїх повноважень;
            \item Об'єктивності та неупередженості;
            \item Колегіальності при прийнятті рішень;
            \item Прозорості та відкритості (з урахуванням обмежень щодо інформації з обмеженим доступом);
            \item Підзвітності та підконтрольності Конференції студентів Університету.
        \end{enumerate}

\subsection*{1.3. Нормативно-правова база діяльності СУД}
    1.3.1. СУД у своїй діяльності керується Конституцією України, Законом України ``Про вищу освіту'', Статутом Університету, Положенням про ОСС, цим Положенням, рішеннями КСУ та іншими нормативно-правовими актами України.

    1.3.2. Це Положення визначає мету, принципи, порядок формування, структуру, повноваження, порядок роботи та інші аспекти діяльності СУД.

\subsection*{1.4. Офіційні рішення та заяви СУД}
    1.4.1. Офіційними рішеннями СУД є рішення, прийняті колегіально на її засіданнях у порядку, встановленому цим Положенням.

    1.4.2. Голова СУД має право робити офіційні заяви та приймати певні операційні рішення від імені СУД у випадках та порядку, передбачених Розділами III та V цього Положення.

    1.4.3. Офіційні рішення СУД, прийняті в межах її компетенції, є обов'язковими до розгляду та/або виконання відповідними органами ОСС та їх посадовими особами. 
\section*{Розділ II. Членство, порядок формування та припинення повноважень}

\subsection*{2.1. Склад та строк повноважень СУД}
    2.1.1. Кількісний склад СУД визначається Конференцією студентів Університету (КСУ) перед кожними виборами та становить не менше 3 (трьох) та не більше 7 (семи) осіб. Для забезпечення ефективного прийняття рішень рекомендується обирати непарну кількість членів СУД.

    2.1.2. Строк повноважень членів СУД становить 2 (два) роки. Це забезпечує стабільність роботи органу, накопичення інституційної пам'яті та досвіду, що є критично важливим для виконання контрольних та наглядових функцій. КСУ може розглянути запровадження механізму часткової ротації складу СУД щороку для поєднання стабільності та оновлення.

\subsection*{2.2. Вимоги до кандидатів та членів СУД}
    2.2.1. Членом СУД може бути обраний студент (курсант) денної форми навчання Державного університету ``Київський авіаційний інститут'', який відповідає таким вимогам:
        \begin{enumerate}[label=\alph*)]
            \item Навчається на другому або старшому курсі (для бакалаврату) або на будь-якому курсі магістратури/аспірантури;
            \item Має досвід роботи в органах студентського самоврядування не менше одного повного строку повноважень на виборній посаді або одного року як член ОСС;
            \item Не має чинних дисциплінарних стягнень;
            \item Має високі моральні якості та бездоганну репутацію;
            \item Ознайомлений з нормативно-правовою базою діяльності ОСС та Університету.
        \end{enumerate}

    2.2.2. \textbf{Несумісність посад:} Член СУД не може одночасно:
        \begin{enumerate}[label=\alph*)]
            \item Обіймати будь-яку посаду (включаючи членство в комітетах, робочих групах тощо) в будь-якому органі студентського самоврядування;
            \item Обіймати будь-яку оплачувану або адміністративну посаду в Університеті (включаючи роботу в деканатах, відділах, лабораторіях, на кафедрах тощо, крім випадків виробничої практики, передбаченої навчальним планом);
            \item Бути членом політичної партії або обіймати керівну посаду в молодіжному крилі політичної партії.
        \end{enumerate}
    2.2.3. Суміщення статусу члена СУД зі статусом делегата КСУ допускається.

\subsection*{2.3. Порядок формування СУД}
    2.3.1. Члени СУД обираються КСУ шляхом таємного голосування.

    2.3.2. Право висування кандидатів у члени СУД має кожен студент (курсант) Університету, який відповідає вимогам, визначеним у пункті 2.2.1 цього Положення. Висування може здійснюватися шляхом самовисування або шляхом подання кандидатури групою студентів.

    2.3.3. Порядок висування кандидатів, передвиборчої агітації, процедура голосування та встановлення результатів виборів членів СУД визначаються Регламентом КСУ та/або окремим рішенням КСУ, що приймається перед оголошенням виборів. ЦВКс забезпечує організаційно-технічну підтримку виборів до СУД за дорученням КСУ.

    2.3.4. Обраними до складу СУД вважаються кандидати, які набрали найбільшу кількість голосів делегатів КСУ відповідно до встановленої КСУ процедури та квоти.

\subsection*{2.4. Припинення повноважень членів СУД}
    2.4.1. Повноваження члена СУД припиняються у разі:
        \begin{enumerate}[label=\alph*)]
            \item Закінчення строку повноважень;
            \item Подання особистої заяви про складення повноважень;
            \item Припинення статусу студента (курсанта) Університету (завершення навчання, відрахування тощо);
            \item Набрання законної сили обвинувальним вироком суду щодо нього;
            \item Визнання його недієздатним або обмежено дієздатним;
            \item Його смерті;
            \item Систематичного невиконання обов'язків члена СУД без поважних причин (більше 3 пропусків засідань СУД поспіль або більше 5 протягом семестру);
            \item Порушення вимог несумісності посад, визначених у пункті 2.2.2 цього Положення;
            \item Висловлення недовіри Конференцією студентів Університету.
        \end{enumerate}

    2.4.2. Рішення про дострокове припинення повноважень члена СУД \textbf{за його особистою заявою, у зв'язку із систематичним невиконанням обов'язків або порушенням вимог несумісності посад} приймається самою СУД більшістю голосів від її загального складу та підлягає затвердженню КСУ на найближчому засіданні. У випадку \textbf{висловлення недовіри Конференцією студентів Університету}, повноваження припиняються з моменту прийняття відповідного рішення КСУ. В інших випадках, передбачених пунктом 2.4.1 цього Положення, повноваження припиняються автоматично.

    2.4.3. У разі дострокового припинення повноважень члена СУД, КСУ на своєму найближчому засіданні може прийняти рішення про проведення довиборів для заміщення вакантної посади на решту строку повноважень. 
\section*{Розділ III. Структура та керівні органи СУД}
\addcontentsline{toc}{section}{Розділ III. Структура та керівні органи СУД}

\subsection*{3.1. Керівні посади СУД}
\addcontentsline{toc}{subsection}{3.1. Керівні посади СУД}
    3.1.1. Для організації своєї роботи Студентська уповноважена делегація (СУД) обирає зі свого складу керівні посади: Голову СУД, Заступника Голови СУД та Секретаря СУД.

    3.1.2. Голова, Заступник Голови та Секретар СУД обираються на першому засіданні новообраного складу СУД більшістю голосів від загального складу членів СУД шляхом таємного або відкритого голосування за рішенням СУД.

    3.1.3. Строк повноважень Голови, Заступника Голови та Секретаря СУД відповідає строку їхніх повноважень як членів СУД, але вони можуть бути достроково переобрані за рішенням СУД, прийнятим не менш як двома третинами голосів від загального складу СУД.

\subsection*{3.2. Голова СУД}
\addcontentsline{toc}{subsection}{3.2. Голова СУД}
    3.2.1. Голова СУД здійснює загальне керівництво діяльністю СУД, організовує її роботу та представляє СУД у відносинах з іншими органами ОСС, адміністрацією Університету та зовнішніми організаціями.

    3.2.2. До повноважень Голови СУД належить:

        \begin{enumerate}[label=\alph*)]
            \item Скликання та ведення засідань СУД;
            \item Формування проєкту порядку денного засідань;
            \item Підписання протоколів засідань та офіційних документів СУД (рішень, висновків, запитів тощо);
            \item Розподіл обов'язків між членами СУД;
            \item Організація виконання рішень СУД та контроль за їх виконанням;
            \item Звітування про діяльність СУД перед КСУ (разом з колегіальним звітом);
            \item Здійснення інших повноважень, передбачених цим Положенням та рішеннями СУД.
        \end{enumerate}

    3.2.3. Голова СУД може приймати операційні рішення та робити офіційні заяви від імені СУД між засіданнями з питань, що не потребують колегіального вирішення або є невідкладними, з подальшим інформуванням членів СУД на найближчому засіданні.

\subsection*{3.3. Заступник Голови СУД}
\addcontentsline{toc}{subsection}{3.3. Заступник Голови СУД}
    3.3.1. Заступник Голови СУД обирається для допомоги Голові у виконанні його функцій та для забезпечення безперервності роботи керівництва СУД.

    3.3.2. Заступник Голови СУД виконує обов'язки Голови СУД у разі його відсутності (відпустка, хвороба, відрядження тощо) або за його дорученням.

    3.3.3. Заступник Голови СУД виконує інші обов'язки, покладені на нього Головою СУД або рішенням СУД.

\subsection*{3.4. Секретар СУД}
\addcontentsline{toc}{subsection}{3.4. Секретар СУД}
    3.4.1. Секретар СУД відповідає за організаційно-технічне забезпечення діяльності СУД та ведення документації.

    3.4.2. До функцій Секретаря СУД належить:

        \begin{enumerate}[label=\alph*)]
            \item Ведення та оформлення протоколів засідань СУД;
            \item Забезпечення ведення діловодства СУД, облік та зберігання документів;
            \item Інформування членів СУД про час, місце та порядок денний засідань;
            \item Підготовка матеріалів до засідань за дорученням Голови СУД;
            \item Забезпечення оприлюднення рішень СУД у встановленому порядку;
            \item Виконання інших організаційних доручень Голови СУД.
        \end{enumerate}

\subsection*{3.5. Колегіальний орган та внутрішні структури СУД}
\addcontentsline{toc}{subsection}{3.5. Колегіальний орган та внутрішні структури СУД}
    3.5.1. Вищим керівним органом СУД є загальні збори її членів, які проводяться у формі засідань. Саме на засіданнях приймаються колегіальні рішення СУД з усіх питань, віднесених до її компетенції.

    3.5.2. Для детального опрацювання окремих напрямків своєї діяльності (наприклад, фінансовий контроль, розгляд скарг, медіація, моніторинг нормативної бази тощо) СУД має право створювати зі свого складу постійні або тимчасові комісії чи робочі групи. Порядок їх створення та діяльності визначається окремими рішеннями СУД.
\section*{Розділ IV. Повноваження та функції СУД}

\subsection*{4.1. Контроль за дотриманням нормативних актів}
    4.1.1. СУД здійснює постійний контроль за дотриманням усіма органами студентського самоврядування (ОСС) вимог законодавства України, Статуту Університету, Положення про ОСС, цього Положення та інших нормативних актів, що регулюють їхню діяльність.

    4.1.2. СУД має право вимагати від ОСС будь-які документи (протоколи, рішення, звіти тощо), необхідні для здійснення контрольних функцій, та отримувати пояснення від членів ОСС.

    4.1.3. У разі виявлення порушень СУД може надавати обов'язкові до розгляду рекомендації щодо їх усунення, виносити \textbf{попередження} відповідному ОСС або його члену, а також \textbf{припис} про усунення виявлених порушень у визначений строк. У випадках, передбачених цим Положенням, СУД може застосовувати інші заходи впливу.

\subsection*{4.2. Контроль за фінансовою діяльністю ОСС}
    4.2.1. СУД здійснює контроль за цільовим та ефективним використанням коштів ОСС.

    4.2.2. СУД розглядає та \textbf{погоджує проєкт єдиного кошторису (бюджету) ОСС} Університету перед його винесенням на затвердження КСУ. У разі відсутності погодження СУД, проєкт кошторису може бути затверджений КСУ лише за умови, якщо за нього проголосувало не менше двох третин (2/3) від загального складу делегатів КСУ.

    4.2.3. СУД розглядає звіти про виконання кошторисів ОСС та надає свої висновки КСУ.

    4.2.4. СУД має право проводити перевірки фінансової звітності ОСС, а також вибіркові перевірки первинної фінансової документації ОСС (рахунків, актів виконаних робіт тощо) у разі отримання скарг, виявлення ознак порушень або за дорученням КСУ.

\subsection*{4.3. Розгляд скарг та звернень (Функції Омбудсмена)}
    4.3.1. СУД розглядає скарги та звернення студентів (курсантів) Університету щодо будь-яких аспектів діяльності ОСС, включаючи порушення нормативних актів, неетичну поведінку членів ОСС, невиконання ними своїх обов'язків, порушення прав студентів з боку ОСС тощо.

    4.3.2. СУД розглядає скарги та звернення студентів (курсантів) щодо дій або бездіяльності посадових осіб та структурних підрозділів Адміністрації Університету, що стосуються прав та інтересів студентів.
        % Детальна процедура розгляду скарг на Адміністрацію визначається Регламентом СУД.

    4.3.3. За результатами розгляду скарги на ОСС, СУД, в межах своєї компетенції, може прийняти одне або декілька з таких рішень:
        \begin{enumerate}[label=\alph*)]
            \item Визнати скаргу обґрунтованою (повністю або частково) або необґрунтованою;
            \item Визнати дії / бездіяльність відповідача (органу ОСС або його члена) неправомірними;
            \item Визнати певне право скаржника або відсутність повноважень відповідача з певного питання;
            \item Визнати неправомірним і скасувати акт / рішення відповідача (органу ОСС);
            \item Винести відповідачу (органу ОСС) припис вчинити певні дії (наприклад, усунути порушення, надати інформацію) або утриматися від певних дій;
            \item Винести відповідачу (органу ОСС) та/або його керівнику (голові) попередження;
            \item Надати рекомендації відповідачу щодо вдосконалення роботи;
            \item Передати матеріали справи на розгляд КСУ для прийняття рішення, що виходить за межі компетенції СУД.
        \end{enumerate}

    4.3.4. Порядок подання та розгляду скарг визначається Розділом V цього Положення.

\subsection*{4.4. Розгляд спорів між ОСС}
    4.4.1. СУД розглядає спори між ОСС різних рівнів або одного рівня щодо їхньої компетенції, виконання рішень або інших питань діяльності.

    4.4.2. За результатами розгляду спору СУД може прийняти \textbf{рішення щодо розмежування компетенції або порядку взаємодії}, яке є обов'язковим для виконання відповідними ОСС, за винятком питань, що належать до виключної компетенції КСУ.

    4.4.3. Процедура розгляду спорів визначається Розділом V цього Положення.

\subsection*{4.5. Офіційне тлумачення Положення про ОСС}
    4.5.1. СУД надає офіційне тлумачення норм Положення про ОСС за запитом ОСС, членів ОСС або групи студентів.

    4.5.2. Рішення СУД щодо тлумачення норм Положення про ОСС є офіційним та обов'язковим для застосування усіма ОСС.

\subsection*{4.6. Дисциплінарні повноваження}
    4.6.1. У разі виявлення грубих або систематичних порушень членом ОСС вимог нормативних актів, невиконання обов'язків або вчинення дій, що шкодять інтересам студентського самоврядування, СУД має право прийняти рішення про припинення повноважень такого члена ОСС. Перелік конкретних підстав для застосування такого заходу визначається Розділом V цього Положення. Розгляд питання про припинення повноважень члена ОСС за цими підставами здійснюється СУД як першою інстанцією.

    4.6.2. За менш значні порушення або як захід до вирішення питання про припинення повноважень, СУД може прийняти рішення про:
        \begin{enumerate}[label=\alph*)]
            \item Визнання дій / бездіяльності члена ОСС неправомірними;
            \item Винесення члену ОСС попередження;
            \item Відсторонення члена ОСС від виконання повноважень на час розгляду справи про порушення (не більше ніж на 1 місяць).
        \end{enumerate}

    4.6.3. Рішення СУД, зазначені в пп. 4.6.1 та 4.6.2.в, можуть бути оскаржені до КСУ у порядку, визначеному Розділом VI цього Положення.

\subsection*{4.7. Повноваження щодо ініціювання та скликання}
    4.7.1. СУД має право ініціювати перед КСУ питання про внесення змін до Положення про ОСС або до положень про інші ОСС.

    4.7.2. СУД має право ініціювати перед КСУ питання про висловлення недовіри будь-якому члену ОСС (включаючи керівників).

    4.7.3. СУД має право ініціювати скликання позачергової КСУ.

    4.7.4. СУД має право скликати позачергову КСУ у невідкладних випадках у порядку, визначеному Положенням про ОСС.

\subsection*{4.8. Інші повноваження}
    4.8.1. СУД визначає порядок поводження з інформацією з обмеженим доступом в системі ОСС.

    4.8.2. СУД виконує інші повноваження, передбачені Положенням про ОСС та рішеннями КСУ. 
\section*{Розділ V. Порядок роботи СУД}
\addcontentsline{toc}{section}{Розділ V. Порядок роботи СУД}

\subsection*{5.1. Засідання СУД}
\addcontentsline{toc}{subsection}{5.1. Засідання СУД}
    5.1.1. Основною формою роботи СУД є засідання, які проводяться за необхідності, але періодичність чергових засідань визначається самою СУД на початку строку повноважень (рекомендується проводити не рідше одного разу на два місяці протягом навчального року).

    5.1.2. Засідання СУД скликаються Головою СУД. Позачергове засідання може бути скликане за ініціативою Голови СУД або на вимогу не менше ніж третини (1/3) від загального складу членів СУД.

    5.1.3. Повідомлення про дату, час, місце та проєкт порядку денного засідання надсилається членам СУД Секретарем СУД не пізніше ніж за 3 (три) календарні дні до засідання. У невідкладних випадках цей строк може бути скорочений за рішенням Голови СУД.

    5.1.4. Засідання СУД є правомочним (має кворум), якщо на ньому присутні не менше двох третин (2/3) від загального (обраного) складу членів СУД.

    5.1.5. СУД може проводити засідання та голосування з використанням засобів дистанційного електронного зв'язку (відеоконференції, системи електронного голосування тощо) за умови забезпечення належної ідентифікації учасників, можливості вільного обговорення питань та достовірної фіксації результатів голосування. Порядок використання дистанційних технологій визначається внутрішнім регламентом СУД або окремим її рішенням.

\subsection*{5.2. Прийняття рішень СУД}
\addcontentsline{toc}{subsection}{5.2. Прийняття рішень СУД}
    5.2.1. Рішення СУД приймаються на її засіданнях шляхом відкритого або таємного голосування (спосіб голосування визначається СУД для кожного питання окремо, якщо інше не встановлено цим Положенням).

    5.2.2. Рішення СУД вважається прийнятим, якщо за нього проголосувала \textbf{більшість голосів від загального (обраного) складу членів СУД}, якщо цим Положенням не встановлено іншої необхідної кількості голосів.

    5.2.3. Рішення з наступних питань потребують \textbf{кваліфікованої більшості – не менше двох третин (2/3) голосів від загального (обраного) складу членів СУД}:

        \begin{enumerate}[label=\alph*)]
            \item Прийняття рішення про припинення повноважень члена ОСС (відповідно до п. 4.6.1);

            \item Прийняття рішення про відсторонення члена ОСС від виконання повноважень (відповідно до п. 4.6.2);

            \item Надання офіційного тлумачення норм Положення про ОСС (відповідно до п. 4.5);

            \item Прийняття рішення щодо розмежування компетенції у спорах між ОСС (відповідно до п. 4.4.2);

            \item Внесення пропозицій до КСУ про висловлення недовіри члену ОСС;
            \item Затвердження внутрішнього регламенту роботи СУД;
            \item Інші питання, визначені цим Положенням або рішенням КСУ.
        \end{enumerate}

    5.2.4. У разі рівного розподілу голосів при голосуванні з будь-якого питання, голос Голови СУД є вирішальним.

\subsection*{5.3. Процедури розгляду питань}
\addcontentsline{toc}{subsection}{5.3. Процедури розгляду питань}
    5.3.1. \textbf{Розгляд скарг та звернень на ОСС:} Розгляд скарг відбувається на засіданнях СУД. Засідання є відкритими, якщо СУД не прийме іншого рішення або якщо цього не вимагає хоча б одна зі сторін (скаржник або представник органу/особа, на яку подано скаргу) з метою захисту конфіденційної інформації або персональних даних. Скаржник та представник органу/особа, дії якої оскаржуються, мають бути повідомлені про засідання та мають право бути присутніми, надавати пояснення та докази. Детальна процедура розгляду скарг, включаючи строки, порядок подання, фіксацію результатів, визначається внутрішнім регламентом СУД.

    5.3.2. \textbf{Розгляд спорів між ОСС:} Розгляд спорів здійснюється на засіданні СУД із запрошенням представників сторін спору. Сторони мають право представити свої позиції та аргументи. Рішення СУД оформлюється письмово та надсилається сторонам спору.

    5.3.3. \textbf{Дисциплінарні процедури щодо членів ОСС:} Рішення про застосування заходів впливу (попередження, визнання дій неправомірними, відсторонення, припинення повноважень) приймається за результатами розгляду справи про порушення, під час якого забезпечується право відповідного члена ОСС на захист (бути повідомленим про суть звинувачень, надавати пояснення, докази). Підставами для прийняття СУД рішення про \textbf{припинення повноважень} члена ОСС (відповідно до п. 4.6.1) є:

        \begin{enumerate}[label=\roman*)]
            \item Грубе порушення Статуту Університету, Положення про ОСС або положення про відповідний ОСС, що завдало істотної шкоди студентському самоврядуванню;
            \item Систематичне (більше 2 разів протягом семестру) невиконання обов'язкових рішень СУД без поважних причин;
            \item Вчинення дій, що дискредитують органи студентського самоврядування або підривають їх авторитет;
            \item Використання статусу члена ОСС у власних інтересах або в інтересах третіх осіб, що суперечить цілям ОСС (доведений конфлікт інтересів);
            \item Нецільове використання коштів або майна ОСС (підтверджене результатами перевірки СУД або іншого уповноваженого органу).
        \end{enumerate}

    \subsubsection*{5.3.4. Особливості розгляду скарг на дії/бездіяльність Адміністрації Університету}
        5.3.4.1. Скарги студентів на дії/бездіяльність посадових осіб або структурних підрозділів Адміністрації Університету подаються до СУД у письмовому вигляді з чітким викладенням суті проблеми та обґрунтуванням порушення прав чи інтересів студента.

        5.3.4.2. При розгляді такої скарги СУД має право:

            \begin{enumerate}[label=\alph*)]
                \item Направляти офіційні запити до відповідної посадової особи/структурного підрозділу Адміністрації для отримання пояснень, інформації та документів у строки, встановлені законодавством про звернення громадян;
                \item Запрошувати представників Адміністрації на свої засідання для надання пояснень та обговорення скарги (за згодою представника Адміністрації).
            \end{enumerate}

        5.3.4.3. За результатами розгляду скарги на Адміністрацію СУД може прийняти одне або декілька з таких рішень:

            \begin{enumerate}[label=\alph*)]
                \item Визнати скаргу обґрунтованою (повністю або частково) або необґрунтованою;
                \item Надати рекомендації відповідній посадовій особі/структурному підрозділу Адміністрації щодо усунення порушення прав чи інтересів студента та/або вдосконалення відповідних процедур;
                \item Звернутися з офіційним поданням або інформаційним листом до Ректора Університету щодо виявленого порушення або системної проблеми;
                \item Ініціювати розгляд питання представниками студентства у Вченій раді Університету;
                \item Оприлюднити знеособлену інформацію про типові порушення або результати розгляду суспільно значущих справ (за рішенням СУД та з дотриманням законодавства про захист персональних даних).
            \end{enumerate}

        5.3.4.4. СУД інформує скаржника та, за необхідності, відповідну посадову особу/структурний підрозділ Адміністрації про результати розгляду скарги.

\subsection*{5.4. Документообіг та оприлюднення}
\addcontentsline{toc}{subsection}{5.4. Документообіг та оприлюднення}
    5.4.1. Хід засідань СУД фіксується у протоколі, який веде Секретар СУД. Протокол підписується Головою СУД та Секретарем СУД.

    5.4.2. Рішення СУД оформлюються окремими документами або як частина протоколу засідання.

    5.4.3. Протоколи засідань (за винятком частин, що містять інформацію з обмеженим доступом) та рішення СУД підлягають оприлюдненню на офіційних інформаційних ресурсах СУД (якщо такі створені) та/або ОСС Університету протягом 5 (п'яти) робочих днів з дня проведення засідання/прийняття рішення. 
\section*{Розділ VI. Взаємодія СУД з іншими органами та підзвітність}

\subsection*{6.1. Взаємодія з Конференцією студентів Університету (КСУ)}
    6.1.1. СУД є підзвітною та підконтрольною виключно КСУ.

    6.1.2. СУД звітує про свою діяльність перед КСУ за вимогою КСУ, але не рідше одного разу на рік. Форма та зміст звіту визначаються СУД за погодженням з КСУ (або її робочим органом).

    6.1.3. КСУ має право заслуховувати позачергові звіти або інформацію Голови чи членів СУД з окремих питань діяльності.

    6.1.4. КСУ може надавати СУД рекомендації або запити щодо напрямків її роботи чи необхідності розгляду певних питань. Такі рекомендації/запити не є обов'язковими дорученнями і приймаються СУД до розгляду в межах її незалежності та компетенції.

    6.1.5. Рішення КСУ, що стосуються діяльності СУД (затвердження Положення, обрання/припинення повноважень членів, затвердження звітів тощо), є обов'язковими для СУД.

\subsection*{6.2. Взаємодія з іншими органами студентського самоврядування (ОСС)}
    6.2.1. СУД взаємодіє з СР КАІ, СР СМ, ЦВКс, СРФ та іншими ОСС з метою здійснення своїх контрольних та наглядових функцій.

    6.2.2. СУД має право запитувати будь-яку інформацію та документацію (протоколи, рішення, звіти, фінансові документи тощо), необхідну для виконання своїх повноважень, від будь-якого ОСС та його посадових осіб.

    6.2.3. Відповідний ОСС та його посадові особи зобов'язані надати запитувану СУД інформацію та/або документацію у повному обсязі та у строк, встановлений СУД, який не може бути меншим за 3 (три) робочі дні, якщо інший термін не обґрунтований невідкладністю питання.

    6.2.4. СУД надає ОСС обов'язкові до розгляду рекомендації щодо усунення виявлених порушень, а також може надавати висновки та роз'яснення щодо застосування нормативних актів.

    6.2.5. СУД інформує відповідні ОСС про результати розгляду скарг на їхню діяльність та прийняті рішення.

\subsection*{6.3. Взаємодія з Адміністрацією Університету}
    6.3.1. СУД взаємодіє з посадовими особами та структурними підрозділами Адміністрації Університету переважно в рамках розгляду скарг та звернень студентів на дії/бездіяльність Адміністрації, а також при здійсненні контролю за дотриманням прав студентів.

    6.3.2. СУД має право направляти офіційні запити до посадових осіб та структурних підрозділів Адміністрації Університету для отримання інформації, документів та пояснень, необхідних для розгляду скарг або здійснення своїх функцій. Адміністрація надає відповідь у строки, встановлені законодавством України про звернення громадян.

    6.3.3. За результатами розгляду скарг на Адміністрацію або виявлення системних проблем, СУД може надавати рекомендації відповідним посадовим особам або структурним підрозділам Адміністрації щодо усунення порушень або вдосконалення роботи.

    6.3.4. СУД може ініціювати проведення спільних засідань, консультацій або робочих зустрічей з представниками Адміністрації для обговорення питань, що належать до її компетенції.

    6.3.5. У разі систематичного невиконання рекомендацій СУД з боку посадових осіб чи структурних підрозділів Адміністрації або ненадання відповіді на запити СУД, Голова СУД має право звернутися з відповідним поданням до Ректора Університету або ініціювати розгляд цього питання представниками студентства у Вченій раді Університету.

\subsection*{6.4. Оскарження рішень СУД до Конференції студентів Університету}
    6.4.1. Рішення СУД з наступних питань можуть бути оскаржені до КСУ:
        \begin{enumerate}[label=\alph*)]
            \item Прийняття рішення про припинення повноважень члена ОСС (п. 4.6.1);
            \item Прийняття рішення про відсторонення члена ОСС від виконання повноважень (п. 4.6.2.в);
            \item Визнання неправомірним і скасування акту / рішення органу ОСС (п. 4.3.3.г);
            \item Винесення органу ОСС обов'язкового припису (як за результатами розгляду скарг (п. 4.3.3), так і за результатами контролю (п. 4.1.3));
            \item Надання офіційного тлумачення норм Положення про ОСС (п. 4.5);
            \item Прийняття обов'язкового рішення щодо розмежування компетенції у спорах між ОСС (п. 4.4.2).
        \end{enumerate}

    6.4.2. Право на оскарження рішення СУД до КСУ мають:
        \begin{enumerate}[label=\alph*)]
            \item Особа або орган ОСС, щодо якого СУД прийняла оскаржуване рішення;
            \item Скаржник, якщо він не погоджується з рішенням СУД за результатами розгляду його скарги (у випадках, коли таке рішення СУД підлягає оскарженню згідно п. 6.4.1);
            \item Інший орган ОСС, якщо оскаржуване рішення СУД безпосередньо порушує його права чи законні інтереси.
        \end{enumerate}

    6.4.3. Апеляція на рішення СУД подається у письмовій формі на ім'я Спікера КСУ протягом 1 (одного) місяця з дня офіційного оприлюднення або отримання копії відповідного рішення СУД.

    6.4.4. Подання апеляції не зупиняє дію оскаржуваного рішення СУД, якщо КСУ (або її Спікер до засідання КСУ) не прийме іншого рішення.

    6.4.5. Спікер КСУ передає апеляцію на розгляд тимчасової апеляційної комісії, що створюється КСУ. До складу цієї комісії не можуть входити члени СУД, особи, щодо яких було прийнято оскаржуване рішення, та особи, які подали скаргу/апеляцію.

    6.4.6. Апеляційна комісія вивчає матеріали справи, заслуховує пояснення сторін (апелянта та представника СУД) і надає свій висновок та проєкт рішення на розгляд пленарного засідання КСУ.

    6.4.7. За результатами розгляду апеляції КСУ більшістю голосів від загального складу делегатів може прийняти одне з таких рішень:
        \begin{enumerate}[label=\alph*)]
            \item Залишити рішення СУД без змін, а апеляцію без задоволення;
            \item Скасувати рішення СУД повністю або в частині;
            \item Змінити рішення СУД;
            \item Направити справу на новий розгляд до СУД з наданням відповідних вказівок.
        \end{enumerate}
        
    6.4.8. Рішення КСУ за результатами розгляду апеляції є остаточним в системі студентського самоврядування. 
\section*{Розділ VII. Прикінцеві та перехідні положення}
\addcontentsline{toc}{section}{Розділ VII. Прикінцеві та перехідні положення}

\subsection*{7.1. Порядок внесення змін та доповнень}
\addcontentsline{toc}{subsection}{7.1. Порядок внесення змін та доповнень}
    7.1.1. Зміни та доповнення до цього Положення вносяться виключно Конференцією студентів Університету (КСУ) більшістю голосів від загального складу її делегатів.

    7.1.2. Право ініціювати внесення змін та доповнень до цього Положення мають Студентська уповноважена делегація (СУД) або група делегатів КСУ кількістю не менше однієї п'ятої (1/5) від загального складу КСУ.

    7.1.3. Проєкт змін та доповнень до цього Положення підлягає попередньому оприлюдненню та обговоренню в порядку, встановленому Регламентом КСУ.

\subsection*{7.2. Тлумачення норм Положення}
\addcontentsline{toc}{subsection}{7.2. Тлумачення норм Положення}
    7.2.1. У разі виникнення потреби в роз'ясненні окремих норм цього Положення, офіційне тлумачення надається Студентською уповноваженою делегацією (СУД) за власною ініціативою або за запитом інших ОСС чи групи студентів.

    7.2.2. Тлумачення, надане СУД, набуває чинності з моменту його оприлюднення. Конференція студентів Університету (КСУ) на своєму найближчому засіданні може переглянути, змінити або скасувати таке тлумачення за власною ініціативою або за зверненням зацікавлених осіб/органів. Після розгляду КСУ рішення щодо тлумачення є остаточним.

\subsection*{7.3. Набуття чинності}
\addcontentsline{toc}{subsection}{7.3. Набуття чинності}
    7.3.1. Це Положення затверджується Конференцією студентів Університету (КСУ) більшістю голосів від загального складу її делегатів.

    7.3.2. Це Положення набуває чинності з дня його офіційного оприлюднення на інформаційних ресурсах ОСС Університету після його затвердження КСУ.

\subsection*{7.4. Перехідні положення}
\addcontentsline{toc}{subsection}{7.4. Перехідні положення}
    7.4.1. Питання, пов'язані з першим формуванням складу СУД після затвердження цього Положення (строки проведення виборів, організаційні аспекти тощо), вирішуються окремими рішеннями КСУ. 

\end{document} 