\section*{Розділ II. Членство, порядок формування та припинення повноважень}
\addcontentsline{toc}{section}{Розділ II. Членство, порядок формування та припинення повноважень}

\subsection*{2.1. Склад та строк повноважень СУД}
\addcontentsline{toc}{subsection}{2.1. Склад та строк повноважень СУД}
    2.1.1. Кількісний склад СУД визначається Конференцією студентів Університету (КСУ) перед кожними виборами та становить не менше 3 (трьох) та не більше 7 (семи) осіб. Для забезпечення ефективного прийняття рішень рекомендується обирати непарну кількість членів СУД.

    2.1.2. Строк повноважень членів СУД становить 2 (два) роки. Це забезпечує стабільність роботи органу, накопичення інституційної пам'яті та досвіду, що є критично важливим для виконання контрольних та наглядових функцій. КСУ може розглянути запровадження механізму часткової ротації складу СУД щороку для поєднання стабільності та оновлення.

\subsection*{2.2. Вимоги до кандидатів та членів СУД}
\addcontentsline{toc}{subsection}{2.2. Вимоги до кандидатів та членів СУД}
    2.2.1. Членом СУД може бути обраний студент (курсант) денної форми навчання Державного університету ``Київський авіаційний інститут'', який відповідає таким вимогам:

        \begin{enumerate}[label=\alph*)]
            \item Навчається на другому або старшому курсі (для бакалаврату) або на будь-якому курсі магістратури/аспірантури;
            \item Має досвід роботи в органах студентського самоврядування не менше одного повного строку повноважень на виборній посаді або одного року як член ОСС;
            \item Ознайомлений з нормативно-правовою базою діяльності ОСС та Університету.
        \end{enumerate}

    2.2.2. \textbf{Несумісність посад:} Член СУД не може одночасно:

        \begin{enumerate}[label=\alph*)]
            \item Обіймати будь-яку посаду (включаючи членство в комітетах, робочих групах тощо) в будь-якому органі студентського самоврядування;
            \item Обіймати будь-яку оплачувану або адміністративну посаду в Університеті (включаючи роботу в деканатах, відділах, лабораторіях, на кафедрах тощо, крім випадків виробничої практики, передбаченої навчальним планом);
            \item Бути членом політичної партії або обіймати керівну посаду в молодіжному крилі політичної партії.
        \end{enumerate}

    2.2.3. Суміщення статусу члена СУД зі статусом делегата КСУ допускається.

\subsection*{2.3. Порядок формування СУД}
\addcontentsline{toc}{subsection}{2.3. Порядок формування СУД}
    2.3.1. Члени СУД обираються КСУ шляхом таємного голосування.

    2.3.2. Право висування кандидатів у члени СУД має кожен студент (курсант) Університету, який відповідає вимогам, визначеним у пункті 2.2.1 цього Положення. Висування може здійснюватися шляхом самовисування або шляхом подання кандидатури групою студентів.

    2.3.3. Порядок висування кандидатів, передвиборчої агітації, процедура голосування та встановлення результатів виборів членів СУД визначаються Регламентом КСУ та/або окремим рішенням КСУ, що приймається перед оголошенням виборів. ЦВКс забезпечує організаційно-технічну підтримку виборів до СУД за дорученням КСУ.

    2.3.4. Обраними до складу СУД вважаються кандидати, які набрали найбільшу кількість голосів делегатів КСУ відповідно до встановленої КСУ процедури та квоти.

\subsection*{2.4. Припинення повноважень членів СУД}
\addcontentsline{toc}{subsection}{2.4. Припинення повноважень членів СУД}
    2.4.1. Повноваження члена СУД припиняються у разі:

        \begin{enumerate}[label=\alph*)]
            \item Закінчення строку повноважень;
            \item Подання особистої заяви про складення повноважень;
            \item Припинення статусу студента (курсанта) Університету (завершення навчання, відрахування тощо);
            \item Набрання законної сили обвинувальним вироком суду щодо нього;
            \item Визнання його недієздатним або обмежено дієздатним;
            \item Його смерті;
            \item Систематичного невиконання обов'язків члена СУД без поважних причин (більше 3 пропусків засідань СУД поспіль або більше 5 протягом семестру);
            \item Порушення вимог несумісності посад, визначених у пункті 2.2.2 цього Положення;

            \item Висловлення недовіри Конференцією студентів Університету.
        \end{enumerate}

    2.4.2. Рішення про дострокове припинення повноважень члена СУД \textbf{за його особистою заявою, у зв'язку із систематичним невиконанням обов'язків або порушенням вимог несумісності посад} приймається самою СУД більшістю голосів від її загального складу та підлягає затвердженню КСУ на найближчому засіданні. У випадку \textbf{висловлення недовіри Конференцією студентів Університету}, повноваження припиняються з моменту прийняття відповідного рішення КСУ. В інших випадках, передбачених пунктом 2.4.1 цього Положення, повноваження припиняються автоматично.

    2.4.3. У разі дострокового припинення повноважень члена СУД, КСУ на своєму найближчому засіданні може прийняти рішення про проведення довиборів для заміщення вакантної посади на решту строку повноважень. 