\section*{Розділ IV. Повноваження та функції СУД}
\addcontentsline{toc}{section}{Розділ IV. Повноваження та функції СУД}

\subsection*{4.1. Контроль за дотриманням нормативних актів}
\addcontentsline{toc}{subsection}{4.1. Контроль за дотриманням нормативних актів}
    4.1.1. СУД здійснює постійний контроль за дотриманням усіма органами студентського самоврядування (ОСС) вимог законодавства України, Статуту Університету, Положення про ОСС, цього Положення та інших нормативних актів, що регулюють їхню діяльність.

    4.1.2. СУД має право вимагати від ОСС будь-які документи (протоколи, рішення, звіти тощо), необхідні для здійснення контрольних функцій, та отримувати пояснення від членів ОСС.

    4.1.3. У разі виявлення порушень СУД може надавати обов'язкові до розгляду рекомендації щодо їх усунення, виносити \textbf{попередження} відповідному ОСС або його члену, а також \textbf{припис} про усунення виявлених порушень у визначений строк. У випадках, передбачених цим Положенням, СУД може застосовувати інші заходи впливу.

\subsection*{4.2. Контроль за фінансовою діяльністю ОСС}
\addcontentsline{toc}{subsection}{4.2. Контроль за фінансовою діяльністю ОСС}
    4.2.1. СУД здійснює контроль за цільовим та ефективним використанням коштів ОСС.

    4.2.2. СУД розглядає та \textbf{погоджує проєкт єдиного кошторису (бюджету) ОСС} Університету перед його винесенням на затвердження КСУ. У разі відсутності погодження СУД, проєкт кошторису може бути затверджений КСУ лише за умови, якщо за нього проголосувало не менше двох третин (2/3) від загального складу делегатів КСУ.

    4.2.3. СУД розглядає звіти про виконання кошторисів ОСС та надає свої висновки КСУ.

    4.2.4. СУД має право проводити перевірки фінансової звітності ОСС, а також вибіркові перевірки первинної фінансової документації ОСС (рахунків, актів виконаних робіт тощо) у разі отримання скарг, виявлення ознак порушень або за дорученням КСУ.

\subsection*{4.3. Розгляд скарг та звернень (Функції Омбудсмена)}
\addcontentsline{toc}{subsection}{4.3. Розгляд скарг та звернень (Функції Омбудсмена)}
    4.3.1. СУД розглядає скарги та звернення студентів (курсантів) Університету щодо будь-яких аспектів діяльності ОСС, включаючи порушення нормативних актів, неетичну поведінку членів ОСС, невиконання ними своїх обов'язків, порушення прав студентів з боку ОСС тощо.

    4.3.2. СУД розглядає скарги та звернення студентів (курсантів) щодо дій або бездіяльності посадових осіб та структурних підрозділів Адміністрації Університету, що стосуються прав та інтересів студентів.

        % Детальна процедура розгляду скарг на Адміністрацію визначається Регламентом СУД.

    4.3.3. За результатами розгляду скарги на ОСС, СУД, в межах своєї компетенції, може прийняти одне або декілька з таких рішень:

        \begin{enumerate}[label=\alph*)]
            \item Визнати скаргу обґрунтованою (повністю або частково) або необґрунтованою;
            \item Визнати дії / бездіяльність відповідача (органу ОСС або його члена) неправомірними;
            \item Визнати певне право скаржника або відсутність повноважень відповідача з певного питання;
            \item Визнати неправомірним і скасувати акт / рішення відповідача (органу ОСС);
            \item Винести відповідачу (органу ОСС) припис вчинити певні дії (наприклад, усунути порушення, надати інформацію) або утриматися від певних дій;
            \item Винести відповідачу (органу ОСС) та/або його керівнику (голові) попередження;
            \item Надати рекомендації відповідачу щодо вдосконалення роботи;
            \item Передати матеріали справи на розгляд КСУ для прийняття рішення, що виходить за межі компетенції СУД.
        \end{enumerate}

    4.3.4. Порядок подання та розгляду скарг визначається Розділом V цього Положення.

\subsection*{4.4. Розгляд спорів між ОСС}
\addcontentsline{toc}{subsection}{4.4. Розгляд спорів між ОСС}
    4.4.1. СУД розглядає спори між ОСС різних рівнів або одного рівня щодо їхньої компетенції, виконання рішень або інших питань діяльності.

    4.4.2. За результатами розгляду спору СУД може прийняти \textbf{рішення щодо розмежування компетенції або порядку взаємодії}, яке є обов'язковим для виконання відповідними ОСС, за винятком питань, що належать до виключної компетенції КСУ.

    4.4.3. Процедура розгляду спорів визначається Розділом V цього Положення.

\subsection*{4.5. Офіційне тлумачення Положення про ОСС}
\addcontentsline{toc}{subsection}{4.5. Офіційне тлумачення Положення про ОСС}
    4.5.1. СУД надає офіційне тлумачення норм Положення про ОСС за запитом ОСС, членів ОСС або групи студентів.

    4.5.2. Рішення СУД щодо тлумачення норм Положення про ОСС є офіційним та обов'язковим для застосування усіма ОСС.

\subsection*{4.6. Дисциплінарні повноваження}
\addcontentsline{toc}{subsection}{4.6. Дисциплінарні повноваження}
    4.6.1. У разі виявлення грубих або систематичних порушень членом ОСС вимог нормативних актів, невиконання обов'язків або вчинення дій, що шкодять інтересам студентського самоврядування, СУД має право прийняти рішення про припинення повноважень такого члена ОСС. Перелік конкретних підстав для застосування такого заходу визначається Розділом V цього Положення. Розгляд питання про припинення повноважень члена ОСС за цими підставами здійснюється СУД як першою інстанцією.

    4.6.2. За менш значні порушення або як захід до вирішення питання про припинення повноважень, СУД може прийняти рішення про:

        \begin{enumerate}[label=\alph*)]
            \item Визнання дій / бездіяльності члена ОСС неправомірними;
            \item Винесення члену ОСС попередження;
            \item Відсторонення члена ОСС від виконання повноважень на час розгляду справи про порушення (не більше ніж на 1 місяць).
        \end{enumerate}

    4.6.3. Рішення СУД, зазначені в пп. 4.6.1 та 4.6.2.в, можуть бути оскаржені до КСУ у порядку, визначеному Розділом VI цього Положення.
    
    4.6.4. СУД має право призначати виконувача обов'язків (в.о.) посадової особи ОСС у разі:
    
        \begin{enumerate}[label=\alph*)]
            \item Прийняття рішення про припинення повноважень відповідної посадової особи (згідно з п. 4.6.1);
            \item Відсторонення відповідної посадової особи від виконання повноважень (згідно з п. 4.6.2.в);
            \item Виникнення ситуації, коли термін повноважень в.о., призначеного відповідним органом студентського самоврядування, перевищує 1 (один) місяць;
            \item Неможливості призначення в.о. у встановленому порядку відповідним органом студентського самоврядування.
        \end{enumerate}
    
    4.6.5. У рішенні про призначення в.о. посадової особи ОСС СУД визначає термін повноважень в.о., а також може встановити додаткові обмеження повноважень. За загальним правилом, в.о. має повний обсяг повноважень відповідної посадової особи, окрім права голосу в колегіальних органах. Термін повноважень в.о., призначеного СУД, не може перевищувати 3 (трьох) місяців або строку, необхідного для організації та проведення виборів нової посадової особи.
    
    4.6.6. Рішення СУД про призначення в.о. посадової особи ОСС підлягає затвердженню на найближчому засіданні Конференції студентів університету (КСУ). До затвердження КСУ, рішення СУД про призначення в.o. має тимчасову силу. У випадку відхилення КСУ призначення в.о., повноваження такого в.о. припиняються з моменту прийняття відповідного рішення КСУ.

\subsection*{4.7. Повноваження щодо ініціювання та скликання}
\addcontentsline{toc}{subsection}{4.7. Повноваження щодо ініціювання та скликання}
    4.7.1. СУД має право ініціювати перед КСУ питання про внесення змін до Положення про ОСС або до положень про інші ОСС.

    4.7.2. СУД має право ініціювати перед КСУ питання про висловлення недовіри будь-якому члену ОСС (включаючи керівників).

    4.7.3. СУД має право ініціювати скликання позачергової КСУ.

    4.7.4. СУД має право скликати позачергову КСУ у невідкладних випадках у порядку, визначеному Положенням про ОСС.

\subsection*{4.8. Інші повноваження}
\addcontentsline{toc}{subsection}{4.8. Інші повноваження}
    4.8.1. СУД визначає порядок поводження з інформацією з обмеженим доступом в системі ОСС.

    4.8.2. СУД виконує інші повноваження, передбачені Положенням про ОСС та рішеннями КСУ. 