\section*{Розділ I. Загальні положення}

\subsection*{1.1. Статус та мета Студентської уповноваженої делегації}
    1.1.1. Студентська уповноважена делегація (далі – СУД) є постійно діючим колегіальним органом студентського самоврядування Університету, який здійснює контрольну та наглядову діяльність в системі органів студентського самоврядування (ОСС) і є підзвітним та підконтрольним виключно Конференції студентів Університету (КСУ).

    1.1.2. Головною метою діяльності СУД є забезпечення дотримання ОСС усіх рівнів вимог законодавства України, Статуту Університету, Положення про органи студентського самоврядування Університету (далі – Положення про ОСС), цього Положення та інших нормативних актів, що регулюють діяльність ОСС, а також захист прав та інтересів студентів в межах системи ОСС, сприяння покращенню ефективності роботи ОСС та утвердженню демократичних принципів у їхній діяльності.

\subsection*{1.2. Принципи діяльності СУД}
    1.2.1. Діяльність СУД ґрунтується на принципах:
        \begin{enumerate}[label=\alph*)]
            \item Законності;
            \item Незалежності від Адміністрації Університету та від інших органів студентського самоврядування (СР КАІ, СР СМ, ЦВКс, СРФ тощо) при здійсненні своїх повноважень;
            \item Об'єктивності та неупередженості;
            \item Колегіальності при прийнятті рішень;
            \item Прозорості та відкритості (з урахуванням обмежень щодо інформації з обмеженим доступом);
            \item Підзвітності та підконтрольності Конференції студентів Університету.
        \end{enumerate}

\subsection*{1.3. Нормативно-правова база діяльності СУД}
    1.3.1. СУД у своїй діяльності керується Конституцією України, Законом України ``Про вищу освіту'', Статутом Університету, Положенням про ОСС, цим Положенням, рішеннями КСУ та іншими нормативно-правовими актами України.

    1.3.2. Це Положення визначає мету, принципи, порядок формування, структуру, повноваження, порядок роботи та інші аспекти діяльності СУД.

\subsection*{1.4. Офіційні рішення та заяви СУД}
    1.4.1. Офіційними рішеннями СУД є рішення, прийняті колегіально на її засіданнях у порядку, встановленому цим Положенням.

    1.4.2. Голова СУД має право робити офіційні заяви та приймати певні операційні рішення від імені СУД у випадках та порядку, передбачених Розділами III та V цього Положення.

    1.4.3. Офіційні рішення СУД, прийняті в межах її компетенції, є обов'язковими до розгляду та/або виконання відповідними органами ОСС та їх посадовими особами. 