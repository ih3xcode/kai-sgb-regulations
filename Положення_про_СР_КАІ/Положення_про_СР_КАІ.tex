\documentclass[12pt, a4paper]{article}
\usepackage[ukrainian]{babel}
\usepackage{fontspec}
\usepackage{amsmath}
\usepackage{amssymb}
\usepackage{graphicx}
\usepackage{indentfirst}
\usepackage{enumitem}
\usepackage{geometry}
\geometry{a4paper, margin=2cm}
\usepackage[nottoc]{tocbibind} % Додаємо для відображення ненумерованих розділів у змісті
\usepackage{hyperref} % Додаємо підтримку гіперпосилань
\hypersetup{
    colorlinks=true,
    linkcolor=black,      % Змінено на чорний колір для змісту
    filecolor=blue,      
    urlcolor=blue,
    pdftitle={Положення про Студентську раду Київського авіаційного інституту},
    pdfauthor={ДУ "КАІ"},
    pdfcreator={LaTeX},
    pdfproducer={LaTeX}
}

\setmainfont{Liberation Serif} % Або інший шрифт, що підтримує кирилицю

\setlength{\parskip}{1ex}
\setlength{\parindent}{1.25cm}

% Метадані документа
\title{Положення про Студентську раду Київського авіаційного інституту}
\author{Державний університет ``Київський авіаційний інститут''}
\date{\today} % Поточна дата компіляції

\begin{document}

\begin{titlepage}
    \centering
    \vspace*{\fill} % Вертикальне центрування

    {\Huge\bfseries Положення}\par % Великий жирний заголовок
    \vspace{1em} % Відступ
    {\LARGE про Студентську раду}\par % Підзаголовок
    \vspace{0.5em} % Менший відступ
    {\large Київського авіаційного інституту}\par % Повна назва

    \vspace*{\fill} % Вертикальне центрування
\end{titlepage}

% Додаємо глосарій термінів та абревіатур
\section*{Глосарій термінів та абревіатур}
\begin{description}[leftmargin=3cm,style=nextline]
    \item[СР КАІ] Студентська рада Київського авіаційного інституту -- вищий виконавчий орган студентського самоврядування.
    \item[ОСС] Органи студентського самоврядування.
    \item[КСУ] Конференція студентів Університету -- вищий представницький орган студентського самоврядування.
    \item[СУД] Студентська уповноважена делегація -- постійно діючий орган, відповідальний за контрольну діяльність в системі ОСС.
    \item[ЦВКс] Центральна виборча комісія студентів -- постійно діючий орган, відповідальний за організацію та проведення виборів до ОСС.
    \item[СР СМ] Студентська рада студмістечка -- орган, що координує діяльність ОСС гуртожитків.
    \item[СРФ/СРІ] Студентська рада факультету/інституту -- основний орган студентського самоврядування на рівні факультету/інституту.
    \item[СО] Студентські організації -- добровільні об'єднання студентів, створені за спільними інтересами.
\end{description}
\newpage

% Додаємо зміст
\renewcommand{\contentsname}{Зміст}
\tableofcontents
\newpage

% --- Підключення розділів ---
\section*{Розділ I. Загальні положення}

\subsection*{1.1. Визначення та мета студентського самоврядування}
    1.1.1. Студентське самоврядування -- це право і можливість студентів вирішувати питання навчання і побуту, захисту прав та інтересів студентів, а також брати участь в управлінні закладом вищої освіти.

    1.1.2. Метою студентського самоврядування в Університеті є створення умов для самореалізації студентів та їхньої участі в управлінні Університетом шляхом забезпечення захисту їхніх прав та інтересів, сприяння їхньому гармонійному розвитку.

    1.1.3. Поряд з органами студентського самоврядування (ОСС), як складовими офіційної системи, в Університеті можуть діяти студентські організації (СО) – добровільні об’єднання студентів, створені за спільними інтересами (культурними, науковими, спортивними, соціальними тощо) для реалізації статутних цілей цих організацій. СО діють незалежно від формальної структури та ієрархії ОСС.

\subsection*{1.2. Принципи діяльності ОСС}
    1.2.1. Діяльність органів студентського самоврядування (ОСС) ґрунтується на принципах: законності, добровільності, колегіальності, виборності та звітності, рівноправності студентів у можливості брати участь у студентському самоврядуванні, прозорості та відкритості, організаційної самостійності в межах повноважень, визначених законодавством, Статутом Університету та цим Положенням.

\subsection*{1.3. Законодавчі та нормативні підстави}
    1.3.1. Правову основу діяльності ОСС становлять Конституція України, Закон України ``Про вищу освіту'', Статут Університету, це Положення та інші нормативно-правові акти України, що стосуються питань студентського самоврядування.

    1.3.2. Це Положення визначає структуру, повноваження, порядок формування та основні засади діяльності органів студентського самоврядування Університету. Проєкт змін та доповнень до цього Положення підлягає попередньому оприлюдненню та громадському обговоренню серед студентів. Зміни та доповнення до цього Положення вносяться Конференцією студентів Університету більшістю не менше двох третин голосів від загального складу делегатів КСУ.

\subsection*{1.4. Основні завдання та сфери діяльності ОСС}
    1.4.1. Забезпечення і захист прав та інтересів студентів, зокрема стосовно організації освітнього процесу.

    1.4.2. Сприяння навчальній, науковій та творчій діяльності студентів.

    1.4.3. Сприяння створенню належних умов для проживання і відпочинку студентів у гуртожитках та соціально-побутового забезпечення.

    1.4.4. Сприяння діяльності різноманітних студентських гуртків, товариств, об\'єднань, клубів за інтересами.

    1.4.5. Організація співробітництва зі студентами інших закладів вищої освіти та молодіжними організаціями в Україні та за її межами.

    1.4.6. Сприяння працевлаштуванню студентів та випускників.

    1.4.7. Участь у вирішенні питань міжнародної мобільності студентів.

    1.4.8. Забезпечення інформаційної підтримки студентів через офіційні ресурси ОСС та інші канали комунікації.

    1.4.9. Представлення інтересів студентської спільноти Університету у взаємовідносинах з адміністрацією Університету, іншими установами та організаціями.

    1.4.10. Сприяння розвитку студентських ініціатив та співпраця з визнаними (зареєстрованими) студентськими організаціями Університету.

\subsection*{1.5. Відповідальність членів ОСС}
    1.5.1. Члени ОСС зобов'язані сумлінно виконувати свої обов'язки, діяти в інтересах студентської спільноти, дотримуватися вимог законодавства, Статуту Університету та цього Положення.

    1.5.2. Відповідальність членів ОСС перед студентською спільнотою реалізується через процедури звітності, оцінки діяльності, а також через механізми дострокового припинення повноважень, визначені Розділом IV цього Положення.

\subsection*{1.6. Право участі ОСС в управлінні Університетом}
    1.6.1. Органи студентського самоврядування мають гарантоване право брати участь в управлінні Університетом у порядку, встановленому Законом України ``Про вищу освіту'' та Статутом Університету.

    1.6.2. Основні форми участі ОСС в управлінні Університетом включають: участь в обговоренні та вирішенні питань удосконалення освітнього процесу, науково-дослідної роботи, призначення стипендій, організації дозвілля, оздоровлення, побуту та харчування студентів.

    1.6.3. Представники ОСС входять до складу Вченої ради Університету. Квоти представництва студентів у Вченій раді Університету визначаються Статутом Університету. Порядок висування кандидатів та обрання представників студентства до Вченої ради Університету встановлюється Положенням про Центральну виборчу комісію студентів (ЦВКс) та/або Регламентом КСУ. Представники ОСС відповідного рівня також входять до складу вчених рад факультетів/інститутів та можуть входити до інших робочих чи дорадчих органів Університету відповідно до квот та порядку, визначених Статутом Університету та положеннями про ці органи. Порядок обрання (делегування) представників студентів до вчених рад факультетів/інститутів визначається Положенням про Центральну виборчу комісію студентів (ЦВКс) з урахуванням вимог Статуту Університету та положень про відповідні факультети/інститути.

    1.6.4. Адміністрація Університету та її структурні підрозділи не мають права втручатися у діяльність ОСС, крім випадків, передбачених законодавством України. Рішення адміністрації, що стосуються прав та інтересів студентів, приймаються з урахуванням пропозицій відповідних ОСС у порядку, визначеному Розділом VII цього Положення.
\section*{Розділ II. Структура та склад СР КАІ}
\addcontentsline{toc}{section}{Розділ II. Структура та склад СР КАІ}

\subsection*{2.1. Структура СР КАІ}
\addcontentsline{toc}{subsection}{2.1. Структура СР КАІ}
    2.1.1. СР КАІ є єдиним колегіальним органом, що здійснює свою діяльність через засідання її членів (далі – засідання СР КАІ).

    2.1.2. Для забезпечення ефективного виконання своїх завдань та повноважень за окремими напрямами діяльності СР КАІ формує зі свого складу або із залученням інших студентів постійні або тимчасові робочі органи (комітети, комісії, департаменти, робочі групи тощо). Порядок їх формування та діяльності визначається цим Положенням та/або внутрішніми документами СР КАІ.

\subsection*{2.2. Склад СР КАІ}
\addcontentsline{toc}{subsection}{2.2. Склад СР КАІ}
    2.2.1. Членами СР КАІ з правом голосу є:

        \begin{enumerate}[label=\alph*)]
            \item Голова СР КАІ, обраний на прямих загальноуніверситетських виборах;
            \item Голови Студентських рад факультетів/інститутів Університету за посадою (ex officio) на час виконання ними повноважень голів СРФ/СРІ.
        \end{enumerate}

    2.2.2. СР КАІ має право залучати до своєї роботи інших студентів Університету денної форми навчання для виконання певних функцій (наприклад, як керівників або членів комітетів/департаментів, секретарів тощо). Такі залучені студенти беруть участь у роботі СР КАІ з правом дорадчого голосу та не вважаються членами СР КАІ у контексті прийняття рішень, що потребують голосування.

    2.2.3. Кількісний склад СР КАІ з правом голосу дорівнює кількості факультетів/інститутів в Університеті плюс одна особа (Голова СР КАІ).

\subsection*{2.3. Вимоги до членів та кандидатів}
\addcontentsline{toc}{subsection}{2.3. Вимоги до членів та кандидатів}
    2.3.1. Кандидатом на посаду Голови СР КАІ може бути будь-який студент Університету денної форми навчання, який відповідає вимогам, встановленим Положенням про Центральну виборчу комісію студентів (ЦВКс) (як правило, відсутність академічної заборгованості та діючих дисциплінарних стягнень).

    2.3.2. Голова СР КАІ не може одночасно обіймати посаду Голови Студентської ради факультету/інституту.

    2.3.3. Членом СР КАІ за посадою може бути виключно Голова СРФ/СРІ, обраний відповідно до встановленого порядку.

    2.3.4. Студенти, які залучаються до роботи СР КАІ з правом дорадчого голосу, повинні бути студентами Університету денної форми навчання.

\subsection*{2.4. Строк та порядок набуття і припинення повноважень}
\addcontentsline{toc}{subsection}{2.4. Строк та порядок набуття і припинення повноважень}
    2.4.1. Голова СР КАІ обирається студентами Університету шляхом прямого таємного голосування строком на 1 (один) рік відповідно до процедури, визначеної Положенням про ЦВКс.

    2.4.2. Повноваження Голови СР КАІ можуть бути достроково припинені за рішенням Конференції студентів Університету (КСУ) у порядку, визначеному Положенням про ОСС та/або Регламентом КСУ, або за рішенням Студентської уповноваженої делегації (СУД) у випадках та порядку, передбачених Положенням про СУД.

    2.4.3. Строк повноважень членів СР КАІ – Голів СРФ/СРІ – становить 1 (один) рік та збігається зі строком їхніх повноважень на посаді Голови СРФ/СРІ.

    2.4.4. Член СР КАІ – Голова СРФ/СРІ – набуває повноважень з моменту його обрання Головою СРФ/СРІ та припиняє їх у разі припинення повноважень Голови СРФ/СРІ.

    2.4.5. Повноваження студентів, залучених до роботи з правом дорадчого голосу, визначаються рішенням СР КАІ та можуть бути припинені достроково за рішенням СР КАІ, або у зв'язку із завершенням роботи відповідного робочого органу чи закінченням строку повноважень складу СР КАІ, що їх залучив. 
\section*{Розділ III. Повноваження СР КАІ}
\addcontentsline{toc}{section}{Розділ III. Повноваження СР КАІ}

\subsection*{3.1. Представницькі повноваження}
\addcontentsline{toc}{subsection}{3.1. Представницькі повноваження}
    3.1.1. СР КАІ представляє та захищає права та законні інтереси студентів Університету перед адміністрацією Університету та її структурними підрозділами.

    3.1.2. СР КАІ представляє студентську спільноту у Вченій раді Університету та інших органах управління Університету відповідно до квот та порядку, визначених Статутом Університету та Положенням про ОСС.

    3.1.3. СР КАІ виступає від імені студентської спільноти Університету у взаємовідносинах з органами студентського самоврядування інших закладів вищої освіти, державними органами, громадськими та іншими організаціями з питань студентського життя.

\subsection*{3.2. Виконавчі повноваження}
\addcontentsline{toc}{subsection}{3.2. Виконавчі повноваження}
    3.2.1. СР КАІ організовує виконання рішень Конференції студентів Університету (КСУ), забезпечуючи розробку необхідних заходів, координацію виконавців та контроль за реалізацією.

    3.2.2. СР КАІ реалізує основні завдання студентського самоврядування, визначені Положенням про ОСС та цим Положенням, у межах своєї компетенції.

    3.2.3. СР КАІ звітує перед КСУ про виконання покладених на неї завдань та рішень КСУ.

\subsection*{3.3. Організаційні повноваження}
\addcontentsline{toc}{subsection}{3.3. Організаційні повноваження}
    3.3.1. СР КАІ організовує та проводить загальноуніверситетські заходи навчального, наукового, культурного, спортивного, соціального та іншого характеру, спрямовані на розвиток студентів та збагачення студентського життя.

    3.3.2. СР КАІ координує роботу своїх структурних підрозділів (комітетів, департаментів тощо).

    3.3.3. СР КАІ сприяє діяльності студентських організацій, гуртків, товариств, клубів за інтересами.

    3.3.4. СР КАІ забезпечує інформаційну підтримку студентів щодо діяльності ОСС та важливих подій в Університеті.

\subsection*{3.4. Повноваження щодо погодження рішень}
\addcontentsline{toc}{subsection}{3.4. Повноваження щодо погодження рішень}
    3.4.1. СР КАІ реалізує право органів студентського самоврядування на участь в управлінні Університетом шляхом погодження рішень адміністрації Університету, що стосуються прав та інтересів студентів, у випадках та порядку, передбачених Законом України "Про вищу освіту", Статутом Університету та Положенням про ОСС.

\subsection*{3.5. Повноваження щодо СРФ/СРІ}
\addcontentsline{toc}{subsection}{3.5. Повноваження щодо СРФ/СРІ}
    3.5.1. СР КАІ здійснює координацію діяльності Студентських рад факультетів/інститутів (СРФ/СРІ), забезпечуючи єдність підходів до реалізації завдань студентського самоврядування.

    3.5.2. СР КАІ надає методичну та організаційну допомогу СРФ/СРІ.

    3.5.3. СР КАІ може розробляти та надавати СРФ/СРІ рекомендації щодо організації їхньої роботи та проведення заходів.

    3.5.4. СР КАІ здійснює контроль за діяльністю СРФ/СРІ в межах повноважень, визначених Положенням про ОСС та/або делегованих КСУ.

    3.5.5. СР КАІ реалізує повноваження щодо тимчасового управління справами студентського самоврядування факультету/інституту у випадках та порядку, визначених Положенням про ОСС.

\subsection*{3.6. Контрольні повноваження}
\addcontentsline{toc}{subsection}{3.6. Контрольні повноваження}
    3.6.1. СР КАІ здійснює внутрішній контроль за діяльністю своїх структурних підрозділів та посадових осіб.

    3.6.2. СР КАІ взаємодіє зі Студентською уповноваженою делегацією (СУД) з питань контролю за дотриманням нормативних документів ОСС та цільовим використанням ресурсів у межах своєї компетенції.

\subsection*{3.7. Обов'язковість рішень СР КАІ}
\addcontentsline{toc}{subsection}{3.7. Обов'язковість рішень СР КАІ}
    3.7.1. Рішення СР КАІ, прийняті колегіально в межах її повноважень відповідно до цього Положення та Положення про ОСС, є обов'язковими для виконання всіма членами СР КАІ, її структурними підрозділами (комітетами, департаментами тощо) та посадовими особами СР КАІ.

    3.7.2. Невиконання або неналежне виконання рішень СР КАІ є підставою для відповідальності відповідно до цього Положення та інших внутрішніх документів ОСС. 
\section*{Розділ IV. Керівні органи та посадові особи СР КАІ}
\addcontentsline{toc}{section}{Розділ IV. Керівні органи та посадові особи СР КАІ}

\subsection*{4.1. Голова Студентської ради КАІ}
\addcontentsline{toc}{subsection}{4.1. Голова Студентської ради КАІ}
    4.1.1. Студентську раду КАІ очолює Голова СР КАІ, який обирається студентами Університету на прямих виборах відповідно до Розділу II цього Положення та Положення про ЦВКс.

    4.1.2. Голова СР КАІ:

        \begin{enumerate}[label=\alph*)]
            \item Організовує роботу СР КАІ та головує на її засіданнях.
            \item Представляє СР КАІ у відносинах з органами державної влади, місцевого самоврядування, адміністрацією Університету, іншими ОСС, підприємствами, установами, організаціями.
            \item Вносить на розгляд СР КАІ пропозиції щодо кандидатур Заступника(-ів) Голови, Секретаря СР КАІ та керівників постійних комітетів/департаментів.
            \item Координує роботу Заступника(-ів) Голови, Секретаря та керівників комітетів/департаментів СР КАІ.
            \item Підписує рішення, протоколи та інші документи СР КАІ.
            \item Здійснює оперативне управління коштами та майном, що перебувають у віданні СР КАІ, підписує фінансові документи та затверджує витрати в межах кошторису, погодженого СР КАІ, та відповідно до її рішень. Це повноваження не поширюється на кошти та майно СРФ/СРІ;
            \item Забезпечує виконання рішень КСУ та СР КАІ.
            \item Звітує про свою діяльність та діяльність СР КАІ перед КСУ та СР КАІ.
            \item Виконує інші повноваження, передбачені цим Положенням та Положенням про ОСС.
        \end{enumerate}

    4.1.3. Голова СР КАІ є підзвітним Конференції студентів Університету. Повноваження Голови СР КАІ можуть бути достроково припинені за рішенням КСУ або за рішенням Студентської уповноваженої делегації (СУД) у випадках та порядку, передбачених Положенням про СУД.

    \subsubsection*{4.1.4. Призначення виконувача обов'язків Голови СР КАІ}
        4.1.4.1. У разі дострокового припинення повноважень Голови СР КАІ, його тимчасової неможливості виконувати свої обов'язки (через хворобу, відрядження тощо) або інших обставин, що унеможливлюють виконання ним своїх повноважень, призначається виконувач обов'язків (в.о.) Голови СР КАІ.
        
        4.1.4.2. В.о. Голови СР КАІ може бути призначений колегіальним рішенням СР КАІ на термін не більше 1 (одного) місяця. Для призначення в.о. на більший термін необхідне рішення КСУ або СУД.
        
        4.1.4.3. В.о. Голови СР КАІ може бути Голова СРФ/СРІ або інший студент Університету, що відповідає загальним вимогам до членів ОСС. На період виконання обов'язків Голови СР КАІ, Голова СРФ/СРІ тимчасово делегує свої повноваження заступнику в СРФ/СРІ без дострокового припинення своїх повноважень у відповідній СРФ/СРІ.
        
        4.1.4.4. В.о. Голови СР КАІ має повний обсяг повноважень Голови СР КАІ, окрім права голосу в СР КАІ як члена з правом голосу. Якщо рішення про призначення в.о. прийнято КСУ або СУД, у ньому можуть зазначатися додаткові обмеження повноважень.
        
        4.1.4.5. Передача справ в.о. Голови СР КАІ здійснюється відповідно до процедури, визначеної розділом 4.6 Положення про ОСС.

\subsection*{4.2. Заступник(-и) Голови та Секретар СР КАІ}
\addcontentsline{toc}{subsection}{4.2. Заступник(-и) Голови та Секретар СР КАІ}
    4.2.1. Голова СР КАІ може мати одного або кількох заступників та Секретаря СР КАІ.

    4.2.2. Заступник(-и) Голови та Секретар СР КАІ призначаються Головою СР КАІ з числа студентів, що беруть участь у роботі СР КАІ (з правом голосу або дорадчим), за погодженням (затвердженням) СР КАІ.

    4.2.3. Заступник Голови СР КАІ виконує обов'язки Голови СР КАІ за його відсутності або за його дорученням, координує роботу визначених комітетів/департаментів або напрямів діяльності.

    4.2.4. Секретар СР КАІ відповідає за ведення протоколів засідань СР КАІ, організацію документообігу та архіву СР КАІ.

    4.2.5. Повноваження Заступника(-ів) Голови та Секретаря СР КАІ можуть бути достроково припинені за рішенням СР КАІ.

\subsection*{4.3. Комітети/Департаменти СР КАІ}
\addcontentsline{toc}{subsection}{4.3. Комітети/Департаменти СР КАІ}
    4.3.1. Для реалізації основних завдань та напрямів діяльності СР КАІ може створювати постійні або тимчасові комітети/департаменти (наприклад: навчально-науковий, соціально-побутовий, інформаційний, культурно-масовий, спортивний, міжнародного співробітництва, фінансовий тощо).

    4.3.2. Структура, перелік та повноваження комітетів/департаментів затверджуються СР КАІ.

    4.3.3. Керівники комітетів/департаментів призначаються Головою СР КАІ з числа студентів, що беруть участь у роботі СР КАІ, за погодженням (затвердженням) СР КАІ.

    4.3.4. Керівник комітету/департаменту організовує роботу відповідного підрозділу, звітує про його діяльність перед СР КАІ та несе відповідальність за виконання покладених на підрозділ завдань.

    4.3.5. Повноваження керівника комітету/департаменту можуть бути достроково припинені за рішенням СР КАІ.

\subsection*{4.4. Відповідальність посадових осіб}
\addcontentsline{toc}{subsection}{4.4. Відповідальність посадових осіб}
    4.4.1. Заступник(-и) Голови, Секретар СР КАІ та керівники комітетів/департаментів несуть відповідальність перед СР КАІ за належне виконання своїх обов'язків.

    4.4.2. У разі систематичного невиконання обов'язків, порушення цього Положення, Положення про ОСС або вчинення дій, що шкодять репутації ОСС, зазначені посадові особи можуть бути достроково відкликані зі своїх посад за рішенням СР КАІ.
     
    \subsubsection*{4.4.3. Призначення виконувачів обов'язків інших посадових осіб СР КАІ}
        4.4.3.1. У разі дострокового припинення повноважень, тимчасової неможливості виконувати обов'язки або інших обставин, що унеможливлюють виконання повноважень Заступника(-ів) Голови, Секретаря СР КАІ або керівників комітетів/департаментів, призначення виконувача обов'язків здійснюється в тому ж порядку, що й призначення відповідних посадових осіб.
         
        4.4.3.2. Термін повноважень, обсяг повноважень та порядок передачі справ для таких в.о. визначаються рішенням про їх призначення. 
\section*{Розділ V. Порядок роботи СР КАІ}
\addcontentsline{toc}{section}{Розділ V. Порядок роботи СР КАІ}

\subsection*{5.1. Засідання СР КАІ}
\addcontentsline{toc}{subsection}{5.1. Засідання СР КАІ}
    5.1.1. Основною формою роботи СР КАІ є засідання.

    5.1.2. Засідання СР КАІ скликаються Головою СР КАІ за потребою. Засідання може бути скликане також на вимогу не менше третини від загального числа членів СР КАІ з правом голосу.

    5.1.3. Порядок денний засідання формується Головою СР КАІ з урахуванням пропозицій членів СР КАІ та доводиться до відома членів та запрошених осіб, як правило, не пізніше ніж за 2 (два) робочі дні до засідання, крім випадків невідкладного розгляду питань.

    5.1.4. Засідання СР КАІ можуть проводитися в очному, дистанційному (з використанням офіційно визначених Університетом або СР КАІ засобів відеоконференцзв'язку) або змішаному форматі. Рішення про формат засідання приймає Голова СР КАІ або ініціатори скликання засідання.

    5.1.5. При проведенні засідання у дистанційному або змішаному форматі Секретар СР КАІ (або інша відповідальна особа) забезпечує технічну можливість для ідентифікації учасників, їхньої участі в обговоренні та голосуванні в режимі реального часу.

\subsection*{5.2. Правомочність засідання (Кворум)}
\addcontentsline{toc}{subsection}{5.2. Правомочність засідання (Кворум)}
    5.2.1. Засідання СР КАІ є правомочним, якщо на ньому присутні (особисто або дистанційно) більше половини від загального числа членів СР КАІ з правом голосу.

\subsection*{5.3. Прийняття рішень}
\addcontentsline{toc}{subsection}{5.3. Прийняття рішень}
    5.3.1. Рішення СР КАІ приймаються на її правомочних засіданнях шляхом голосування членів з правом голосу.

    5.3.2. Рішення вважається прийнятим, якщо за нього проголосувало більше половини від \textbf{загального} числа членів СР КАІ з правом голосу.

    5.3.3. Основною формою голосування на засіданнях СР КАІ є відкрите голосування. Таємне голосування проводиться у випадках, прямо передбачених цим Положенням, або за рішенням більшості від присутніх членів СР КАІ з правом голосу.

    5.3.4. Голосування з кадрових питань, що належать до компетенції СР КАІ (погодження кандидатур Заступників Голови, Секретаря, керівників комітетів, відкликання цих посадових осіб), проводиться таємно, якщо цього вимагає хоча б один член СР КАІ з правом голосу.

    5.3.5. На засіданні головує Голова СР КАІ, а за його відсутності – один із Заступників Голови СР КАІ за його дорученням або за рішенням СР КАІ. Головуючий на засіданні бере участь у голосуванні на рівних підставах з іншими членами СР КАІ.

\subsection*{5.4. Ведення протоколу}
\addcontentsline{toc}{subsection}{5.4. Ведення протоколу}
    5.4.1. Хід кожного засідання СР КАІ фіксується у протоколі, який веде Секретар СР КАІ або інша особа за дорученням головуючого.

    5.4.2. Протокол засідання повинен містити: дату, місце (або формат проведення), час початку та закінчення засідання; список присутніх членів СР КАІ (із зазначенням тих, хто має право голосу) та запрошених осіб; порядок денний; стислий виклад обговорення питань порядку денного; результати голосування по кожному питанню (``за'', ``проти'', ``утримались'') та повний текст прийнятих рішень.

    5.4.3. Протокол підписується головуючим на засіданні та секретарем засідання не пізніше 3 (трьох) робочих днів після проведення засідання.

    5.4.4. Протоколи засідань СР КАІ є офіційними документами, зберігаються в СР КАІ протягом строку повноважень відповідного складу та передаються наступному складу або до архіву ОСС в установленому порядку. Копії протоколів або витяги з них надаються за запитами КСУ, СУД, адміністрації Університету.

\subsection*{5.5. Планування та звітність}
\addcontentsline{toc}{subsection}{5.5. Планування та звітність}
    5.5.1. СР КАІ здійснює свою діяльність відповідно до річного плану роботи, затвердженого на одному з перших засідань нового складу.

    5.5.2. СР КАІ регулярно звітує про свою діяльність та виконання плану роботи перед Конференцією студентів Університету у порядку та строки, визначені Положенням про ОСС та Регламентом КСУ. 
\section*{Розділ VI. Студентські ради факультетів/інститутів}
\addcontentsline{toc}{section}{Розділ VI. Студентські ради факультетів/інститутів}

\subsection*{6.1. Статус та завдання СРФ/СРІ}
\addcontentsline{toc}{subsection}{6.1. Статус та завдання СРФ/СРІ}
    6.1.1. Студентська рада факультету/інституту (далі – СРФ/СРІ) є основним органом студентського самоврядування на рівні факультету/інституту, який діє відповідно до Закону України "Про вищу освіту", Статуту Університету, Положення про ОСС та цього Положення.

    6.1.2. СРФ/СРІ є підзвітною та підконтрольною студентам факультету/інституту (через процедуру виборів та звітування Голови), Конференції студентів Університету (КСУ) та Студентській раді КАІ в межах їхніх повноважень.

    6.1.3. Основним завданням СРФ/СРІ є представництво та захист прав та інтересів студентів свого факультету/інституту, а також реалізація завдань студентського самоврядування, визначених Положенням про ОСС та цим Положенням, на рівні факультету/інституту.

\subsection*{6.2. Формування та склад СРФ/СРІ}
\addcontentsline{toc}{subsection}{6.2. Формування та склад СРФ/СРІ}
    6.2.1. СРФ/СРІ очолює Голова, який обирається студентами факультету/інституту шляхом прямого таємного голосування строком на 1 (один) рік відповідно до процедури, визначеної Положенням про ЦВКс.

    6.2.2. Інші члени СРФ/СРІ залучаються до складу Головою СРФ/СРІ на підставі особистої заяви студента денної форми навчання відповідного факультету/інституту.

    6.2.3. Голова СРФ/СРІ видає розпорядження про включення студента до складу СРФ/СРІ, якщо відсутні підстави для відмови, визначені у п. 6.2.4.

    6.2.4. Рішення про відмову студенту у включенні до складу СРФ/СРІ приймається \textbf{колегіально СРФ/СРІ} більшістю голосів від загального складу виключно з таких підстав:

        \begin{enumerate}[label=\alph*)]
            \item Невідповідність кандидата вимогам (не є студентом денної форми навчання факультету/інституту);
            \item Наявність у кандидата на момент розгляду питання діючого дисциплінарного стягнення;
            \item Попереднє виключення кандидата зі складу ОСС за грубе порушення нормативних документів ОСС або вчинення дій, що завдали значної шкоди репутації ОСС.
        \end{enumerate}

    6.2.5. Рішення СРФ/СРІ про відмову у включенні до складу може бути оскаржене студентом до Студентської уповноваженої делегації (СУД) протягом 7 (семи) календарних днів з моменту його прийняття.

    6.2.6. Загальна кількість членів СРФ/СРІ (включаючи Голову) з правом голосу обмежується \textbf{15 (п'ятнадцятьма) особами}.

    6.2.7. Студенти, залучені до складу СРФ/СРІ понад встановлений ліміт (15 осіб), беруть участь у роботі з правом \textbf{дорадчого голосу} та не можуть претендувати на додаткові бали чи інші заохочення, передбачені для членів ОСС з правом голосу.

    6.2.8. Секретар СРФ/СРІ веде актуальний реєстр членів СРФ/СРІ із зазначенням дати включення до складу та чітким розмежуванням членів з правом голосу та членів з правом дорадчого голосу.

    6.2.9. У виняткових випадках, за наявності обґрунтованої потреби, Конференція студентів Університету (КСУ) за поданням Студентської ради КАІ (СР КАІ) або Студентської уповноваженої делегації (СУД) може прийняти рішення про збільшення максимальної кількості членів відповідної СРФ/СРІ з правом голосу понад ліміт, встановлений п. 6.2.6 цього Положення. Таке рішення має визначати новий ліміт та, за потреби, термін його дії.

\subsection*{6.3. Структура СРФ/СРІ}
\addcontentsline{toc}{subsection}{6.3. Структура СРФ/СРІ}
    6.3.1. Обов'язковими посадами в СРФ/СРІ є:

        \begin{enumerate}[label=\alph*)]
            \item Голова СРФ/СРІ;
            \item Заступник (або заступники) Голови СРФ/СРІ — виконує обов'язки Голови за його відсутності або дорученням, координує роботу визначених напрямів або структурних підрозділів;
            \item Секретар СРФ/СРІ — відповідає за ведення протоколів засідань, діловодство, ведення реєстру членів та архіву СРФ/СРІ.
        \end{enumerate}

    6.3.2. Заступник(-и) Голови та Секретар СРФ/СРІ призначаються Головою СРФ/СРІ з числа членів СРФ/СРІ з правом голосу.

    6.3.3. В СРФ/СРІ обов'язково створюються наступні структурні підрозділи (відділи/комітети):

        \begin{enumerate}[label=\alph*)]
            \item Медіавідділ: відповідає за створення та поширення інформаційних матеріалів про діяльність СРФ/СРІ та ОСС, ведення офіційних сторінок СРФ/СРІ у соціальних мережах, внутрішні та зовнішні комунікації.
            \item Організаційний відділ: відповідає за розробку планів заходів СРФ/СРІ, логістичне забезпечення, організацію та проведення освітніх, культурних, спортивних, соціальних та інших заходів на рівні факультету/інституту.
        \end{enumerate}

    6.3.4. СРФ/СРІ за поданням Голови може створювати інші постійні або тимчасові структурні підрозділи (відділи, комітети, проєктні групи) для виконання специфічних завдань. Рішення про створення або ліквідацію таких підрозділів приймається колегіально СРФ/СРІ.

    6.3.5. Керівники структурних підрозділів призначаються Головою СРФ/СРІ з числа членів СРФ/СРІ (переважно з правом голосу).

\subsection*{6.4. Повноваження Голови СРФ/СРІ}
\addcontentsline{toc}{subsection}{6.4. Повноваження Голови СРФ/СРІ}
    6.4.1. Голова СРФ/СРІ, окрім повноважень, визначених у п. 6.1.2 та інших розділах цього Положення:

        \begin{enumerate}[label=\alph*)]
            \item Здійснює загальне керівництво діяльністю СРФ/СРІ.
            \item Забезпечує формування складу СРФ/СРІ шляхом розгляду заяв студентів та видання розпоряджень про включення до складу відповідно до п. 6.2.2-6.2.3.
            \item Ініціює перед СРФ/СРІ розгляд питання про відмову у включенні до складу відповідно до п. 6.2.4.
            \item Призначає Заступника(-ів) Голови, Секретаря та керівників структурних підрозділів СРФ/СРІ.
            \item Організовує та контролює роботу членів та структурних підрозділів СРФ/СРІ.
            \item Скликає та веде засідання СРФ/СРІ.
            \item Видає розпорядження організаційно-виконавчого характеру в межах своєї компетенції, що не належать до виключної компетенції колегіального органу СРФ/СРІ.
            \item Вносить на розгляд СРФ/СРІ подання про дострокове припинення повноважень (виключення) члена СРФ/СРІ з підстав, визначених цим Положенням.
            \item Представляє СРФ/СРІ у відносинах з деканатом/дирекцією, СР КАІ, іншими органами та організаціями.
            \item Підписує протоколи засідань, рішення та інші офіційні документи СРФ/СРІ.
            \item Звітує про свою діяльність та діяльність СРФ/СРІ перед студентами факультету/інституту та СР КАІ.
        \end{enumerate}

\subsection*{6.5. Припинення повноважень членів СРФ/СРІ}
\addcontentsline{toc}{subsection}{6.5. Припинення повноважень членів СРФ/СРІ}
    6.5.1. Повноваження члена СРФ/СРІ (включаючи Голову) припиняються у зв'язку із закінченням строку, на який його було обрано/призначено (1 рік).

    6.5.2. Повноваження члена СРФ/СРІ (включаючи Голову) припиняються достроково та автоматично у разі:

        \begin{enumerate}[label=\alph*)]
            \item Подання особистої заяви про складання повноважень;
            \item Втрати статусу студента Університету;
            \item Набрання законної сили обвинувальним вироком суду щодо нього;
            \item Смерті або визнання його безвісно відсутнім чи померлим.
        \end{enumerate}

    % --- Дострокове припинення повноважень Голови СРФ/СРІ ---
    6.5.3. Повноваження \textbf{Голови} СРФ/СРІ можуть бути достроково припинені (він може бути відкликаний/усунутий з посади) також за рішенням:

        \begin{enumerate}[label=\alph*)]
            \item Студентів факультету/інституту (відкликання виборцями) у порядку, визначеному Положенням про ЦВКс;
            \item Конференції студентів Університету (КСУ) за висловленням недовіри або з інших підстав, передбачених Положенням про ОСС, у порядку, визначеному Регламентом КСУ та/або Положенням про ОСС;
            \item Студентської уповноваженої делегації (СУД) у випадках та порядку, передбачених Положенням про СУД.
        \end{enumerate}

    % --- Дострокове припинення повноважень інших членів СРФ/СРІ ---
    6.5.4. Повноваження члена СРФ/СРІ (\textbf{окрім Голови}) можуть бути достроково припинені за рішенням СРФ/СРІ (прийнятим колегіально більшістю голосів від загального складу) за поданням Голови СРФ/СРІ у разі:

        \begin{enumerate}[label=\alph*)]
            \item Систематичного невиконання обов'язків члена СРФ/СРІ, зокрема неучасті у засіданнях чи роботі структурного підрозділу;
            \item Грубого порушення нормативних документів ОСС або Статуту Університету;
            \item Вчинення дій, що завдали значної шкоди репутації ОСС Університету.
        \end{enumerate}

    6.5.5. Перед прийняттям рішення відповідно до п. 6.5.4 СРФ/СРІ зобов'язана надати можливість члену, стосовно якого розглядається питання, надати пояснення.

    6.5.6. Рішення СРФ/СРІ про дострокове припинення повноважень члена (окрім Голови) може бути оскаржене до Студентської уповноваженої делегації (СУД) протягом 7 (семи) календарних днів з моменту його прийняття.

    \subsubsection*{6.5.7. Призначення виконувача обов'язків Голови СРФ/СРІ}
        6.5.7.1. У разі дострокового припинення повноважень Голови СРФ/СРІ, його тимчасової неможливості виконувати свої обов'язки (через хворобу, відрядження тощо) або інших обставин, що унеможливлюють виконання ним своїх повноважень, призначається виконувач обов'язків (в.о.) Голови СРФ/СРІ.
        
        6.5.7.2. В.о. Голови СРФ/СРІ може бути призначений колегіальним рішенням СРФ/СРІ на термін не більше 1 (одного) місяця. Для призначення в.о. на більший термін необхідне рішення КСУ або СУД за поданням СР КАІ.
        
        6.5.7.3. В.о. Голови СРФ/СРІ має повний обсяг повноважень Голови СРФ/СРІ, окрім права голосу в СР КАІ як члена з правом голосу та права голосу в СРФ/СРІ. Якщо рішення про призначення в.о. прийнято КСУ або СУД, у ньому можуть зазначатися додаткові обмеження повноважень.
        
        6.5.7.4. Передача справ в.о. Голови СРФ/СРІ здійснюється відповідно до процедури, визначеної розділом 4.6 Положення про ОСС.
        
    \subsubsection*{6.5.8. Призначення виконувачів обов'язків інших посадових осіб СРФ/СРІ}
        6.5.8.1. У разі дострокового припинення повноважень, тимчасової неможливості виконувати обов'язки або інших обставин, що унеможливлюють виконання повноважень Заступника(-ів) Голови, Секретаря СРФ/СРІ або керівників структурних підрозділів, призначення виконувача обов'язків здійснюється в тому ж порядку, що й призначення відповідних посадових осіб.
        
        6.5.8.2. Термін повноважень, обсяг повноважень та порядок передачі справ для таких в.о. визначаються рішенням про їх призначення.

\subsection*{6.6. Основні повноваження та порядок роботи СРФ/СРІ}
\addcontentsline{toc}{subsection}{6.6. Основні повноваження та порядок роботи СРФ/СРІ}
    6.6.1. СРФ/СРІ виконує завдання та повноваження, визначені у п. 6.1.3. Основні напрями діяльності включають, але не обмежуються:

        \begin{enumerate}[label=\alph*)]
            \item Представництво та захист прав студентів факультету/інституту перед адміністрацією факультету/інституту та іншими структурами.
            \item Участь у Вченій раді факультету/інституту.
            \item Сприяння покращенню якості освітнього процесу на факультеті/інституті.
            \item Організація та проведення заходів для студентів факультету/інституту.
            \item Інформаційне забезпечення студентів факультету/інституту.
            \item Взаємодія з СР КАІ та іншими ОСС.
        \end{enumerate}
        
    6.6.2. Основною формою роботи СРФ/СРІ є засідання, які проводяться за потребою, але не рідше одного разу на місяць протягом навчального семестру.

    6.6.3. У засіданні СРФ/СРІ беруть участь з правом голосу члени СРФ/СРІ в межах встановленої квоти (до 15 осіб) та з правом дорадчого голосу - інші залучені члени. Засідання є правомочним, якщо на ньому присутні більше половини від складу членів з правом голосу.

    6.6.4. Рішення СРФ/СРІ приймаються більшістю голосів від загального складу членів з правом голосу, якщо інше не передбачено цим Положенням. Процедурні питання можуть вирішуватися більшістю від присутніх членів з правом голосу.

    6.6.5. \textbf{Виключно колегіально} СРФ/СРІ приймає рішення з таких питань:

        \begin{enumerate}[label=\alph*)]
            \item Відмова у включенні студента до складу СРФ/СРІ (п. 6.2.4);

            \item Дострокове припинення повноважень (виключення) члена СРФ/СРІ (п. 6.5.4);

            \item Створення або ліквідація структурних підрозділів СРФ/СРІ (окрім обов'язкових);
            \item Затвердження плану роботи та звіту про діяльність СРФ/СРІ;
            \item Затвердження позиції СРФ/СРІ щодо погодження рішень адміністрації факультету/інституту з питань, що належать до компетенції ОСС (напр., щодо ОПП, відрахування тощо);
            \item Інші питання, віднесені до виключної компетенції СРФ/СРІ цим Положенням або за рішенням самої СРФ/СРІ.
        \end{enumerate}

    6.6.6. Хід засідання фіксується у протоколі, який веде Секретар СРФ/СРІ.

    6.6.7. СРФ/СРІ здійснює свою діяльність відповідно до плану роботи, затвердженого колегіально на початку навчального року.

    6.6.8. Внутрішні спори та суперечності між членами СРФ/СРІ або між Головою та членами СРФ/СРІ вирішуються шляхом обговорення та прийняття колегіального рішення на засіданні СРФ/СРІ. У разі неможливості дійти згоди, будь-яка зі сторін може звернутися до СР КАІ для надання допомоги у вирішенні спору (медіації).

    6.6.9. Рішення про зміну статусу члена СРФ/СРІ щодо права голосу (надання або позбавлення права голосу) в межах встановленого п. 6.2.6 ліміту приймається колегіально СРФ/СРІ кваліфікованою більшістю голосів - не менше 2/3 (двох третин) від загального складу членів з правом голосу. Таке рішення не може призводити до перевищення ліміту у 15 членів з правом голосу. Секретар вносить відповідні зміни до реєстру членів СРФ/СРІ.

\subsection*{6.7. Взаємодія СРФ/СРІ з СР КАІ}
\addcontentsline{toc}{subsection}{6.7. Взаємодія СРФ/СРІ з СР КАІ}
    6.7.1. СРФ/СРІ зобов'язані брати участь у координаційних заходах, що проводяться СР КАІ.

    6.7.2. СРФ/СРІ надають СР КАІ інформацію та звіти про свою діяльність у порядку та строки, встановлені СР КАІ.

    6.7.3. СРФ/СРІ розглядають рекомендації СР КАІ щодо організації своєї роботи та проведення заходів та інформують СР КАІ про результати розгляду.

    6.7.4. СРФ/СРІ сприяють виконанню рішень СР КАІ, що стосуються загальноуніверситетських питань, на рівні факультету/інституту.

\subsection*{6.8. Порядок діяльності СРФ/СРІ при реорганізації структурного підрозділу}
\addcontentsline{toc}{subsection}{6.8. Порядок діяльності СРФ/СРІ при реорганізації структурного підрозділу}
    6.8.1. Підставою для початку процедур реорганізації СРФ/СРІ є офіційний Наказ Ректора Університету або Рішення Вченої ради Університету про відповідну реорганізацію структурного підрозділу (факультету/інституту).

    6.8.2. \textbf{При перейменуванні факультету/інституту:} Існуюча СРФ/СРІ автоматично змінює свою назву відповідно до нової назви структурного підрозділу та продовжує виконувати свої повноваження до закінчення строку, на який її було обрано.

    6.8.3. \textbf{При злитті/приєднанні факультетів/інститутів:}

        \begin{enumerate}[label=\alph*)]
            \item Студентські ради факультетів/інститутів, що реорганізуються, продовжують діяльність протягом 1 (одного) місяця з дати офіційного об'єднання (перехідний період) з метою завершення поточних справ, передачі документації та активів, після чого їхні повноваження припиняються.
            \item Конференція студентів Університету (КСУ) на найближчому засіданні після офіційного об'єднання приймає рішення про доцільність створення єдиної Студентської ради для новоутвореного факультету/інституту.
            \item У разі прийняття КСУ рішення про створення нової єдиної СРФ/СРІ, Центральна виборча комісія студентів (ЦВКс) організовує та проводить вибори Голови цієї СРФ/СРІ протягом 2 (двох) місяців з дати офіційного об'єднання факультетів/інститутів. Новообраний Голова формує склад СРФ/СРІ відповідно до цього Положення.
            \item Делегати КСУ та представники в інших органах, обрані від студентів факультетів/інститутів, що реорганізуються, продовжують виконувати свої повноваження до моменту обрання нових представників від новоутвореного структурного підрозділу.
        \end{enumerate}

    6.8.4. \textbf{При розформуванні/ліквідації факультету/інституту (без злиття):}

        \begin{enumerate}[label=\alph*)]
            \item Студентська рада факультету/інституту, що ліквідується, продовжує діяльність протягом 2 (двох) тижнів з дати офіційної ліквідації (перехідний період) з метою завершення поточних справ, передачі документації та активів, після чого її повноваження припиняються.
            \item Уся документація, активи та справи розформованої СРФ/СРІ передаються до Студентської ради Київського авіаційного інституту (СР КАІ).
            \item Студенти, переведені на інші факультети/інститути, автоматично підпадають під представництво та юрисдикцію Студентських рад відповідних факультетів/інститутів, куди їх було переведено.
        \end{enumerate}

    6.8.5. \textbf{Тимчасове управління від СР КАІ:}

        \begin{enumerate}[label=\alph*)]
            \item У випадку, якщо проведення виборів Голови новоствореної СРФ/СРІ (відповідно до п. 6.8.3.в) цього Положення) є неможливим у встановлений 2-місячний термін, КСУ за поданням СР КАІ або ЦВКс може прийняти рішення про запровадження тимчасового управління справами студентського самоврядування відповідного факультету/інституту.

            \item Тимчасове управління здійснюється Тимчасовим комітетом, що формується та призначається СР КАІ.
            \item Повноваження Тимчасового комітету за замовчуванням обмежуються підтриманням базової операційної діяльності, забезпеченням мінімально необхідного представництва інтересів студентів факультету/інституту та першочерговою організацією та сприянням проведенню виборів Голови СРФ/СРІ. За окремим рішенням КСУ Тимчасовому комітету можуть бути надані розширені повноваження.
            \item Тимчасове управління триває до моменту обрання та початку роботи нового Голови СРФ/СРІ та формування ним основного складу СРФ/СРІ, але не може перевищувати 3 (трьох) місяців. Якщо протягом цього терміну вибори не проведено, питання про подальші дії виноситься на розгляд КСУ.
            \item Тимчасовий комітет є підзвітним СР КАІ та КСУ.
        \end{enumerate} 
\section*{Розділ VII. Взаємодія з іншими органами}
\addcontentsline{toc}{section}{Розділ VII. Взаємодія з іншими органами}

\subsection*{7.1. Взаємодія з Конференцією студентів Університету (КСУ)}
\addcontentsline{toc}{subsection}{7.1. Взаємодія з Конференцією студентів Університету (КСУ)}
    7.1.1. СР КАІ є підзвітною та підконтрольною КСУ в межах, визначених Положенням про ОСС та цим Положенням.

    7.1.2. СР КАІ щорічно подає детальний письмовий звіт про свою діяльність на розгляд КСУ. Звіт має містити інформацію про виконання плану роботи, реалізовані проєкти, фінансову діяльність (за наявності фінансування) та інші ключові аспекти роботи.

    7.1.3. КСУ має право вимагати від СР КАІ надання позачергових звітів або інформації з окремих питань її діяльності. Такі запити КСУ є обов'язковими до виконання СР КАІ у встановлені КСУ терміни.

\subsection*{7.2. Взаємодія зі Студентською уповноваженою делегацією (СУД)}
\addcontentsline{toc}{subsection}{7.2. Взаємодія зі Студентською уповноваженою делегацією (СУД)}
    7.2.1. СР КАІ взаємодіє з СУД з метою забезпечення дотримання нормативних актів ОСС та захисту прав студентів.

    7.2.2. СР КАІ зобов'язана надавати СУД на її запит будь-яку інформацію та документацію (протоколи, рішення, звіти, внутрішні документи тощо), необхідну для виконання СУД своїх повноважень, у терміни, встановлені СУД відповідно до Положення про СУД.

    7.2.3. Запити та рішення СУД, що потребують колегіального розгляду СР КАІ, виносяться на розгляд найближчого засідання СР КАІ. Голова СР КАІ забезпечує підготовку відповіді або організацію виконання рішення СУД у термін, встановлений СУД.

    7.2.4. СР КАІ бере до розгляду та виконання рекомендації та рішення СУД, прийняті в межах її компетенції.

\subsection*{7.3. Взаємодія з Центральною виборчою комісією студентів (ЦВКс)}
\addcontentsline{toc}{subsection}{7.3. Взаємодія з Центральною виборчою комісією студентів (ЦВКс)}
    7.3.1. СР КАІ сприяє ЦВКс в організації та проведенні виборів до органів студентського самоврядування Університету.

    7.3.2. Сприяння полягає у:

        \begin{enumerate}[label=\alph*)]
            \item Наданні ЦВКс необхідної інформації для організації виборчого процесу (наприклад, контактні дані СРФ/СРІ, інформація про структуру факультетів/інститутів тощо).
            \item Сприянні в інформаційному супроводі виборчої кампанії серед студентів, поширенні оголошень та матеріалів ЦВКс через інформаційні ресурси СР КАІ та СРФ/СРІ.
            \item Наданні (за можливості та за попереднім погодженням) приміщень або технічних ресурсів, що знаходяться у розпорядженні СР КАІ, для потреб ЦВКс під час виборчого процесу.
        \end{enumerate}

    7.3.3. СР КАІ не втручається у здійснення ЦВКс її виключних повноважень щодо організації та проведення виборів.

\subsection*{7.4. Взаємодія зі Студентською радою студмістечка (СР СМ)}
\addcontentsline{toc}{subsection}{7.4. Взаємодія зі Студентською радою студмістечка (СР СМ)}
    7.4.1. СР КАІ та СР СМ взаємодіють на засадах партнерства та координації діяльності з метою представлення інтересів студентів, що мешкають у гуртожитках, та забезпечення належних умов проживання, навчання та дозвілля у студмістечку.

    7.4.2. Питання, що стосуються виключно внутрішнього розпорядку та умов проживання в гуртожитках, належать до компетенції СР СМ, з урахуванням необхідності погодження з СР КАІ у випадках, передбачених п. 7.2.5 Положення про ОСС.

    7.4.3. Питання, що стосуються розвитку інфраструктури, безпеки, благоустрою території студмістечка або інші питання, що впливають на всіх студентів Університету, вирішуються СР КАІ за обов'язковим погодженням (консультаціями) з СР СМ.

    7.4.4. У разі виникнення спорів щодо розмежування компетенції або неможливості досягнення згоди між СР КАІ та СР СМ, питання виноситься на розгляд КСУ.

\subsection*{7.5. Взаємодія з адміністрацією Університету}
\addcontentsline{toc}{subsection}{7.5. Взаємодія з адміністрацією Університету}
    7.5.1. Взаємодія СР КАІ з адміністрацією Університету здійснюється відповідно до принципів та процедур, визначених Розділом VII Положення про ОСС, Статутом Університету та цим Положенням.

    7.5.2. Для забезпечення ефективного розгляду та підготовки проєктів рішень щодо погодження дій Адміністрації з питань, зазначених у п. 7.2.5 Положення про ОСС та інших нормативних актах, в структурі СР КАІ створюється постійно діюча \textbf{Комісія з питань взаємодії з адміністрацією та захисту прав студентів} (назва може бути уточнена окремим рішенням СР КАІ).

    7.5.3. Порядок формування, склад та регламент роботи Комісії затверджуються СР КАІ. Комісія здійснює попередній розгляд відповідних подань адміністрації, готує висновки та проєкти рішень для їх затвердження на засіданні СР КАІ.

    7.5.4. Рішення про погодження дій Адміністрації приймається виключно колегіально на засіданні СР КАІ.

\subsection*{7.6. Взаємодія зі студентськими організаціями (СО)}
\addcontentsline{toc}{subsection}{7.6. Взаємодія зі студентськими організаціями (СО)}
    7.6.1. СР КАІ визнає та підтримує діяльність студентських організацій (СО), зареєстрованих в Університеті у встановленому порядку, спрямовану на розвиток студентства та реалізацію їхніх ініціатив.

    7.6.2. З метою систематизації співпраці СР КАІ може укладати Меморандуми або Угоди про співпрацю з СО, в яких визначаються основні напрями взаємодії, спільні цілі, порядок координації дій та обміну інформацією.

    7.6.3. СР КАІ проводить регулярні (як правило, не рідше одного разу на семестр) координаційні зустрічі з керівниками (представниками) визнаних СО для обговорення планів діяльності, узгодження спільних заходів та вирішення поточних питань.

    7.6.4. СР КАІ може надавати інформаційну, організаційну та іншу підтримку проєктам СО відповідно до процедур, визначених цим Положенням та внутрішніми документами СР КАІ.

\subsection*{7.7. Взаємодія з іншими органами та організаціями}
\addcontentsline{toc}{subsection}{7.7. Взаємодія з іншими органами та організаціями}
    7.7.1. СР КАІ розвиває співпрацю з органами студентського самоврядування інших закладів вищої освіти України та світу з метою обміну досвідом, реалізації спільних проєктів та представлення інтересів студентства Університету на міжвишівському та міжнародному рівнях.

    7.7.2. СР КАІ може взаємодіяти з державними органами, органами місцевого самоврядування, громадськими об'єднаннями, молодіжними та іншими організаціями з питань, що стосуються студентського життя, молодіжної політики та захисту прав студентів. 
\section*{Розділ VIII. Фінансування та майно}
\addcontentsline{toc}{section}{Розділ VIII. Фінансування та майно}

\subsection*{8.1. Джерела фінансування}
\addcontentsline{toc}{subsection}{8.1. Джерела фінансування}
    8.1.1. Діяльність СР КАІ фінансується за рахунок коштів, передбачених у єдиному кошторисі (бюджеті) ОСС, що формується з джерел, детально визначених у Розділі V Положення про ОСС.

    8.1.2. СР КАІ має право залучати благодійну допомогу, грантові та спонсорські кошти відповідно до процедур, визначених Положенням про ОСС та законодавством України.

\subsection*{8.2. Кошторис СР КАІ та його виконання}
\addcontentsline{toc}{subsection}{8.2. Кошторис СР КАІ та його виконання}
    8.2.1. Кошторис СР КАІ є складовою частиною єдиного кошторису (бюджету) ОСС, що затверджується Конференцією студентів Університету (КСУ).

    8.2.2. Фінансовий комітет СР КАІ координує збір пропозицій та розробляє проєкт кошторису СР КАІ, включаючи пропозиції щодо фінансування діяльності СРФ/СРІ, для подальшого включення до єдиного кошторису ОСС, що подається на затвердження КСУ.

    8.2.3. Використання коштів СР КАІ здійснюється суворо відповідно до затвердженого КСУ єдиного кошторису ОСС та цільового призначення.

    8.2.4. Фінансовий комітет СР КАІ здійснює загальний моніторинг виконання кошторису СР КАІ. Погодження Фінансовим комітетом потребують значні або нетипові витрати, критерії яких визначаються окремим рішенням СР КАІ. Поточні операційні витрати в межах затверджених статей кошторису погоджуються Головою СР КАІ або відповідальним заступником Голови.

    8.2.5. Усі фінансові операції здійснюються через бухгалтерію Університету відповідно до встановлених процедур.

\subsection*{8.3. Фінансування діяльності СРФ/СРІ}
\addcontentsline{toc}{subsection}{8.3. Фінансування діяльності СРФ/СРІ}
    8.3.1. Фінансування діяльності СРФ/СРІ здійснюється в межах коштів, передбачених для них у єдиному кошторисі ОСС, затвердженому КСУ.

    8.3.2. СРФ/СРІ самостійно ініціюють та здійснюють витрати в межах затверджених для них бюджетних ліній відповідно до процедур фінансової діяльності, встановлених в Університеті.

    8.3.3. Фінансовий комітет СР КАІ здійснює моніторинг відповідності витрат СРФ/СРІ затвердженим бюджетним лініям та цільовому призначенню коштів, але не здійснює попереднього погодження кожної окремої транзакції в межах затверджених ліній.

\subsection*{8.4. Майно СР КАІ}
\addcontentsline{toc}{subsection}{8.4. Майно СР КАІ}
    8.4.1. СР КАІ користується приміщеннями, обладнанням та іншим майном, наданим Адміністрацією Університету на безоплатній основі для забезпечення діяльності відповідно до Положення про ОСС.

    8.4.2. СР КАІ може мати власне майно, придбане за кошти ОСС або отримане як благодійна допомога, дарунок.

    8.4.3. Матеріальну відповідальність за облік, збереження та цільове використання майна, що знаходиться у користуванні СР КАІ, несе Секретар СР КАІ (або інша посадова особа, визначена рішенням СР КАІ відповідальною за адміністративно-господарські питання).

    8.4.4. Використання майна СР КАІ здійснюється виключно для виконання статутних завдань студентського самоврядування.

\subsection*{8.5. Фінансова звітність}
\addcontentsline{toc}{subsection}{8.5. Фінансова звітність}
    8.5.1. СР КАІ забезпечує прозорість використання фінансових ресурсів.

    8.5.2. Фінансовий комітет СР КАІ готує та подає на затвердження СР КАІ щорічний звіт про виконання кошторису СР КАІ. Цей звіт включається до загального звіту СР КАІ перед КСУ.

    8.5.3. Звіти про використання коштів ОСС підлягають оприлюдненню у порядку, визначеному Положенням про ОСС та рішеннями КСУ.

    8.5.4. Контроль за цільовим використанням коштів СР КАІ та СРФ/СРІ здійснює Студентська уповноважена делегація (СУД) відповідно до її повноважень. 
\section*{Розділ IX. Прикінцеві положення}
\addcontentsline{toc}{section}{Розділ IX. Прикінцеві положення}

\subsection*{9.1. Затвердження та набуття чинності Положенням}
\addcontentsline{toc}{subsection}{9.1. Затвердження та набуття чинності Положенням}
    9.1.1. Це Положення затверджується Конференцією студентів Університету (КСУ).

    9.1.2. Це Положення набуває чинності з дня його офіційного оприлюднення на інформаційних ресурсах ОСС Університету після затвердження КСУ.

\subsection*{9.2. Порядок внесення змін та доповнень}
\addcontentsline{toc}{subsection}{9.2. Порядок внесення змін та доповнень}
    9.2.1. Зміни та доповнення до цього Положення вносяться Конференцією студентів Університету (КСУ).

    9.2.2. Право ініціювати внесення змін та доповнень до цього Положення мають Конференція студентів Університету (КСУ), Студентська рада Київського авіаційного інституту (СР КАІ), Студентська рада студмістечка (СР СМ) та Студентська уповноважена делегація (СУД).

    9.2.3. Проєкт змін та доповнень до цього Положення підлягає попередньому оприлюдненню та обговоренню серед студентів у порядку, визначеному Регламентом КСУ або СР КАІ.

    9.2.4. Зміни та доповнення до цього Положення затверджуються простою більшістю голосів від загального складу делегатів КСУ.

\subsection*{9.3. Реорганізація та припинення діяльності СР КАІ}
\addcontentsline{toc}{subsection}{9.3. Реорганізація та припинення діяльності СР КАІ}
    9.3.1. Рішення про реорганізацію (злиття, приєднання, поділ, перетворення) або ліквідацію СР КАІ як органу студентського самоврядування приймається Конференцією студентів Університету (КСУ).

    9.3.2. СР КАІ може ініціювати перед КСУ питання про власну реорганізацію або ліквідацію, надавши відповідне обґрунтування.

    9.3.3. Порядок проведення реорганізації або ліквідації СР КАІ визначається рішенням КСУ.

\subsection*{9.4. Правонаступництво}
\addcontentsline{toc}{subsection}{9.4. Правонаступництво}
    9.4.1. У разі реорганізації СР КАІ її права та обов'язки переходять до новоствореного органу (органів) відповідно до рішення КСУ.

    9.4.2. У разі ліквідації СР КАІ питання правонаступництва щодо її майна, коштів, документів та невиконаних зобов'язань вирішується окремим рішенням КСУ при прийнятті рішення про ліквідацію. КСУ визначає орган (або створює тимчасовий орган), який стане правонаступником СР КАІ.

\subsection*{9.5. Зберігання документації}
\addcontentsline{toc}{subsection}{9.5. Зберігання документації}
    9.5.1. Усі офіційні документи СР КАІ та СРФ/СРІ (протоколи засідань, рішення, розпорядження Голови, звіти, кошториси, листування тощо) створюються та зберігаються в електронному форматі.

    9.5.2. Офіційні електронні документи СР КАІ та СРФ/СРІ підписуються кваліфікованим електронним підписом (КЕП) уповноважених осіб (відповідно Голови, Секретаря, керівників комітетів/відділів – відповідно до їх компетенції).

    9.5.3. СР КАІ та СРФ/СРІ використовують хмарні сховища (офіційні облікові записи Університету або інші визначені відповідними радами ресурси) для централізованого зберігання своїх електронних документів.

    9.5.4. Відповідальність за організацію системи електронного документообігу, структуру папок, надання та контроль прав доступу, забезпечення збереження та резервного копіювання даних у хмарному сховищі покладається:

        \begin{itemize}
            \item Для документів СР КАІ – на Секретаря СР КАІ;
            \item Для документів СРФ/СРІ – на Секретаря відповідної СРФ/СРІ.
        \end{itemize}
    9.5.5. Електронні документи СР КАІ та СРФ/СРІ зберігаються у відповідних сховищах щонайменше протягом 5 (п'яти) років після завершення каденції відповідного складу ради. Після закінчення цього терміну рішення про подальше зберігання (для документів, що мають історичну або практичну цінність) чи видалення документів приймається поточним складом відповідної ради (СР КАІ або СРФ/СРІ) за погодженням з СУД. 


\end{document} 