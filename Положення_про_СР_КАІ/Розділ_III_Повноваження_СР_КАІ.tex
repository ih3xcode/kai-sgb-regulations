\section*{Розділ III. Повноваження СР КАІ}
\addcontentsline{toc}{section}{Розділ III. Повноваження СР КАІ}

\subsection*{3.1. Представницькі повноваження}
\addcontentsline{toc}{subsection}{3.1. Представницькі повноваження}
    3.1.1. СР КАІ представляє та захищає права та законні інтереси студентів Університету перед адміністрацією Університету та її структурними підрозділами.

    3.1.2. СР КАІ представляє студентську спільноту у Вченій раді Університету та інших органах управління Університету відповідно до квот та порядку, визначених Статутом Університету та Положенням про ОСС.

    3.1.3. СР КАІ виступає від імені студентської спільноти Університету у взаємовідносинах з органами студентського самоврядування інших закладів вищої освіти, державними органами, громадськими та іншими організаціями з питань студентського життя.

\subsection*{3.2. Виконавчі повноваження}
\addcontentsline{toc}{subsection}{3.2. Виконавчі повноваження}
    3.2.1. СР КАІ організовує виконання рішень Конференції студентів Університету (КСУ), забезпечуючи розробку необхідних заходів, координацію виконавців та контроль за реалізацією.

    3.2.2. СР КАІ реалізує основні завдання студентського самоврядування, визначені Положенням про ОСС та цим Положенням, у межах своєї компетенції.

    3.2.3. СР КАІ звітує перед КСУ про виконання покладених на неї завдань та рішень КСУ.

\subsection*{3.3. Організаційні повноваження}
\addcontentsline{toc}{subsection}{3.3. Організаційні повноваження}
    3.3.1. СР КАІ організовує та проводить загальноуніверситетські заходи навчального, наукового, культурного, спортивного, соціального та іншого характеру, спрямовані на розвиток студентів та збагачення студентського життя.

    3.3.2. СР КАІ координує роботу своїх структурних підрозділів (комітетів, департаментів тощо).

    3.3.3. СР КАІ сприяє діяльності студентських організацій, гуртків, товариств, клубів за інтересами.

    3.3.4. СР КАІ забезпечує інформаційну підтримку студентів щодо діяльності ОСС та важливих подій в Університеті.

\subsection*{3.4. Повноваження щодо погодження рішень}
\addcontentsline{toc}{subsection}{3.4. Повноваження щодо погодження рішень}
    3.4.1. СР КАІ реалізує право органів студентського самоврядування на участь в управлінні Університетом шляхом погодження рішень адміністрації Університету, що стосуються прав та інтересів студентів, у випадках та порядку, передбачених Законом України "Про вищу освіту", Статутом Університету та Положенням про ОСС.

\subsection*{3.5. Повноваження щодо СРФ/СРІ}
\addcontentsline{toc}{subsection}{3.5. Повноваження щодо СРФ/СРІ}
    3.5.1. СР КАІ здійснює координацію діяльності Студентських рад факультетів/інститутів (СРФ/СРІ), забезпечуючи єдність підходів до реалізації завдань студентського самоврядування.

    3.5.2. СР КАІ надає методичну та організаційну допомогу СРФ/СРІ.

    3.5.3. СР КАІ може розробляти та надавати СРФ/СРІ рекомендації щодо організації їхньої роботи та проведення заходів.

    3.5.4. СР КАІ здійснює контроль за діяльністю СРФ/СРІ в межах повноважень, визначених Положенням про ОСС та/або делегованих КСУ.

    3.5.5. СР КАІ реалізує повноваження щодо тимчасового управління справами студентського самоврядування факультету/інституту у випадках та порядку, визначених Положенням про ОСС.

\subsection*{3.6. Контрольні повноваження}
\addcontentsline{toc}{subsection}{3.6. Контрольні повноваження}
    3.6.1. СР КАІ здійснює внутрішній контроль за діяльністю своїх структурних підрозділів та посадових осіб.

    3.6.2. СР КАІ взаємодіє зі Студентською уповноваженою делегацією (СУД) з питань контролю за дотриманням нормативних документів ОСС та цільовим використанням ресурсів у межах своєї компетенції.

\subsection*{3.7. Обов'язковість рішень СР КАІ}
\addcontentsline{toc}{subsection}{3.7. Обов'язковість рішень СР КАІ}
    3.7.1. Рішення СР КАІ, прийняті колегіально в межах її повноважень відповідно до цього Положення та Положення про ОСС, є обов'язковими для виконання всіма членами СР КАІ, її структурними підрозділами (комітетами, департаментами тощо) та посадовими особами СР КАІ.

    3.7.2. Невиконання або неналежне виконання рішень СР КАІ є підставою для відповідальності відповідно до цього Положення та інших внутрішніх документів ОСС. 