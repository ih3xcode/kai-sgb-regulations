\section*{Розділ II. Структура та склад СР КАІ}
\addcontentsline{toc}{section}{Розділ II. Структура та склад СР КАІ}

\subsection*{2.1. Структура СР КАІ}
\addcontentsline{toc}{subsection}{2.1. Структура СР КАІ}
    2.1.1. СР КАІ є єдиним колегіальним органом, що здійснює свою діяльність через засідання її членів (далі – засідання СР КАІ).

    2.1.2. Для забезпечення ефективного виконання своїх завдань та повноважень за окремими напрямами діяльності СР КАІ формує зі свого складу або із залученням інших студентів постійні або тимчасові робочі органи (комітети, комісії, департаменти, робочі групи тощо). Порядок їх формування та діяльності визначається цим Положенням та/або внутрішніми документами СР КАІ.

\subsection*{2.2. Склад СР КАІ}
\addcontentsline{toc}{subsection}{2.2. Склад СР КАІ}
    2.2.1. Членами СР КАІ з правом голосу є:

        \begin{enumerate}[label=\alph*)]
            \item Голова СР КАІ, обраний на прямих загальноуніверситетських виборах;
            \item Голови Студентських рад факультетів/інститутів Університету за посадою (ex officio) на час виконання ними повноважень голів СРФ/СРІ.
        \end{enumerate}

    2.2.2. СР КАІ має право залучати до своєї роботи інших студентів Університету денної форми навчання для виконання певних функцій (наприклад, як керівників або членів комітетів/департаментів, секретарів тощо). Такі залучені студенти беруть участь у роботі СР КАІ з правом дорадчого голосу та не вважаються членами СР КАІ у контексті прийняття рішень, що потребують голосування.

    2.2.3. Кількісний склад СР КАІ з правом голосу дорівнює кількості факультетів/інститутів в Університеті плюс одна особа (Голова СР КАІ).

\subsection*{2.3. Вимоги до членів та кандидатів}
\addcontentsline{toc}{subsection}{2.3. Вимоги до членів та кандидатів}
    2.3.1. Кандидатом на посаду Голови СР КАІ може бути будь-який студент Університету денної форми навчання, який відповідає вимогам, встановленим Положенням про Центральну виборчу комісію студентів (ЦВКс) (як правило, відсутність академічної заборгованості та діючих дисциплінарних стягнень).

    2.3.2. Голова СР КАІ не може одночасно обіймати посаду Голови Студентської ради факультету/інституту.

    2.3.3. Членом СР КАІ за посадою може бути виключно Голова СРФ/СРІ, обраний відповідно до встановленого порядку.

    2.3.4. Студенти, які залучаються до роботи СР КАІ з правом дорадчого голосу, повинні бути студентами Університету денної форми навчання.

\subsection*{2.4. Строк та порядок набуття і припинення повноважень}
\addcontentsline{toc}{subsection}{2.4. Строк та порядок набуття і припинення повноважень}
    2.4.1. Голова СР КАІ обирається студентами Університету шляхом прямого таємного голосування строком на 1 (один) рік відповідно до процедури, визначеної Положенням про ЦВКс.

    2.4.2. Повноваження Голови СР КАІ можуть бути достроково припинені за рішенням Конференції студентів Університету (КСУ) у порядку, визначеному Положенням про ОСС та/або Регламентом КСУ, або за рішенням Студентської уповноваженої делегації (СУД) у випадках та порядку, передбачених Положенням про СУД.

    2.4.3. Строк повноважень членів СР КАІ – Голів СРФ/СРІ – становить 1 (один) рік та збігається зі строком їхніх повноважень на посаді Голови СРФ/СРІ.

    2.4.4. Член СР КАІ – Голова СРФ/СРІ – набуває повноважень з моменту його обрання Головою СРФ/СРІ та припиняє їх у разі припинення повноважень Голови СРФ/СРІ.

    2.4.5. Повноваження студентів, залучених до роботи з правом дорадчого голосу, визначаються рішенням СР КАІ та можуть бути припинені достроково за рішенням СР КАІ, або у зв'язку із завершенням роботи відповідного робочого органу чи закінченням строку повноважень складу СР КАІ, що їх залучив. 