\section*{Розділ V. Порядок роботи СР КАІ}
\addcontentsline{toc}{section}{Розділ V. Порядок роботи СР КАІ}

\subsection*{5.1. Засідання СР КАІ}
\addcontentsline{toc}{subsection}{5.1. Засідання СР КАІ}
    5.1.1. Основною формою роботи СР КАІ є засідання.

    5.1.2. Засідання СР КАІ скликаються Головою СР КАІ за потребою. Засідання може бути скликане також на вимогу не менше третини від загального числа членів СР КАІ з правом голосу.

    5.1.3. Порядок денний засідання формується Головою СР КАІ з урахуванням пропозицій членів СР КАІ та доводиться до відома членів та запрошених осіб, як правило, не пізніше ніж за 2 (два) робочі дні до засідання, крім випадків невідкладного розгляду питань.

    5.1.4. Засідання СР КАІ можуть проводитися в очному, дистанційному (з використанням офіційно визначених Університетом або СР КАІ засобів відеоконференцзв'язку) або змішаному форматі. Рішення про формат засідання приймає Голова СР КАІ або ініціатори скликання засідання.

    5.1.5. При проведенні засідання у дистанційному або змішаному форматі Секретар СР КАІ (або інша відповідальна особа) забезпечує технічну можливість для ідентифікації учасників, їхньої участі в обговоренні та голосуванні в режимі реального часу.

\subsection*{5.2. Правомочність засідання (Кворум)}
\addcontentsline{toc}{subsection}{5.2. Правомочність засідання (Кворум)}
    5.2.1. Засідання СР КАІ є правомочним, якщо на ньому присутні (особисто або дистанційно) більше половини від загального числа членів СР КАІ з правом голосу.

\subsection*{5.3. Прийняття рішень}
\addcontentsline{toc}{subsection}{5.3. Прийняття рішень}
    5.3.1. Рішення СР КАІ приймаються на її правомочних засіданнях шляхом голосування членів з правом голосу.

    5.3.2. Рішення вважається прийнятим, якщо за нього проголосувало більше половини від \textbf{загального} числа членів СР КАІ з правом голосу.

    5.3.3. Основною формою голосування на засіданнях СР КАІ є відкрите голосування. Таємне голосування проводиться у випадках, прямо передбачених цим Положенням, або за рішенням більшості від присутніх членів СР КАІ з правом голосу.

    5.3.4. Голосування з кадрових питань, що належать до компетенції СР КАІ (погодження кандидатур Заступників Голови, Секретаря, керівників комітетів, відкликання цих посадових осіб), проводиться таємно, якщо цього вимагає хоча б один член СР КАІ з правом голосу.

    5.3.5. На засіданні головує Голова СР КАІ, а за його відсутності – один із Заступників Голови СР КАІ за його дорученням або за рішенням СР КАІ. Головуючий на засіданні бере участь у голосуванні на рівних підставах з іншими членами СР КАІ.

\subsection*{5.4. Ведення протоколу}
\addcontentsline{toc}{subsection}{5.4. Ведення протоколу}
    5.4.1. Хід кожного засідання СР КАІ фіксується у протоколі, який веде Секретар СР КАІ або інша особа за дорученням головуючого.

    5.4.2. Протокол засідання повинен містити: дату, місце (або формат проведення), час початку та закінчення засідання; список присутніх членів СР КАІ (із зазначенням тих, хто має право голосу) та запрошених осіб; порядок денний; стислий виклад обговорення питань порядку денного; результати голосування по кожному питанню (``за'', ``проти'', ``утримались'') та повний текст прийнятих рішень.

    5.4.3. Протокол підписується головуючим на засіданні та секретарем засідання не пізніше 3 (трьох) робочих днів після проведення засідання.

    5.4.4. Протоколи засідань СР КАІ є офіційними документами, зберігаються в СР КАІ протягом строку повноважень відповідного складу та передаються наступному складу або до архіву ОСС в установленому порядку. Копії протоколів або витяги з них надаються за запитами КСУ, СУД, адміністрації Університету.

\subsection*{5.5. Планування та звітність}
\addcontentsline{toc}{subsection}{5.5. Планування та звітність}
    5.5.1. СР КАІ здійснює свою діяльність відповідно до річного плану роботи, затвердженого на одному з перших засідань нового складу.

    5.5.2. СР КАІ регулярно звітує про свою діяльність та виконання плану роботи перед Конференцією студентів Університету у порядку та строки, визначені Положенням про ОСС та Регламентом КСУ. 