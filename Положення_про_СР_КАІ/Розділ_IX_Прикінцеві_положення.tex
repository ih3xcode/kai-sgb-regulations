\section*{Розділ IX. Прикінцеві положення}
\addcontentsline{toc}{section}{Розділ IX. Прикінцеві положення}

\subsection*{9.1. Затвердження та набуття чинності Положенням}
\addcontentsline{toc}{subsection}{9.1. Затвердження та набуття чинності Положенням}
    9.1.1. Це Положення затверджується Конференцією студентів Університету (КСУ).

    9.1.2. Це Положення набуває чинності з дня його офіційного оприлюднення на інформаційних ресурсах ОСС Університету після затвердження КСУ.

\subsection*{9.2. Порядок внесення змін та доповнень}
\addcontentsline{toc}{subsection}{9.2. Порядок внесення змін та доповнень}
    9.2.1. Зміни та доповнення до цього Положення вносяться Конференцією студентів Університету (КСУ).

    9.2.2. Право ініціювати внесення змін та доповнень до цього Положення мають Конференція студентів Університету (КСУ), Студентська рада Київського авіаційного інституту (СР КАІ), Студентська рада студмістечка (СР СМ) та Студентська уповноважена делегація (СУД).

    9.2.3. Проєкт змін та доповнень до цього Положення підлягає попередньому оприлюдненню та обговоренню серед студентів у порядку, визначеному Регламентом КСУ або СР КАІ.

    9.2.4. Зміни та доповнення до цього Положення затверджуються простою більшістю голосів від загального складу делегатів КСУ.

\subsection*{9.3. Реорганізація та припинення діяльності СР КАІ}
\addcontentsline{toc}{subsection}{9.3. Реорганізація та припинення діяльності СР КАІ}
    9.3.1. Рішення про реорганізацію (злиття, приєднання, поділ, перетворення) або ліквідацію СР КАІ як органу студентського самоврядування приймається Конференцією студентів Університету (КСУ).

    9.3.2. СР КАІ може ініціювати перед КСУ питання про власну реорганізацію або ліквідацію, надавши відповідне обґрунтування.

    9.3.3. Порядок проведення реорганізації або ліквідації СР КАІ визначається рішенням КСУ.

\subsection*{9.4. Правонаступництво}
\addcontentsline{toc}{subsection}{9.4. Правонаступництво}
    9.4.1. У разі реорганізації СР КАІ її права та обов'язки переходять до новоствореного органу (органів) відповідно до рішення КСУ.

    9.4.2. У разі ліквідації СР КАІ питання правонаступництва щодо її майна, коштів, документів та невиконаних зобов'язань вирішується окремим рішенням КСУ при прийнятті рішення про ліквідацію. КСУ визначає орган (або створює тимчасовий орган), який стане правонаступником СР КАІ.

\subsection*{9.5. Зберігання документації}
\addcontentsline{toc}{subsection}{9.5. Зберігання документації}
    9.5.1. Усі офіційні документи СР КАІ та СРФ/СРІ (протоколи засідань, рішення, розпорядження Голови, звіти, кошториси, листування тощо) створюються та зберігаються в електронному форматі.

    9.5.2. Офіційні електронні документи СР КАІ та СРФ/СРІ підписуються кваліфікованим електронним підписом (КЕП) уповноважених осіб (відповідно Голови, Секретаря, керівників комітетів/відділів – відповідно до їх компетенції).

    9.5.3. СР КАІ та СРФ/СРІ використовують хмарні сховища (офіційні облікові записи Університету або інші визначені відповідними радами ресурси) для централізованого зберігання своїх електронних документів.

    9.5.4. Відповідальність за організацію системи електронного документообігу, структуру папок, надання та контроль прав доступу, забезпечення збереження та резервного копіювання даних у хмарному сховищі покладається:

        \begin{itemize}
            \item Для документів СР КАІ – на Секретаря СР КАІ;
            \item Для документів СРФ/СРІ – на Секретаря відповідної СРФ/СРІ.
        \end{itemize}
    9.5.5. Електронні документи СР КАІ та СРФ/СРІ зберігаються у відповідних сховищах щонайменше протягом 5 (п'яти) років після завершення каденції відповідного складу ради. Після закінчення цього терміну рішення про подальше зберігання (для документів, що мають історичну або практичну цінність) чи видалення документів приймається поточним складом відповідної ради (СР КАІ або СРФ/СРІ) за погодженням з СУД. 