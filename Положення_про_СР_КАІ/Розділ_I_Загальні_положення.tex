\section*{Розділ I. Загальні положення}
\addcontentsline{toc}{section}{Розділ I. Загальні положення}

\subsection*{1.1. Статус Студентської ради КАІ}
\addcontentsline{toc}{subsection}{1.1. Статус Студентської ради КАІ}
    1.1.1. Студентська рада Київського авіаційного інституту (далі – СР КАІ) є постійно діючим вищим колегіальним виконавчим органом студентського самоврядування Державного університету "Київський авіаційний інститут" (далі – Університет).

    1.1.2. СР КАІ діє відповідно до Конституції України, Закону України "Про вищу освіту", Статуту Університету, Положення про органи студентського самоврядування Університету (далі – Положення про ОСС), цього Положення та інших нормативно-правових актів України.

    1.1.3. Це Положення визначає повноваження, порядок формування, структуру, порядок роботи та інші аспекти діяльності СР КАІ і діє відповідно до та на виконання Положення про ОСС.

\subsection*{1.2. Мета та основні завдання СР КАІ}
\addcontentsline{toc}{subsection}{1.2. Мета та основні завдання СР КАІ}
    1.2.1. Метою діяльності СР КАІ є ефективне представництво та захист прав та законних інтересів студентів Університету на загальноуніверситетському рівні, сприяння їхньому гармонійному розвитку та активній участі в житті Університету.

    1.2.2. Основними завданнями СР КАІ є:

        \begin{enumerate}[label=\alph*)]
            \item Представлення інтересів студентів перед адміністрацією Університету та її структурними підрозділами, у Вченій раді Університету та інших органах управління Університету.
            \item Забезпечення виконання рішень Конференції студентів Університету (КСУ) та реалізація завдань студентського самоврядування, визначених Положенням про ОСС.
            \item Координація діяльності органів студентського самоврядування факультетів/інститутів (СРФ/СРІ).
            \item Сприяння навчальній, науковій, культурно-просвітницькій, спортивній та соціальній діяльності студентів.
            \item Участь у вирішенні питань щодо поліпшення умов навчання, побуту та дозвілля студентів.
            \item Налагодження співпраці зі студентськими організаціями Університету, органами студентського самоврядування інших закладів вищої освіти та молодіжними організаціями.
            \item Інформування студентів про діяльність ОСС та важливі аспекти університетського життя.
        \end{enumerate}

\subsection*{1.3. Принципи діяльності СР КАІ}
\addcontentsline{toc}{subsection}{1.3. Принципи діяльності СР КАІ}
    1.3.1. СР КАІ здійснює свою діяльність на принципах, визначених Положенням про ОСС, зокрема: законності, добровільності, колегіальності, виборності, рівності прав студентів на участь у самоврядуванні, прозорості, організаційної самостійності.

    1.3.2. Ключовими принципами, що визначають виконавчий характер діяльності СР КАІ, є підзвітність перед КСУ та ефективність у реалізації покладених завдань та повноважень.

\subsection*{1.4. Взаємозв'язок з іншими суб'єктами}
\addcontentsline{toc}{subsection}{1.4. Взаємозв'язок з іншими суб'єктами}
    1.4.1. СР КАІ є підзвітною та підконтрольною Конференції студентів Університету.

    1.4.2. СР КАІ взаємодіє з іншими органами студентського самоврядування (Студентською уповноваженою делегацією (СУД), Центральною виборчою комісією студентів (ЦВКс), Студентською радою студмістечка (СР СМ), студентськими радами факультетів/інститутів (СРФ/СРІ)), адміністрацією Університету та студентськими організаціями у порядку, визначеному Положенням про ОСС та цим Положенням. 