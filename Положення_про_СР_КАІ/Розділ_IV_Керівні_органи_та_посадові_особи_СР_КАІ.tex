\section*{Розділ IV. Керівні органи та посадові особи СР КАІ}
\addcontentsline{toc}{section}{Розділ IV. Керівні органи та посадові особи СР КАІ}

\subsection*{4.1. Голова Студентської ради КАІ}
\addcontentsline{toc}{subsection}{4.1. Голова Студентської ради КАІ}
    4.1.1. Студентську раду КАІ очолює Голова СР КАІ, який обирається студентами Університету на прямих виборах відповідно до Розділу II цього Положення та Положення про ЦВКс.

    4.1.2. Голова СР КАІ:

        \begin{enumerate}[label=\alph*)]
            \item Організовує роботу СР КАІ та головує на її засіданнях.
            \item Представляє СР КАІ у відносинах з органами державної влади, місцевого самоврядування, адміністрацією Університету, іншими ОСС, підприємствами, установами, організаціями.
            \item Вносить на розгляд СР КАІ пропозиції щодо кандидатур Заступника(-ів) Голови, Секретаря СР КАІ та керівників постійних комітетів/департаментів.
            \item Координує роботу Заступника(-ів) Голови, Секретаря та керівників комітетів/департаментів СР КАІ.
            \item Підписує рішення, протоколи та інші документи СР КАІ.
            \item Здійснює оперативне управління коштами та майном, що перебувають у віданні СР КАІ, підписує фінансові документи та затверджує витрати в межах кошторису, погодженого СР КАІ, та відповідно до її рішень. Це повноваження не поширюється на кошти та майно СРФ/СРІ;
            \item Забезпечує виконання рішень КСУ та СР КАІ.
            \item Звітує про свою діяльність та діяльність СР КАІ перед КСУ та СР КАІ.
            \item Виконує інші повноваження, передбачені цим Положенням та Положенням про ОСС.
        \end{enumerate}

    4.1.3. Голова СР КАІ є підзвітним Конференції студентів Університету. Повноваження Голови СР КАІ можуть бути достроково припинені за рішенням КСУ або за рішенням Студентської уповноваженої делегації (СУД) у випадках та порядку, передбачених Положенням про СУД.

    \subsubsection*{4.1.4. Призначення виконувача обов'язків Голови СР КАІ}
        4.1.4.1. У разі дострокового припинення повноважень Голови СР КАІ, його тимчасової неможливості виконувати свої обов'язки (через хворобу, відрядження тощо) або інших обставин, що унеможливлюють виконання ним своїх повноважень, призначається виконувач обов'язків (в.о.) Голови СР КАІ.
        
        4.1.4.2. В.о. Голови СР КАІ може бути призначений колегіальним рішенням СР КАІ на термін не більше 1 (одного) місяця. Для призначення в.о. на більший термін необхідне рішення КСУ або СУД.
        
        4.1.4.3. В.о. Голови СР КАІ може бути Голова СРФ/СРІ або інший студент Університету, що відповідає загальним вимогам до членів ОСС. На період виконання обов'язків Голови СР КАІ, Голова СРФ/СРІ тимчасово делегує свої повноваження заступнику в СРФ/СРІ без дострокового припинення своїх повноважень у відповідній СРФ/СРІ.
        
        4.1.4.4. В.о. Голови СР КАІ має повний обсяг повноважень Голови СР КАІ, окрім права голосу в СР КАІ як члена з правом голосу. Якщо рішення про призначення в.о. прийнято КСУ або СУД, у ньому можуть зазначатися додаткові обмеження повноважень.
        
        4.1.4.5. Передача справ в.о. Голови СР КАІ здійснюється відповідно до процедури, визначеної розділом 4.6 Положення про ОСС.

\subsection*{4.2. Заступник(-и) Голови та Секретар СР КАІ}
\addcontentsline{toc}{subsection}{4.2. Заступник(-и) Голови та Секретар СР КАІ}
    4.2.1. Голова СР КАІ може мати одного або кількох заступників та Секретаря СР КАІ.

    4.2.2. Заступник(-и) Голови та Секретар СР КАІ призначаються Головою СР КАІ з числа студентів, що беруть участь у роботі СР КАІ (з правом голосу або дорадчим), за погодженням (затвердженням) СР КАІ.

    4.2.3. Заступник Голови СР КАІ виконує обов'язки Голови СР КАІ за його відсутності або за його дорученням, координує роботу визначених комітетів/департаментів або напрямів діяльності.

    4.2.4. Секретар СР КАІ відповідає за ведення протоколів засідань СР КАІ, організацію документообігу та архіву СР КАІ.

    4.2.5. Повноваження Заступника(-ів) Голови та Секретаря СР КАІ можуть бути достроково припинені за рішенням СР КАІ.

\subsection*{4.3. Комітети/Департаменти СР КАІ}
\addcontentsline{toc}{subsection}{4.3. Комітети/Департаменти СР КАІ}
    4.3.1. Для реалізації основних завдань та напрямів діяльності СР КАІ може створювати постійні або тимчасові комітети/департаменти (наприклад: навчально-науковий, соціально-побутовий, інформаційний, культурно-масовий, спортивний, міжнародного співробітництва, фінансовий тощо).

    4.3.2. Структура, перелік та повноваження комітетів/департаментів затверджуються СР КАІ.

    4.3.3. Керівники комітетів/департаментів призначаються Головою СР КАІ з числа студентів, що беруть участь у роботі СР КАІ, за погодженням (затвердженням) СР КАІ.

    4.3.4. Керівник комітету/департаменту організовує роботу відповідного підрозділу, звітує про його діяльність перед СР КАІ та несе відповідальність за виконання покладених на підрозділ завдань.

    4.3.5. Повноваження керівника комітету/департаменту можуть бути достроково припинені за рішенням СР КАІ.

\subsection*{4.4. Відповідальність посадових осіб}
\addcontentsline{toc}{subsection}{4.4. Відповідальність посадових осіб}
    4.4.1. Заступник(-и) Голови, Секретар СР КАІ та керівники комітетів/департаментів несуть відповідальність перед СР КАІ за належне виконання своїх обов'язків.

    4.4.2. У разі систематичного невиконання обов'язків, порушення цього Положення, Положення про ОСС або вчинення дій, що шкодять репутації ОСС, зазначені посадові особи можуть бути достроково відкликані зі своїх посад за рішенням СР КАІ.
     
    \subsubsection*{4.4.3. Призначення виконувачів обов'язків інших посадових осіб СР КАІ}
        4.4.3.1. У разі дострокового припинення повноважень, тимчасової неможливості виконувати обов'язки або інших обставин, що унеможливлюють виконання повноважень Заступника(-ів) Голови, Секретаря СР КАІ або керівників комітетів/департаментів, призначення виконувача обов'язків здійснюється в тому ж порядку, що й призначення відповідних посадових осіб.
         
        4.4.3.2. Термін повноважень, обсяг повноважень та порядок передачі справ для таких в.о. визначаються рішенням про їх призначення. 