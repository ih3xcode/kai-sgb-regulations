\section*{Розділ VI. Студентські ради факультетів/інститутів}
\addcontentsline{toc}{section}{Розділ VI. Студентські ради факультетів/інститутів}

\subsection*{6.1. Статус та завдання СРФ/СРІ}
\addcontentsline{toc}{subsection}{6.1. Статус та завдання СРФ/СРІ}
    6.1.1. Студентська рада факультету/інституту (далі – СРФ/СРІ) є основним органом студентського самоврядування на рівні факультету/інституту, який діє відповідно до Закону України "Про вищу освіту", Статуту Університету, Положення про ОСС та цього Положення.

    6.1.2. СРФ/СРІ є підзвітною та підконтрольною студентам факультету/інституту (через процедуру виборів та звітування Голови), Конференції студентів Університету (КСУ) та Студентській раді КАІ в межах їхніх повноважень.

    6.1.3. Основним завданням СРФ/СРІ є представництво та захист прав та інтересів студентів свого факультету/інституту, а також реалізація завдань студентського самоврядування, визначених Положенням про ОСС та цим Положенням, на рівні факультету/інституту.

\subsection*{6.2. Формування та склад СРФ/СРІ}
\addcontentsline{toc}{subsection}{6.2. Формування та склад СРФ/СРІ}
    6.2.1. СРФ/СРІ очолює Голова, який обирається студентами факультету/інституту шляхом прямого таємного голосування строком на 1 (один) рік відповідно до процедури, визначеної Положенням про ЦВКс.

    6.2.2. Інші члени СРФ/СРІ залучаються до складу Головою СРФ/СРІ на підставі особистої заяви студента денної форми навчання відповідного факультету/інституту.

    6.2.3. Голова СРФ/СРІ видає розпорядження про включення студента до складу СРФ/СРІ, якщо відсутні підстави для відмови, визначені у п. 6.2.4.

    6.2.4. Рішення про відмову студенту у включенні до складу СРФ/СРІ приймається \textbf{колегіально СРФ/СРІ} більшістю голосів від загального складу виключно з таких підстав:

        \begin{enumerate}[label=\alph*)]
            \item Невідповідність кандидата вимогам (не є студентом денної форми навчання факультету/інституту);
            \item Наявність у кандидата на момент розгляду питання діючого дисциплінарного стягнення;
            \item Попереднє виключення кандидата зі складу ОСС за грубе порушення нормативних документів ОСС або вчинення дій, що завдали значної шкоди репутації ОСС.
        \end{enumerate}

    6.2.5. Рішення СРФ/СРІ про відмову у включенні до складу може бути оскаржене студентом до Студентської уповноваженої делегації (СУД) протягом 7 (семи) календарних днів з моменту його прийняття.

    6.2.6. Загальна кількість членів СРФ/СРІ (включаючи Голову) з правом голосу обмежується \textbf{15 (п'ятнадцятьма) особами}.

    6.2.7. Студенти, залучені до складу СРФ/СРІ понад встановлений ліміт (15 осіб), беруть участь у роботі з правом \textbf{дорадчого голосу} та не можуть претендувати на додаткові бали чи інші заохочення, передбачені для членів ОСС з правом голосу.

    6.2.8. Секретар СРФ/СРІ веде актуальний реєстр членів СРФ/СРІ із зазначенням дати включення до складу та чітким розмежуванням членів з правом голосу та членів з правом дорадчого голосу.

    6.2.9. У виняткових випадках, за наявності обґрунтованої потреби, Конференція студентів Університету (КСУ) за поданням Студентської ради КАІ (СР КАІ) або Студентської уповноваженої делегації (СУД) може прийняти рішення про збільшення максимальної кількості членів відповідної СРФ/СРІ з правом голосу понад ліміт, встановлений п. 6.2.6 цього Положення. Таке рішення має визначати новий ліміт та, за потреби, термін його дії.

\subsection*{6.3. Структура СРФ/СРІ}
\addcontentsline{toc}{subsection}{6.3. Структура СРФ/СРІ}
    6.3.1. Обов'язковими посадами в СРФ/СРІ є:

        \begin{enumerate}[label=\alph*)]
            \item Голова СРФ/СРІ;
            \item Заступник (або заступники) Голови СРФ/СРІ — виконує обов'язки Голови за його відсутності або дорученням, координує роботу визначених напрямів або структурних підрозділів;
            \item Секретар СРФ/СРІ — відповідає за ведення протоколів засідань, діловодство, ведення реєстру членів та архіву СРФ/СРІ.
        \end{enumerate}

    6.3.2. Заступник(-и) Голови та Секретар СРФ/СРІ призначаються Головою СРФ/СРІ з числа членів СРФ/СРІ з правом голосу.

    6.3.3. В СРФ/СРІ обов'язково створюються наступні структурні підрозділи (відділи/комітети):

        \begin{enumerate}[label=\alph*)]
            \item Медіавідділ: відповідає за створення та поширення інформаційних матеріалів про діяльність СРФ/СРІ та ОСС, ведення офіційних сторінок СРФ/СРІ у соціальних мережах, внутрішні та зовнішні комунікації.
            \item Організаційний відділ: відповідає за розробку планів заходів СРФ/СРІ, логістичне забезпечення, організацію та проведення освітніх, культурних, спортивних, соціальних та інших заходів на рівні факультету/інституту.
        \end{enumerate}

    6.3.4. СРФ/СРІ за поданням Голови може створювати інші постійні або тимчасові структурні підрозділи (відділи, комітети, проєктні групи) для виконання специфічних завдань. Рішення про створення або ліквідацію таких підрозділів приймається колегіально СРФ/СРІ.

    6.3.5. Керівники структурних підрозділів призначаються Головою СРФ/СРІ з числа членів СРФ/СРІ (переважно з правом голосу).

\subsection*{6.4. Повноваження Голови СРФ/СРІ}
\addcontentsline{toc}{subsection}{6.4. Повноваження Голови СРФ/СРІ}
    6.4.1. Голова СРФ/СРІ, окрім повноважень, визначених у п. 6.1.2 та інших розділах цього Положення:

        \begin{enumerate}[label=\alph*)]
            \item Здійснює загальне керівництво діяльністю СРФ/СРІ.
            \item Забезпечує формування складу СРФ/СРІ шляхом розгляду заяв студентів та видання розпоряджень про включення до складу відповідно до п. 6.2.2-6.2.3.
            \item Ініціює перед СРФ/СРІ розгляд питання про відмову у включенні до складу відповідно до п. 6.2.4.
            \item Призначає Заступника(-ів) Голови, Секретаря та керівників структурних підрозділів СРФ/СРІ.
            \item Організовує та контролює роботу членів та структурних підрозділів СРФ/СРІ.
            \item Скликає та веде засідання СРФ/СРІ.
            \item Видає розпорядження організаційно-виконавчого характеру в межах своєї компетенції, що не належать до виключної компетенції колегіального органу СРФ/СРІ.
            \item Вносить на розгляд СРФ/СРІ подання про дострокове припинення повноважень (виключення) члена СРФ/СРІ з підстав, визначених цим Положенням.
            \item Представляє СРФ/СРІ у відносинах з деканатом/дирекцією, СР КАІ, іншими органами та організаціями.
            \item Підписує протоколи засідань, рішення та інші офіційні документи СРФ/СРІ.
            \item Звітує про свою діяльність та діяльність СРФ/СРІ перед студентами факультету/інституту та СР КАІ.
        \end{enumerate}

\subsection*{6.5. Припинення повноважень членів СРФ/СРІ}
\addcontentsline{toc}{subsection}{6.5. Припинення повноважень членів СРФ/СРІ}
    6.5.1. Повноваження члена СРФ/СРІ (включаючи Голову) припиняються у зв'язку із закінченням строку, на який його було обрано/призначено (1 рік).

    6.5.2. Повноваження члена СРФ/СРІ (включаючи Голову) припиняються достроково та автоматично у разі:

        \begin{enumerate}[label=\alph*)]
            \item Подання особистої заяви про складання повноважень;
            \item Втрати статусу студента Університету;
            \item Набрання законної сили обвинувальним вироком суду щодо нього;
            \item Смерті або визнання його безвісно відсутнім чи померлим.
        \end{enumerate}

    % --- Дострокове припинення повноважень Голови СРФ/СРІ ---
    6.5.3. Повноваження \textbf{Голови} СРФ/СРІ можуть бути достроково припинені (він може бути відкликаний/усунутий з посади) також за рішенням:

        \begin{enumerate}[label=\alph*)]
            \item Студентів факультету/інституту (відкликання виборцями) у порядку, визначеному Положенням про ЦВКс;
            \item Конференції студентів Університету (КСУ) за висловленням недовіри або з інших підстав, передбачених Положенням про ОСС, у порядку, визначеному Регламентом КСУ та/або Положенням про ОСС;
            \item Студентської уповноваженої делегації (СУД) у випадках та порядку, передбачених Положенням про СУД.
        \end{enumerate}

    % --- Дострокове припинення повноважень інших членів СРФ/СРІ ---
    6.5.4. Повноваження члена СРФ/СРІ (\textbf{окрім Голови}) можуть бути достроково припинені за рішенням СРФ/СРІ (прийнятим колегіально більшістю голосів від загального складу) за поданням Голови СРФ/СРІ у разі:

        \begin{enumerate}[label=\alph*)]
            \item Систематичного невиконання обов'язків члена СРФ/СРІ, зокрема неучасті у засіданнях чи роботі структурного підрозділу;
            \item Грубого порушення нормативних документів ОСС або Статуту Університету;
            \item Вчинення дій, що завдали значної шкоди репутації ОСС Університету.
        \end{enumerate}

    6.5.5. Перед прийняттям рішення відповідно до п. 6.5.4 СРФ/СРІ зобов'язана надати можливість члену, стосовно якого розглядається питання, надати пояснення.

    6.5.6. Рішення СРФ/СРІ про дострокове припинення повноважень члена (окрім Голови) може бути оскаржене до Студентської уповноваженої делегації (СУД) протягом 7 (семи) календарних днів з моменту його прийняття.

    \subsubsection*{6.5.7. Призначення виконувача обов'язків Голови СРФ/СРІ}
        6.5.7.1. У разі дострокового припинення повноважень Голови СРФ/СРІ, його тимчасової неможливості виконувати свої обов'язки (через хворобу, відрядження тощо) або інших обставин, що унеможливлюють виконання ним своїх повноважень, призначається виконувач обов'язків (в.о.) Голови СРФ/СРІ.
        
        6.5.7.2. В.о. Голови СРФ/СРІ може бути призначений колегіальним рішенням СРФ/СРІ на термін не більше 1 (одного) місяця. Для призначення в.о. на більший термін необхідне рішення КСУ або СУД за поданням СР КАІ.
        
        6.5.7.3. В.о. Голови СРФ/СРІ має повний обсяг повноважень Голови СРФ/СРІ, окрім права голосу в СР КАІ як члена з правом голосу та права голосу в СРФ/СРІ. Якщо рішення про призначення в.о. прийнято КСУ або СУД, у ньому можуть зазначатися додаткові обмеження повноважень.
        
        6.5.7.4. Передача справ в.о. Голови СРФ/СРІ здійснюється відповідно до процедури, визначеної розділом 4.6 Положення про ОСС.
        
    \subsubsection*{6.5.8. Призначення виконувачів обов'язків інших посадових осіб СРФ/СРІ}
        6.5.8.1. У разі дострокового припинення повноважень, тимчасової неможливості виконувати обов'язки або інших обставин, що унеможливлюють виконання повноважень Заступника(-ів) Голови, Секретаря СРФ/СРІ або керівників структурних підрозділів, призначення виконувача обов'язків здійснюється в тому ж порядку, що й призначення відповідних посадових осіб.
        
        6.5.8.2. Термін повноважень, обсяг повноважень та порядок передачі справ для таких в.о. визначаються рішенням про їх призначення.

\subsection*{6.6. Основні повноваження та порядок роботи СРФ/СРІ}
\addcontentsline{toc}{subsection}{6.6. Основні повноваження та порядок роботи СРФ/СРІ}
    6.6.1. СРФ/СРІ виконує завдання та повноваження, визначені у п. 6.1.3. Основні напрями діяльності включають, але не обмежуються:

        \begin{enumerate}[label=\alph*)]
            \item Представництво та захист прав студентів факультету/інституту перед адміністрацією факультету/інституту та іншими структурами.
            \item Участь у Вченій раді факультету/інституту.
            \item Сприяння покращенню якості освітнього процесу на факультеті/інституті.
            \item Організація та проведення заходів для студентів факультету/інституту.
            \item Інформаційне забезпечення студентів факультету/інституту.
            \item Взаємодія з СР КАІ та іншими ОСС.
        \end{enumerate}
        
    6.6.2. Основною формою роботи СРФ/СРІ є засідання, які проводяться за потребою, але не рідше одного разу на місяць протягом навчального семестру.

    6.6.3. У засіданні СРФ/СРІ беруть участь з правом голосу члени СРФ/СРІ в межах встановленої квоти (до 15 осіб) та з правом дорадчого голосу - інші залучені члени. Засідання є правомочним, якщо на ньому присутні більше половини від складу членів з правом голосу.

    6.6.4. Рішення СРФ/СРІ приймаються більшістю голосів від загального складу членів з правом голосу, якщо інше не передбачено цим Положенням. Процедурні питання можуть вирішуватися більшістю від присутніх членів з правом голосу.

    6.6.5. \textbf{Виключно колегіально} СРФ/СРІ приймає рішення з таких питань:

        \begin{enumerate}[label=\alph*)]
            \item Відмова у включенні студента до складу СРФ/СРІ (п. 6.2.4);

            \item Дострокове припинення повноважень (виключення) члена СРФ/СРІ (п. 6.5.4);

            \item Створення або ліквідація структурних підрозділів СРФ/СРІ (окрім обов'язкових);
            \item Затвердження плану роботи та звіту про діяльність СРФ/СРІ;
            \item Затвердження позиції СРФ/СРІ щодо погодження рішень адміністрації факультету/інституту з питань, що належать до компетенції ОСС (напр., щодо ОПП, відрахування тощо);
            \item Інші питання, віднесені до виключної компетенції СРФ/СРІ цим Положенням або за рішенням самої СРФ/СРІ.
        \end{enumerate}

    6.6.6. Хід засідання фіксується у протоколі, який веде Секретар СРФ/СРІ.

    6.6.7. СРФ/СРІ здійснює свою діяльність відповідно до плану роботи, затвердженого колегіально на початку навчального року.

    6.6.8. Внутрішні спори та суперечності між членами СРФ/СРІ або між Головою та членами СРФ/СРІ вирішуються шляхом обговорення та прийняття колегіального рішення на засіданні СРФ/СРІ. У разі неможливості дійти згоди, будь-яка зі сторін може звернутися до СР КАІ для надання допомоги у вирішенні спору (медіації).

    6.6.9. Рішення про зміну статусу члена СРФ/СРІ щодо права голосу (надання або позбавлення права голосу) в межах встановленого п. 6.2.6 ліміту приймається колегіально СРФ/СРІ кваліфікованою більшістю голосів - не менше 2/3 (двох третин) від загального складу членів з правом голосу. Таке рішення не може призводити до перевищення ліміту у 15 членів з правом голосу. Секретар вносить відповідні зміни до реєстру членів СРФ/СРІ.

\subsection*{6.7. Взаємодія СРФ/СРІ з СР КАІ}
\addcontentsline{toc}{subsection}{6.7. Взаємодія СРФ/СРІ з СР КАІ}
    6.7.1. СРФ/СРІ зобов'язані брати участь у координаційних заходах, що проводяться СР КАІ.

    6.7.2. СРФ/СРІ надають СР КАІ інформацію та звіти про свою діяльність у порядку та строки, встановлені СР КАІ.

    6.7.3. СРФ/СРІ розглядають рекомендації СР КАІ щодо організації своєї роботи та проведення заходів та інформують СР КАІ про результати розгляду.

    6.7.4. СРФ/СРІ сприяють виконанню рішень СР КАІ, що стосуються загальноуніверситетських питань, на рівні факультету/інституту.

\subsection*{6.8. Порядок діяльності СРФ/СРІ при реорганізації структурного підрозділу}
\addcontentsline{toc}{subsection}{6.8. Порядок діяльності СРФ/СРІ при реорганізації структурного підрозділу}
    6.8.1. Підставою для початку процедур реорганізації СРФ/СРІ є офіційний Наказ Ректора Університету або Рішення Вченої ради Університету про відповідну реорганізацію структурного підрозділу (факультету/інституту).

    6.8.2. \textbf{При перейменуванні факультету/інституту:} Існуюча СРФ/СРІ автоматично змінює свою назву відповідно до нової назви структурного підрозділу та продовжує виконувати свої повноваження до закінчення строку, на який її було обрано.

    6.8.3. \textbf{При злитті/приєднанні факультетів/інститутів:}

        \begin{enumerate}[label=\alph*)]
            \item Студентські ради факультетів/інститутів, що реорганізуються, продовжують діяльність протягом 1 (одного) місяця з дати офіційного об'єднання (перехідний період) з метою завершення поточних справ, передачі документації та активів, після чого їхні повноваження припиняються.
            \item Конференція студентів Університету (КСУ) на найближчому засіданні після офіційного об'єднання приймає рішення про доцільність створення єдиної Студентської ради для новоутвореного факультету/інституту.
            \item У разі прийняття КСУ рішення про створення нової єдиної СРФ/СРІ, Центральна виборча комісія студентів (ЦВКс) організовує та проводить вибори Голови цієї СРФ/СРІ протягом 2 (двох) місяців з дати офіційного об'єднання факультетів/інститутів. Новообраний Голова формує склад СРФ/СРІ відповідно до цього Положення.
            \item Делегати КСУ та представники в інших органах, обрані від студентів факультетів/інститутів, що реорганізуються, продовжують виконувати свої повноваження до моменту обрання нових представників від новоутвореного структурного підрозділу.
        \end{enumerate}

    6.8.4. \textbf{При розформуванні/ліквідації факультету/інституту (без злиття):}

        \begin{enumerate}[label=\alph*)]
            \item Студентська рада факультету/інституту, що ліквідується, продовжує діяльність протягом 2 (двох) тижнів з дати офіційної ліквідації (перехідний період) з метою завершення поточних справ, передачі документації та активів, після чого її повноваження припиняються.
            \item Уся документація, активи та справи розформованої СРФ/СРІ передаються до Студентської ради Київського авіаційного інституту (СР КАІ).
            \item Студенти, переведені на інші факультети/інститути, автоматично підпадають під представництво та юрисдикцію Студентських рад відповідних факультетів/інститутів, куди їх було переведено.
        \end{enumerate}

    6.8.5. \textbf{Тимчасове управління від СР КАІ:}

        \begin{enumerate}[label=\alph*)]
            \item У випадку, якщо проведення виборів Голови новоствореної СРФ/СРІ (відповідно до п. 6.8.3.в) цього Положення) є неможливим у встановлений 2-місячний термін, КСУ за поданням СР КАІ або ЦВКс може прийняти рішення про запровадження тимчасового управління справами студентського самоврядування відповідного факультету/інституту.

            \item Тимчасове управління здійснюється Тимчасовим комітетом, що формується та призначається СР КАІ.
            \item Повноваження Тимчасового комітету за замовчуванням обмежуються підтриманням базової операційної діяльності, забезпеченням мінімально необхідного представництва інтересів студентів факультету/інституту та першочерговою організацією та сприянням проведенню виборів Голови СРФ/СРІ. За окремим рішенням КСУ Тимчасовому комітету можуть бути надані розширені повноваження.
            \item Тимчасове управління триває до моменту обрання та початку роботи нового Голови СРФ/СРІ та формування ним основного складу СРФ/СРІ, але не може перевищувати 3 (трьох) місяців. Якщо протягом цього терміну вибори не проведено, питання про подальші дії виноситься на розгляд КСУ.
            \item Тимчасовий комітет є підзвітним СР КАІ та КСУ.
        \end{enumerate} 