\section*{Розділ VII. Взаємодія з іншими органами}
\addcontentsline{toc}{section}{Розділ VII. Взаємодія з іншими органами}

\subsection*{7.1. Взаємодія з Конференцією студентів Університету (КСУ)}
\addcontentsline{toc}{subsection}{7.1. Взаємодія з Конференцією студентів Університету (КСУ)}
    7.1.1. СР КАІ є підзвітною та підконтрольною КСУ в межах, визначених Положенням про ОСС та цим Положенням.

    7.1.2. СР КАІ щорічно подає детальний письмовий звіт про свою діяльність на розгляд КСУ. Звіт має містити інформацію про виконання плану роботи, реалізовані проєкти, фінансову діяльність (за наявності фінансування) та інші ключові аспекти роботи.

    7.1.3. КСУ має право вимагати від СР КАІ надання позачергових звітів або інформації з окремих питань її діяльності. Такі запити КСУ є обов'язковими до виконання СР КАІ у встановлені КСУ терміни.

\subsection*{7.2. Взаємодія зі Студентською уповноваженою делегацією (СУД)}
\addcontentsline{toc}{subsection}{7.2. Взаємодія зі Студентською уповноваженою делегацією (СУД)}
    7.2.1. СР КАІ взаємодіє з СУД з метою забезпечення дотримання нормативних актів ОСС та захисту прав студентів.

    7.2.2. СР КАІ зобов'язана надавати СУД на її запит будь-яку інформацію та документацію (протоколи, рішення, звіти, внутрішні документи тощо), необхідну для виконання СУД своїх повноважень, у терміни, встановлені СУД відповідно до Положення про СУД.

    7.2.3. Запити та рішення СУД, що потребують колегіального розгляду СР КАІ, виносяться на розгляд найближчого засідання СР КАІ. Голова СР КАІ забезпечує підготовку відповіді або організацію виконання рішення СУД у термін, встановлений СУД.

    7.2.4. СР КАІ бере до розгляду та виконання рекомендації та рішення СУД, прийняті в межах її компетенції.

\subsection*{7.3. Взаємодія з Центральною виборчою комісією студентів (ЦВКс)}
\addcontentsline{toc}{subsection}{7.3. Взаємодія з Центральною виборчою комісією студентів (ЦВКс)}
    7.3.1. СР КАІ сприяє ЦВКс в організації та проведенні виборів до органів студентського самоврядування Університету.

    7.3.2. Сприяння полягає у:

        \begin{enumerate}[label=\alph*)]
            \item Наданні ЦВКс необхідної інформації для організації виборчого процесу (наприклад, контактні дані СРФ/СРІ, інформація про структуру факультетів/інститутів тощо).
            \item Сприянні в інформаційному супроводі виборчої кампанії серед студентів, поширенні оголошень та матеріалів ЦВКс через інформаційні ресурси СР КАІ та СРФ/СРІ.
            \item Наданні (за можливості та за попереднім погодженням) приміщень або технічних ресурсів, що знаходяться у розпорядженні СР КАІ, для потреб ЦВКс під час виборчого процесу.
        \end{enumerate}

    7.3.3. СР КАІ не втручається у здійснення ЦВКс її виключних повноважень щодо організації та проведення виборів.

\subsection*{7.4. Взаємодія зі Студентською радою студмістечка (СР СМ)}
\addcontentsline{toc}{subsection}{7.4. Взаємодія зі Студентською радою студмістечка (СР СМ)}
    7.4.1. СР КАІ та СР СМ взаємодіють на засадах партнерства та координації діяльності з метою представлення інтересів студентів, що мешкають у гуртожитках, та забезпечення належних умов проживання, навчання та дозвілля у студмістечку.

    7.4.2. Питання, що стосуються виключно внутрішнього розпорядку та умов проживання в гуртожитках, належать до компетенції СР СМ, з урахуванням необхідності погодження з СР КАІ у випадках, передбачених п. 7.2.5 Положення про ОСС.

    7.4.3. Питання, що стосуються розвитку інфраструктури, безпеки, благоустрою території студмістечка або інші питання, що впливають на всіх студентів Університету, вирішуються СР КАІ за обов'язковим погодженням (консультаціями) з СР СМ.

    7.4.4. У разі виникнення спорів щодо розмежування компетенції або неможливості досягнення згоди між СР КАІ та СР СМ, питання виноситься на розгляд КСУ.

\subsection*{7.5. Взаємодія з адміністрацією Університету}
\addcontentsline{toc}{subsection}{7.5. Взаємодія з адміністрацією Університету}
    7.5.1. Взаємодія СР КАІ з адміністрацією Університету здійснюється відповідно до принципів та процедур, визначених Розділом VII Положення про ОСС, Статутом Університету та цим Положенням.

    7.5.2. Для забезпечення ефективного розгляду та підготовки проєктів рішень щодо погодження дій Адміністрації з питань, зазначених у п. 7.2.5 Положення про ОСС та інших нормативних актах, в структурі СР КАІ створюється постійно діюча \textbf{Комісія з питань взаємодії з адміністрацією та захисту прав студентів} (назва може бути уточнена окремим рішенням СР КАІ).

    7.5.3. Порядок формування, склад та регламент роботи Комісії затверджуються СР КАІ. Комісія здійснює попередній розгляд відповідних подань адміністрації, готує висновки та проєкти рішень для їх затвердження на засіданні СР КАІ.

    7.5.4. Рішення про погодження дій Адміністрації приймається виключно колегіально на засіданні СР КАІ.

\subsection*{7.6. Взаємодія зі студентськими організаціями (СО)}
\addcontentsline{toc}{subsection}{7.6. Взаємодія зі студентськими організаціями (СО)}
    7.6.1. СР КАІ визнає та підтримує діяльність студентських організацій (СО), зареєстрованих в Університеті у встановленому порядку, спрямовану на розвиток студентства та реалізацію їхніх ініціатив.

    7.6.2. З метою систематизації співпраці СР КАІ може укладати Меморандуми або Угоди про співпрацю з СО, в яких визначаються основні напрями взаємодії, спільні цілі, порядок координації дій та обміну інформацією.

    7.6.3. СР КАІ проводить регулярні (як правило, не рідше одного разу на семестр) координаційні зустрічі з керівниками (представниками) визнаних СО для обговорення планів діяльності, узгодження спільних заходів та вирішення поточних питань.

    7.6.4. СР КАІ може надавати інформаційну, організаційну та іншу підтримку проєктам СО відповідно до процедур, визначених цим Положенням та внутрішніми документами СР КАІ.

\subsection*{7.7. Взаємодія з іншими органами та організаціями}
\addcontentsline{toc}{subsection}{7.7. Взаємодія з іншими органами та організаціями}
    7.7.1. СР КАІ розвиває співпрацю з органами студентського самоврядування інших закладів вищої освіти України та світу з метою обміну досвідом, реалізації спільних проєктів та представлення інтересів студентства Університету на міжвишівському та міжнародному рівнях.

    7.7.2. СР КАІ може взаємодіяти з державними органами, органами місцевого самоврядування, громадськими об'єднаннями, молодіжними та іншими організаціями з питань, що стосуються студентського життя, молодіжної політики та захисту прав студентів. 