\documentclass[12pt, a4paper]{article}
\usepackage[ukrainian]{babel}
\usepackage{fontspec}
\usepackage{amsmath}
\usepackage{amssymb}
\usepackage{graphicx}
\usepackage{indentfirst}
\usepackage{enumitem}
\usepackage{geometry}
\geometry{a4paper, margin=2cm}
\usepackage[nottoc]{tocbibind} % Додаємо для відображення ненумерованих розділів у змісті
\usepackage{hyperref} % Додаємо підтримку гіперпосилань
\hypersetup{
    colorlinks=true,
    linkcolor=black,
    filecolor=blue,      
    urlcolor=blue,
    pdftitle={Положення про органи студентського самоврядування},
    pdfauthor={ДУ ``КАІ''},
    pdfcreator={LaTeX},
    pdfproducer={LaTeX}
}

\setmainfont{Liberation Serif}

\setlength{\parskip}{1ex}
\setlength{\parindent}{1.25cm}

% Метадані документа
\title{Положення про органи студентського самоврядування}
\author{Державний університет ``Київський авіаційний інститут''}
\date{\today} % Поточна дата компіляції

% Можна додати пакет для кращого керування заголовками розділів, якщо потрібно
% \usepackage{titlesec}

\begin{document}

\begin{titlepage}
    \centering
    \vspace*{\fill} % Вертикальне центрування

    {\Huge\bfseries Положення}\par % Великий жирний заголовок
    \vspace{1em} % Відступ
    {\LARGE про органи студентського самоврядування}\par % Підзаголовок
    \vspace{0.5em} % Менший відступ
    {\large Державного некомерційного підприємства \\ ``Державний університет ``Київський авіаційний інститут''}\par % Повна назва

    \vspace*{\fill} % Вертикальне центрування
\end{titlepage}

% Додаємо глосарій термінів та абревіатур
\section*{Глосарій термінів та абревіатур}
\begin{description}[leftmargin=3cm,style=nextline]
    \item[ОСС] Органи студентського самоврядування.
    \item[КСУ] Конференція студентів Університету -- вищий представницький орган студентського самоврядування.
    \item[СР КАІ] Студентська рада Київського авіаційного інституту -- вищий виконавчий орган студентського самоврядування.
    \item[СУД] Студентська уповноважена делегація -- постійно діючий орган, відповідальний за контрольну діяльність в системі ОСС.
    \item[ЦВКс] Центральна виборча комісія студентів -- постійно діючий орган, відповідальний за організацію та проведення виборів до ОСС.
    \item[СР СМ] Студентська рада студмістечка -- орган, що координує діяльність ОСС гуртожитків.
    \item[СРФ/СРІ] Студентська рада факультету/інституту -- основний орган студентського самоврядування на рівні факультету/інституту.
    \item[СО] Студентські організації -- добровільні об'єднання студентів, створені за спільними інтересами.
\end{description}
\newpage

% Додаємо зміст
\renewcommand{\contentsname}{Зміст}
\tableofcontents
\newpage

% --- Підключення розділів --- 
\section*{Розділ I. Загальні положення}
\addcontentsline{toc}{section}{Розділ I. Загальні положення}

\subsection*{1.1. Статус Студентської ради КАІ}
\addcontentsline{toc}{subsection}{1.1. Статус Студентської ради КАІ}
    1.1.1. Студентська рада Київського авіаційного інституту (далі – СР КАІ) є постійно діючим вищим колегіальним виконавчим органом студентського самоврядування Державного університету "Київський авіаційний інститут" (далі – Університет).

    1.1.2. СР КАІ діє відповідно до Конституції України, Закону України "Про вищу освіту", Статуту Університету, Положення про органи студентського самоврядування Університету (далі – Положення про ОСС), цього Положення та інших нормативно-правових актів України.

    1.1.3. Це Положення визначає повноваження, порядок формування, структуру, порядок роботи та інші аспекти діяльності СР КАІ і діє відповідно до та на виконання Положення про ОСС.

\subsection*{1.2. Мета та основні завдання СР КАІ}
\addcontentsline{toc}{subsection}{1.2. Мета та основні завдання СР КАІ}
    1.2.1. Метою діяльності СР КАІ є ефективне представництво та захист прав та законних інтересів студентів Університету на загальноуніверситетському рівні, сприяння їхньому гармонійному розвитку та активній участі в житті Університету.

    1.2.2. Основними завданнями СР КАІ є:

        \begin{enumerate}[label=\alph*)]
            \item Представлення інтересів студентів перед адміністрацією Університету та її структурними підрозділами, у Вченій раді Університету та інших органах управління Університету.
            \item Забезпечення виконання рішень Конференції студентів Університету (КСУ) та реалізація завдань студентського самоврядування, визначених Положенням про ОСС.
            \item Координація діяльності органів студентського самоврядування факультетів/інститутів (СРФ/СРІ).
            \item Сприяння навчальній, науковій, культурно-просвітницькій, спортивній та соціальній діяльності студентів.
            \item Участь у вирішенні питань щодо поліпшення умов навчання, побуту та дозвілля студентів.
            \item Налагодження співпраці зі студентськими організаціями Університету, органами студентського самоврядування інших закладів вищої освіти та молодіжними організаціями.
            \item Інформування студентів про діяльність ОСС та важливі аспекти університетського життя.
        \end{enumerate}

\subsection*{1.3. Принципи діяльності СР КАІ}
\addcontentsline{toc}{subsection}{1.3. Принципи діяльності СР КАІ}
    1.3.1. СР КАІ здійснює свою діяльність на принципах, визначених Положенням про ОСС, зокрема: законності, добровільності, колегіальності, виборності, рівності прав студентів на участь у самоврядуванні, прозорості, організаційної самостійності.

    1.3.2. Ключовими принципами, що визначають виконавчий характер діяльності СР КАІ, є підзвітність перед КСУ та ефективність у реалізації покладених завдань та повноважень.

\subsection*{1.4. Взаємозв'язок з іншими суб'єктами}
\addcontentsline{toc}{subsection}{1.4. Взаємозв'язок з іншими суб'єктами}
    1.4.1. СР КАІ є підзвітною та підконтрольною Конференції студентів Університету.

    1.4.2. СР КАІ взаємодіє з іншими органами студентського самоврядування (Студентською уповноваженою делегацією (СУД), Центральною виборчою комісією студентів (ЦВКс), Студентською радою студмістечка (СР СМ), студентськими радами факультетів/інститутів (СРФ/СРІ)), адміністрацією Університету та студентськими організаціями у порядку, визначеному Положенням про ОСС та цим Положенням. 

\section*{Розділ II. Структура органів студентського самоврядування}

\subsection*{2.1. Система та рівні ОСС}
    2.1.1. Студентське самоврядування в Університеті здійснюється через єдину систему органів студентського самоврядування (ОСС), що діє на принципах законності, добровільності, колегіальності, виборності, підзвітності, прозорості та незалежності від адміністрації Університету, з метою забезпечення виконання завдань студентського самоврядування та захисту прав та інтересів студентів.

    2.1.2. Система ОСС має таку структуру рівнів:
        \begin{enumerate}[label=\alph*)]
            \item Загальноуніверситетський рівень;
            \item Локальний рівень (органи студентського самоврядування факультетів/інститутів та гуртожитків).
        \end{enumerate}

    2.1.3. Органи студентського самоврядування різних рівнів діють на засадах єдності мети, координації діяльності, субсидіарності (прийняття рішень на найнижчому можливому ефективному рівні), розмежування та делегування повноважень, а також ієрархічної підпорядкованості ОСС нижчого рівня ОСС вищого рівня в межах їхньої компетенції.

\subsection*{2.2. Вищий представницький орган ОСС}
    2.2.1. Вищим органом студентського самоврядування в Університеті є Конференція студентів Університету (КСУ).

    2.2.2. Статус, повноваження, порядок формування делегатів, скликання та процедури прийняття рішень КСУ визначаються Розділом ІІІ цього Положення.

\subsection*{2.3. Вищий виконавчий орган ОСС}
    2.3.1. Вищим виконавчим органом студентського самоврядування Університету є Студентська рада Київського авіаційного інституту (СР КАІ).

    2.3.2. Основні повноваження, завдання та напрями діяльності СР КАІ визначаються Положенням про Студентську раду Київського авіаційного інституту, яке затверджується КСУ.

    2.3.3. Порядок формування, структура та порядок роботи СР КАІ визначаються Положенням про Студентську раду Київського авіаційного інституту.

\subsection*{2.4. Інші загальноуніверситетські ОСС}
    2.4.1. Центральна виборча комісія студентів (ЦВКс) є постійно діючим органом, відповідальним за організацію та проведення виборів до ОСС. Статус та повноваження ЦВКс визначаються Положенням про ЦВКс, яке затверджується КСУ.

    2.4.2. Студентська уповноважена делегація (СУД) є постійно діючим органом, відповідальним за контрольну діяльність в системі ОСС. Статус та повноваження СУД визначаються Положенням про СУД, яке затверджується КСУ.

    2.4.3. Студентська рада студмістечка (СР СМ) є органом, що координує діяльність ОСС гуртожитків та представляє інтереси мешканців студмістечка на загальноуніверситетському рівні. Статус та повноваження СР СМ визначаються окремим Положенням про Студентську раду студмістечка, яке затверджується КСУ.

\subsection*{2.5. Органи студентського самоврядування факультетів/інститутів}
    2.5.1. Основним органом студентського самоврядування на рівні факультету/інституту є Студентська рада факультету/інституту.

    2.5.2. Повноваження, порядок формування та діяльність ОСС факультету/інституту визначаються Положенням про Студентську раду Київського авіаційного інституту, яке затверджується КСУ, та не повинні суперечити цьому Положенню.

    \subsubsection*{2.5.3. Порядок діяльності ОСС факультету/інституту при реорганізації структурного підрозділу}
        2.5.3.1. Підставою для початку процедур реорганізації ОСС факультету/інституту є офіційний Наказ Ректора Університету або Рішення Вченої ради Університету про відповідну реорганізацію структурного підрозділу.

        2.5.3.2. \textbf{При перейменуванні факультету/інституту:} Існуюча Студентська рада факультету (СРФ) автоматично змінює свою назву відповідно до нової назви структурного підрозділу та продовжує виконувати свої повноваження до закінчення строку, на який її було обрано.

        2.5.3.3. \textbf{При злитті/приєднанні факультетів/інститутів:}
            \begin{enumerate}[label=\alph*)]
                \item Студентські ради факультетів, що реорганізуються, продовжують діяльність протягом 1 (одного) місяця з дати офіційного об\'єднання (перехідний період) з метою завершення поточних справ, передачі документації та активів, після чого їхні повноваження припиняються.
                \item Конференція студентів Університету (КСУ) на найближчому засіданні після офіційного об\'єднання приймає рішення про доцільність створення єдиної Студентської ради для новоутвореного факультету (зазвичай, така рада створюється).
                \item У разі прийняття КСУ рішення про створення нової єдиної СРФ, Центральна виборча комісія студентів (ЦВКс) організовує та проводить вибори до її складу протягом 2 (двох) місяців з дати офіційного об\'єднання факультетів/інститутів.
                \item Делегати КСУ та представники в інших органах, обрані від студентів факультетів, що реорганізуються, продовжують виконувати свої повноваження до моменту обрання нових представників від новоутвореного факультету.
            \end{enumerate}

        2.5.3.4. \textbf{При розформуванні/ліквідації факультету/інституту (без злиття):}
            \begin{enumerate}[label=\alph*)]
                \item Студентська рада факультету, що ліквідується, продовжує діяльність протягом 2 (двох) тижнів з дати офіційної ліквідації (перехідний період) з метою завершення поточних справ, передачі документації та активів, після чого її повноваження припиняються.
                \item Уся документація, активи та справи розформованої СРФ передаються до Студентської ради Київського авіаційного інституту (СР КАІ).
                \item Студенти, переведені на інші факультети/інститути, автоматично підпадають під представництво та юрисдикцію Студентських рад відповідних факультетів, куди їх було переведено.
            \end{enumerate}

        2.5.3.5. \textbf{Тимчасове управління від СР КАІ:}
            \begin{enumerate}[label=\alph*)]
                \item У випадку, якщо проведення виборів до новоствореної СРФ (відповідно до п. 2.5.3.3.в) є неможливим у встановлений 2-місячний термін, КСУ за поданням СР КАІ або ЦВКс може прийняти рішення про запровадження тимчасового управління справами студентського самоврядування відповідного факультету.
                \item Тимчасове управління здійснюється Тимчасовим комітетом, що формується та призначається СР КАІ.
                \item Повноваження Тимчасового комітету за замовчуванням обмежуються підтриманням базової операційної діяльності, забезпеченням мінімально необхідного представництва інтересів студентів факультету та першочерговою організацією та сприянням проведенню виборів до СРФ. За окремим рішенням КСУ Тимчасовому комітету можуть бути надані розширені повноваження.
                \item Тимчасове управління триває до моменту обрання та початку роботи нової СРФ, але не може перевищувати 3 (трьох) місяців. Якщо протягом цього терміну вибори не проведено, питання про подальші дії виноситься на розгляд КСУ.
                \item Тимчасовий комітет є підзвітним СР КАІ та КСУ.
            \end{enumerate}

\subsection*{2.6. Органи студентського самоврядування гуртожитків}
    2.6.1. Основним органом студентського самоврядування на рівні гуртожитку є Студентська рада гуртожитку.

    2.6.2. Повноваження, порядок формування та діяльність ОСС гуртожитку визначаються Положенням про Студентську раду студмістечка, яке затверджується КСУ, та не повинні суперечити цьому Положенню.

\subsection*{2.7. Взаємодія між ОСС}
    2.7.1. Взаємодія між ОСС різних рівнів та одного рівня будується на принципах координації, співпраці, взаємної поваги, підзвітності відповідно до ієрархії, визначеної цим Положенням, та розмежування повноважень. Рішення ОСС вищого рівня є обов'язковими для ОСС нижчого рівня в межах їх компетенції.

    2.7.2. Механізмами взаємодії ОСС є, зокрема: проведення спільних засідань керівних органів; створення спільних робочих груп та комісій за напрямками діяльності; регулярний обмін інформацією, планами та звітами; взаємне делегування представників до складу ОСС іншого рівня або органу; проведення координаційних нарад голів ОСС.

    2.7.3. Спори між ОСС різних рівнів або одного рівня щодо компетенції, виконання рішень або інших питань їхньої діяльності вирішуються шляхом переговорів. У разі недосягнення згоди, спір може бути передано на розгляд Студентської уповноваженої делегації (СУД) для надання висновку або медіації. Порядок розгляду спорів СУД визначається Положенням про СУД.

\subsection*{2.8. Студентські організації (СО)}
    2.8.1. Студентські організації (СО) є добровільними об'єднаннями студентів за інтересами, що діють незалежно від системи та ієрархії органів студентського самоврядування (ОСС).

    2.8.2. Порядок створення, реєстрації (визнання), діяльності та припинення діяльності СО, їх права та обов'язки визначаються окремим Положенням про студентські організації Університету, яке затверджується Конференцією студентів Університету (КСУ).

    2.8.3. Органи студентського самоврядування взаємодіють зі студентськими організаціями, що визнані (зареєстровані) у порядку, встановленому Положенням про студентські організації Університету.

\section*{Розділ III. Конференція студентів Університету}
\addcontentsline{toc}{section}{Розділ III. Конференція студентів Університету}

\subsection*{3.1. Статус та загальні засади діяльності КСУ}
\addcontentsline{toc}{subsection}{3.1. Статус та загальні засади діяльності КСУ}
    3.1.1. Конференція студентів Університету (КСУ) є вищим представницьким органом студентського самоврядування Університету.

    3.1.2. Чергова КСУ скликається Спікером КСУ не рідше одного разу на рік. Позачергова КСУ може бути скликана за ініціативою Студентської ради Київського авіаційного інституту (СР КАІ), Студентської ради студмістечка (СР СМ), Студентської уповноваженої делегації (СУД), Центральної виборчої комісії студентів (ЦВКс), Спікера КСУ, не менше третини від загального складу делегатів КСУ або за ініціативою не менше 5\% студентів від загальної кількості студентів Університету.

    3.1.3. Діяльність КСУ здійснюється на принципах колегіальності, виборності делегатів, рівності прав делегатів, гласності та відкритості.

    3.1.4. За рішенням КСУ, прийнятим більшістю голосів від загального складу делегатів, окремі питання можуть розглядатися на закритому засіданні. Закриті засідання можуть проводитися при розгляді персональних питань, питань, що містять конфіденційну інформацію, або в інших випадках, коли цього вимагає захист прав та інтересів осіб чи забезпечення ефективного обговорення. Рішення про проведення закритого засідання та його результати фіксуються в протоколі КСУ.

    3.1.5. Представники адміністрації Університету мають право брати участь у засіданнях КСУ з правом дорадчого голосу.

\subsection*{3.2. Повноваження КСУ}
\addcontentsline{toc}{subsection}{3.2. Повноваження КСУ}
    До виключних повноважень КСУ належить:
        \begin{enumerate}[label=\alph*)]
            \item Затвердження Положення про студентське самоврядування Університету та внесення змін до нього;
            \item Затвердження положень, що регламентують створення та діяльність органів студентського самоврядування та інших структур, передбачених цим Положенням (включаючи, але не обмежуючись, положеннями про СР КАІ, СР СМ, СУД, ЦВКс, студентські організації), та внесення змін до них;
            \item Заслуховування та затвердження звітів СР КАІ, СР СМ, СУД, ЦВКс та інших органів, підзвітних КСУ;
            \item Затвердження результатів виборів Голови СР КАІ та Голови СР СМ, проведених у встановленому порядку;
            \item Обрання та припинення повноважень членів СУД та ЦВКс;
            \item Визначення структури та стратегічних напрямів діяльності ОСС Університету;
            \item Затвердження єдиного кошторису (бюджету) ОСС Університету та звіту про його виконання. У разі, якщо проєкт кошторису не був погоджений Студентською уповноваженою делегацією (СУД), КСУ може затвердити його лише за умови, якщо за таке рішення проголосувало не менше двох третин (2/3) від загального складу делегатів КСУ;
            \item Розгляд питання про висловлення недовіри керівникам та членам СР КАІ, СР СМ, СУД, ЦВКс, а також іншим членам ОСС. Питання про висловлення недовіри за підставами, розгляд яких належить до компетенції СУД згідно з Положенням про СУД, розглядається КСУ, як правило, за поданням СУД або в порядку апеляції на рішення СУД. В інших випадках питання ініціюється делегатом КСУ або СУД. Рішення про висловлення недовіри приймається не менше ніж двома третинами голосів від загального складу делегатів КСУ та має наслідком автоматичне припинення повноважень відповідної особи;
            \item Призначення виконувачів обов'язків (в.о.) посадових осіб ОСС на термін більше 1 (одного) місяця;
            \item Затвердження або відхилення призначення виконувачів обов'язків (в.о.) посадових осіб ОСС, призначених Студентською уповноваженою делегацією (СУД) у випадках, передбачених Положенням про СУД. Рішення про затвердження приймається простою більшістю голосів від загального складу делегатів КСУ;
            \item Визначення розміру членських внесків студентів (якщо вони запроваджуються);
            \item Вирішення інших питань, віднесених до компетенції КСУ цим Положенням, Статутом Університету або законодавством України.
        \end{enumerate}

\subsection*{3.3. Порядок формування делегатів КСУ}
\addcontentsline{toc}{subsection}{3.3. Порядок формування делегатів КСУ}
    3.3.1. Делегати КСУ обираються (делегуються) від студентів факультетів, інститутів та гуртожитків на основі пропорційного представництва та рівних можливостей.

    3.3.2. Квота представництва для обрання делегатів КСУ встановлюється Центральною виборчою комісією студентів (ЦВКс) перед кожними виборами і становить: 1 (один) делегат від 500 студентів факультету/інституту; 0,5\% від кількості мешканців гуртожитку, але не більше 5 (п'яти) делегатів від одного гуртожитку.

    3.3.3. Порядок обрання (делегування) представників на КСУ від студентів факультетів, інститутів, гуртожитків визначається Положенням про Центральну виборчу комісію студентів (ЦВКс).

    3.3.4. Строк повноважень делегатів КСУ становить 1 (один) рік.

    3.3.5. Голова Студентської ради Київського авіаційного інституту (СР КАІ) та Голова Студентської ради студмістечка (СР СМ) є членами КСУ за посадою (ex officio). Вони мають право голосу на засіданнях КСУ лише у випадку, якщо вони були обрані на свої посади у встановленому порядку відповідно до положень про СР КАІ та СР СМ, і не є тимчасово виконуючими обов'язки. Члени КСУ за посадою, які мають право голосу, враховуються при визначенні кворуму та підрахунку голосів як частина загального складу КСУ.

\subsection*{3.4. Порядок скликання та проведення КСУ}
\addcontentsline{toc}{subsection}{3.4. Порядок скликання та проведення КСУ}
    3.4.1. Порядок скликання КСУ визначено у пункті 3.1.2 цього Положення.

    3.4.2. Повідомлення про дату, час, місце проведення та проєкт порядку денного КСУ оприлюднюється Спікером КСУ (або органом, що скликає позачергову КСУ, окрім СУД у невідкладних випадках) на офіційних інформаційних ресурсах ОСС не пізніше ніж за 10 (десять) календарних днів до її проведення. У невідкладних випадках, що потребують термінового розгляду, Студентська уповноважена делегація (СУД) має право скликати позачергову КСУ, повідомивши про це не пізніше ніж за 3 (три) календарні дні до дати проведення, з обов'язковим обґрунтуванням терміновості.

    3.4.3. КСУ є правомочною, якщо на її засіданні зареєстровано більше половини від загальної кількості обраних (затверджених) делегатів.

    3.4.4. На першому засіданні новообраної КСУ зі свого складу обираються Спікер КСУ та Секретар КСУ, які діють на постійній основі протягом усього строку повноважень КСУ даного скликання. Рішення про обрання приймається більшістю голосів від загального складу делегатів КСУ.

    3.4.5. Рішення КСУ приймаються відкритим поіменним голосуванням більшістю голосів від загального складу делегатів КСУ, якщо інша процедура або інша кількість голосів не передбачена цим Положенням або Регламентом КСУ (зокрема, для питань, що потребують 2/3 голосів).

    3.4.6. Хід засідання КСУ протоколюється Секретарем КСУ. Рішення КСУ оформлюються протоколом, який підписується Спікером КСУ та Секретарем КСУ, та підлягають оприлюдненню на офіційних інформаційних ресурсах ОСС протягом 3 (трьох) робочих днів з дня проведення засідання.

    3.4.7. Контроль за виконанням рішень КСУ покладається на Спікера КСУ та на органи студентського самоврядування, до компетенції яких належить виконання відповідних рішень.

\subsection*{3.5. Робочі органи КСУ}
\addcontentsline{toc}{subsection}{3.5. Робочі органи КСУ}
    3.5.1. КСУ має право створювати зі складу своїх делегатів постійні та тимчасові комісії для попереднього розгляду та підготовки питань, що належать до її компетенції.

    3.5.2. Перелік постійних комісій, порядок їх формування, повноваження та організація роботи визначаються Регламентом КСУ або окремими рішеннями КСУ.

\section*{Розділ IV. Участь в органах студентського самоврядування}

\subsection*{4.1. Загальні положення}
    4.1.1. Кожен студент Університету має право брати участь у студентському самоврядуванні на умовах, визначених Законом України ``Про вищу освіту'', Статутом Університету та цим Положенням.

    4.1.2. Участь студентів у студентському самоврядуванні є добровільною. Ніхто не може бути примушений до участі або неучасті в діяльності органів студентського самоврядування.

    4.1.3. Студенти мають право вільно обирати та бути обраними до складу ОСС відповідно до порядку, встановленого цим Положенням (активне та пасивне виборче право).

    4.1.4. Обмеження щодо участі в ОСС можуть бути встановлені виключно на підставах, передбачених законодавством України та цим Положенням.

    4.1.5. Усі члени ОСС мають рівні права та обов'язки, якщо інше не передбачено цим Положенням для окремих посад чи функцій.

\subsection*{4.2. Вибори до ОСС}
    4.2.1. Вибори до органів студентського самоврядування організовуються та проводяться Центральною виборчою комісією студентів (ЦВКс).

    4.2.2. Порядок організації та проведення виборів до ОСС, вимоги до кандидатів, процедури висування, голосування, встановлення результатів та інші виборчі процедури визначаються Положенням про Центральну виборчу комісію студентів Університету.

    4.2.3. ЦВКс забезпечує прозорість, чесність та демократичність виборчого процесу.

\subsection*{4.3. Набуття членства та повноважень в ОСС}
    \subsubsection*{4.3.1. Набуття повноважень виборними членами ОСС}
        4.3.1.1. Повноваження обраного члена ОСС починаються з моменту офіційного оголошення результатів виборів Центральною виборчою комісією студентів, якщо інше не встановлено рішенням ЦВКс або Конференцією студентів Університету (КСУ).

        4.3.1.2. Факт набуття повноважень може бути засвідчений відповідним рішенням органу студентського самоврядування або ЦВКс.
    \subsubsection*{4.3.2. Набуття повноважень призначеними членами ОСС}
        4.3.2.1. Особи, які входять до складу ОСС за посадою (ex officio) або призначаються до складу ОСС відповідно до процедур, визначених Статутом Університету, цим Положенням або положеннями про відповідні органи ОСС, набувають повноважень з моменту прийняття відповідного рішення уповноваженим органом або особою.

        4.3.2.2. Рішення про призначення має містити дату початку повноважень.
    \subsubsection*{4.3.3. Загальні процедури набуття статусу}
        4.3.3.1. Новообрані або призначені члени ОСС можуть складати присягу, текст та порядок складання якої визначається відповідним органом ОСС.

        4.3.3.2. Орган студентського самоврядування веде реєстр своїх членів. Відповідальність за ведення та актуалізацію реєстру покладається на Секретаря відповідного ОСС, якщо інше не передбачено положенням про цей орган або рішенням його керівника (Голови).

\subsection*{4.4. Припинення повноважень члена ОСС}
    \subsubsection*{4.4.1. Підстави для дострокового припинення повноважень}
        4.4.1.1. Власне бажання (відставка з посади).

        4.4.1.2. Неможливість виконувати обов'язки за станом здоров'я, підтверджена медичним висновком.

        4.4.1.3. Набрання законної сили обвинувальним вироком суду щодо нього.

        4.4.1.4. Порушення Статуту Університету або цього Положення.

        4.4.1.5. Систематичне невиконання без поважних причин обов'язків, покладених на нього як на члена ОСС.

        4.4.1.6. Висловлення недовіри Конференцією студентів Університету (КСУ) (застосовується до виборних посад).

        4.4.1.7. Відкликання виборцями у порядку, встановленому відповідним положенням (застосовується до виборних посад).

        4.4.1.8. Прийняття рішення про звільнення органом, що призначив (застосовується до призначених посад).

        4.4.1.9. Рішення Студентської уповноваженої делегації (СУД) про припинення повноважень у випадках та порядку, передбачених Положенням про СУД.
    \subsubsection*{4.4.2. Процедура дострокового припинення повноважень}
        4.4.2.1. Припинення повноважень за власним бажанням (відставка) відбувається на підставі письмової заяви члена ОСС, поданої до керівника відповідного ОСС або до органу, який його обрав/призначив. Повноваження припиняються з дати, зазначеної в заяві, але не раніше дня її подання, або з моменту прийняття рішення про відставку уповноваженим органом, якщо така процедура передбачена.

        4.4.2.2. У випадках, передбачених підпунктами 4.4.1.2 – 4.4.1.5, 4.4.1.8 цього Положення, питання про дострокове припинення повноважень ініціюється керівником відповідного ОСС, групою членів ОСС, або органом, що обрав/призначив члена ОСС.

        4.4.2.3. Питання про дострокове припинення повноважень розглядається:
            \begin{enumerate}[label=\alph*)]
                \item Конференцією студентів Університету (КСУ) – стосовно виборних членів ОСС рівня Університету та голів студентських рад факультетів/інститутів/гуртожитків.
                \item Засіданням відповідного органу студентського самоврядування (студентської ради факультету/інституту/гуртожитку тощо) – стосовно призначених членів цього органу або інших виборних членів цього органу, якщо інше не передбачено положенням про цей орган.
                \item Органом, що призначив члена ОСС, – якщо це передбачено процедурою призначення або положенням про відповідний орган.
            \end{enumerate}

        4.4.2.4. Член ОСС, стосовно якого розглядається питання про припинення повноважень, має бути повідомлений про дату, час та місце засідання і має право бути присутнім та надавати пояснення (окрім випадків об'єктивної неможливості, наприклад, набрання чинності вироком суду).

        4.4.2.5. Рішення про дострокове припинення повноважень приймається шляхом голосування відповідно до процедури, встановленої цим Положенням або положенням про відповідний орган ОСС. Рішення вважається прийнятим, якщо за нього проголосувала більшість голосів від загального складу відповідного органу, якщо інша квота не встановлена положенням про цей орган.

        4.4.2.6. Процедури висловлення недовіри КСУ (пп. 4.4.1.6) та відкликання виборцями (пп. 4.4.1.7) визначаються окремими положеннями або Положенням про ЦВКс.

        4.4.2.7. Факт припинення повноважень фіксується у протоколі засідання відповідного органу та доводиться до відома члена ОСС, повноваження якого припинено.

        4.4.2.8. Студентська уповноважена делегація (СУД) може відсторонити члена ОСС від виконання повноважень або прийняти рішення про припинення його повноважень у випадках та порядку, передбачених Положенням про Студентську уповноважену делегацію (пп. 4.4.1.9).

        4.4.2.9. Припинення членства в ОСС з підстав, визначених у Розділі 4.5 цього Положення, автоматично припиняє повноваження члена ОСС без необхідності ухвалення окремого рішення згідно з процедурами цього Розділу.
    \subsubsection*{4.4.3. Припинення повноважень у зв'язку із закінченням строку}
        4.4.3.1. Повноваження члена ОСС припиняються автоматично в день закінчення строку, на який його було обрано або призначено.

        4.4.3.2. Окремого рішення про припинення повноважень у зв'язку із закінченням строку не потребується, якщо інше не встановлено положенням про відповідний ОСС.

\subsection*{4.5. Припинення членства в ОСС}
    \subsubsection*{4.5.1. Підстави для припинення членства}
        4.5.1.1. Завершення навчання або відрахування з Університету.

        4.5.1.2. Добровільний вихід зі складу ОСС (повне припинення участі).

        4.5.1.3. Втрата статусу здобувача вищої освіти з інших причин, передбачених законодавством або Статутом Університету.

        4.5.1.4. Смерть особи.
    \subsubsection*{4.5.2. Процедура припинення членства}
        4.5.2.1. Припинення членства в ОСС у зв'язку із завершенням навчання, відрахуванням з Університету, втратою статусу здобувача вищої освіти з інших причин або смертю відбувається автоматично з моменту настання відповідної події (наприклад, дата наказу про відрахування, дата смерті, зазначена у свідоцтві про смерть).

        4.5.2.2. Секретар відповідного ОСС на підставі офіційної інформації (наприклад, наказу про відрахування, даних відділу кадрів, копії свідоцтва про смерть тощо) фіксує факт припинення членства у реєстрі членів ОСС.

        4.5.2.3. Добровільний вихід зі складу ОСС здійснюється на підставі письмової заяви студента, поданої керівнику або секретарю відповідного ОСС. Членство припиняється з дати подання заяви, якщо в заяві не вказано іншу дату.

        4.5.2.4. Особа, членство якої в ОСС припинено, втрачає всі права та обов'язки, пов'язані зі статусом члена ОСС.


\section*{Розділ V. Фінансування та матеріально-технічне забезпечення}
\addcontentsline{toc}{section}{Розділ V. Фінансування та матеріально-технічне забезпечення}

\subsection*{5.1. Джерела фінансування}
\addcontentsline{toc}{subsection}{5.1. Джерела фінансування}
    5.1.1. Фінансовою основою діяльності органів студентського самоврядування (ОСС) є кошти, отримані з таких джерел:

        \begin{enumerate}[label=\alph*)]
            \item Кошти, визначені Вченою радою Університету в розмірі не менш як 0,5 відсотка власних надходжень, отриманих Університетом від основної діяльності;
            \item Членські внески студентів, розмір яких встановлюється Конференцією студентів Університету (КСУ), але не може перевищувати 1 відсотка прожиткового мінімуму, встановленого законодавством, на одну особу на місяць (рішення про запровадження та розмір внесків приймається КСУ);
            \item Благодійна допомога, гранти, дарунки, спонсорські кошти, отримані від фізичних та юридичних осіб відповідно до законодавства України;
            \item Інші надходження, не заборонені законодавством України.
        \end{enumerate}

\subsection*{5.2. Порядок використання коштів}
\addcontentsline{toc}{subsection}{5.2. Порядок використання коштів}
    5.2.1. Кошти ОСС обліковуються та зберігаються на окремому субрахунку Університету.

    5.2.2. Використання коштів ОСС здійснюється відповідно до єдиного кошторису (бюджету) ОСС, затвердженого Конференцією студентів Університету (КСУ), на виконання завдань та напрямів діяльності, передбачених цим Положенням та статутними документами відповідних ОСС.

    5.2.3. Координацію фінансової діяльності, збір та перевірку проєктів кошторисів ОСС факультетів/інститутів та СР КАІ здійснює Фінансовий комітет СР КАІ. Аналогічні функції щодо ОСС гуртожитків та СР СМ виконує Фінансовий комітет СР СМ. Порядок їхньої діяльності визначається положеннями про СР КАІ та СР СМ. Контроль за цільовим використанням коштів усіх ОСС здійснюється Студентською уповноваженою делегацією (СУД).

    5.2.4. ОСС забезпечують прозорість використання фінансових ресурсів шляхом оприлюднення кошторисів та звітів про їх виконання у порядку, визначеному Розділом VI цього Положення.

\subsection*{5.3. Матеріально-технічне забезпечення}
\addcontentsline{toc}{subsection}{5.3. Матеріально-технічне забезпечення}
    5.3.1. Адміністрація Університету надає органам студентського самоврядування необхідні для їхньої діяльності приміщення з відповідним обладнанням та інвентарем на безоплатній основі.

    5.3.2. Адміністрація Університету забезпечує ОСС доступом до мережі Інтернет, телефонного зв'язку, можливості користування оргтехнікою та іншими необхідними ресурсами на умовах, що визначаються угодою між ОСС та адміністрацією або окремими рішеннями.

    5.3.3. ОСС дбайливо ставляться до наданого Університетом майна. ОСС також мають право володіти та користуватися майном, набутим за власні кошти або отриманим як благодійна допомога, для здійснення своїх статутних завдань.

\subsection*{5.4. Грантове фінансування та спонсорство}
\addcontentsline{toc}{subsection}{5.4. Грантове фінансування та спонсорство}
    5.4.1. ОСС мають право самостійно або спільно з іншими організаціями брати участь у конкурсах на отримання грантів від українських та міжнародних фондів і організацій.

    5.4.2. ОСС можуть отримувати благодійну допомогу та спонсорські кошти від фізичних та юридичних осіб відповідно до мети своєї діяльності та вимог законодавства України.

    5.4.3. Право на укладання угод про отримання благодійної допомоги чи спонсорських коштів та управління такими коштами належить виключно Студентській раді Київського авіаційного інституту (СР КАІ) та Студентській раді студмістечка (СР СМ) через їхні уповноважені комітети або посадових осіб. Органи студентського самоврядування факультетів/інститутів та гуртожитків не мають права самостійно залучати та розпоряджатися такими коштами.

    5.4.4. Облік та використання грантових коштів, благодійної допомоги та спонсорських надходжень здійснюються відповідно до умов їх надання, вимог законодавства України та внутрішніх положень ОСС, забезпечуючи прозорість та цільове призначення.

\subsection*{5.5. Фінансова взаємодія ОСС зі студентськими організаціями}
\addcontentsline{toc}{subsection}{5.5. Фінансова взаємодія ОСС зі студентськими організаціями}
    5.5.1. Визнані (зареєстровані) студентські організації (СО) мають право подавати проєкти на конкурсній основі для отримання цільового фінансування з коштів єдиного бюджету ОСС на реалізацію конкретних заходів, що відповідають цілям та завданням студентського самоврядування, визначеним у Розділі I цього Положення.

    5.5.2. Порядок проведення конкурсу проєктів СО, критерії відбору, обсяги фінансування та порядок звітності визначаються відповідним органом ОСС (СР КАІ, СР СМ) та/або Положенням про студентські організації Університету.

    5.5.3. Надання коштів з бюджету ОСС для покриття операційних витрат (поточна діяльність, утримання тощо) студентських організацій не передбачається. СО діють на засадах самофінансування або залучення зовнішніх ресурсів.

\section*{Розділ VI. Контроль та звітність}

\subsection*{6.1. Контроль за діяльністю ОСС}
    6.1.1. Контроль за діяльністю ОСС здійснюється з метою забезпечення дотримання ними законодавства України, Статуту Університету, цього Положення та інших нормативних актів, що регулюють діяльність студентського самоврядування.

    6.1.2. Студентська уповноважена делегація (СУД) є основним органом контролю в системі студентського самоврядування. Повноваження СУД щодо контролю за дотриманням нормативних документів ОСС, фінансовою діяльністю (у випадках, передбачених Положенням про СУД), розгляду скарг та спорів, а також відповідні інструменти контролю визначаються Положенням про СУД.

    6.1.3. Здійснення ОСС звітності у порядку, передбаченому пунктом 6.2 цього Положення, та забезпечення прозорості діяльності згідно з пунктом 6.3 цього Положення є формами громадського контролю за діяльністю органів студентського самоврядування.

\subsection*{6.2. Звітність ОСС}
    6.2.1. Органи студентського самоврядування є підзвітними студентській спільноті Університету та Конференції студентів Університету (КСУ).

    6.2.2. Усі органи студентського самоврядування зобов'язані звітувати про свою діяльність перед Конференцією студентів Університету (КСУ) не рідше одного разу на рік. Звітування відбувається шляхом усної доповіді керівника (або уповноваженої особи) ОСС та подання письмового звіту під час засідання КСУ.

    6.2.3. Кожен ОСС зобов'язаний створити та підтримувати офіційні інформаційні ресурси (канал в месенджері Telegram, сховище документів Google Drive в межах університетського робочого простору та/або інші ресурси, визначені відповідним ОСС) для інформування студентів про свою діяльність та оприлюднення документів відповідно до пункту 6.3 цього Положення.

    6.2.4. Органи студентського самоврядування факультетів, інститутів, гуртожитків тощо, окрім звітування перед КСУ, звітують перед студентською спільнотою відповідного факультету, інституту, гуртожитку у порядку та строки, визначені положеннями про ці ОСС, але не рідше одного разу на рік.

\subsection*{6.3. Прозорість та відкритість діяльності ОСС} 
    6.3.1. Статут Університету в частині студентського самоврядування, це Положення, а також зміни до них, підлягають обов'язковому оприлюдненню на офіційних інформаційних ресурсах відповідних ОСС, таких як Google Drive та визначений інформаційний канал, протягом 3 робочих днів з моменту їх затвердження.

    6.3.2. Рішення органів студентського самоврядування (протоколи засідань, рішення, накази тощо) підлягають оприлюдненню на офіційних інформаційних ресурсах відповідного ОСС протягом 3 робочих днів з моменту їх прийняття/підписання, за винятком інформації з обмеженим доступом, перелік та порядок поводження з якою визначається Положенням про СУД та законодавством України.

    6.3.3. Звіти ОСС про свою діяльність підлягають обов'язковому розміщенню на офіційних інформаційних ресурсах відповідного ОСС протягом 3 робочих днів з моменту їх затвердження або представлення.

    6.3.4. Кожен студент має право на вільний доступ до інформації про діяльність ОСС, оприлюдненої відповідно до пунктів 6.3.1 – 6.3.3 цього Положення, через офіційні інформаційні ресурси ОСС.

\section*{Розділ VII. Взаємодія з іншими структурами}
\addcontentsline{toc}{section}{Розділ VII. Взаємодія з іншими структурами}

\subsection*{7.1. Загальні засади взаємодії}
\addcontentsline{toc}{subsection}{7.1. Загальні засади взаємодії}
    7.1.1. Органи студентського самоврядування (ОСС) взаємодіють з адміністрацією Університету, іншими структурними підрозділами Університету, студентськими та молодіжними організаціями в Україні та за кордоном з метою ефективного виконання своїх завдань, представництва та захисту прав і інтересів студентів, спільного розвитку студентського середовища, залучення додаткових ресурсів та формування позитивного іміджу Університету.

    7.1.2. Взаємодія ОСС з іншими структурами здійснюється на принципах партнерства, взаємної поваги, конструктивного діалогу, законності, прозорості та відповідальності.

\subsection*{7.2. Взаємодія з адміністрацією Університету}
\addcontentsline{toc}{subsection}{7.2. Взаємодія з адміністрацією Університету}
    7.2.1. Взаємодія ОСС з адміністрацією Університету здійснюється шляхом: проведення регулярних зустрічей та консультацій; участі представників ОСС у засіданнях робочих та дорадчих органів адміністрації; створення спільних робочих груп для вирішення конкретних питань; взаємного обміну інформацією та документами; узгодження планів та заходів, що стосуються студентського життя.

    7.2.2. ОСС мають право звертатися до адміністрації Університету та її структурних підрозділів з пропозиціями, запитами, заявами з питань, що належать до їхньої компетенції. Порядок подання таких звернень визначається загальним порядком діловодства Університету з урахуванням вимог, встановлених регламентами відповідних ОСС.

    7.2.3. Адміністрація Університету розглядає звернення та пропозиції ОСС у встановлені законодавством або Статутом Університету строки. Рішення адміністрації, що стосуються прав та інтересів студентів, приймаються після обов'язкового розгляду пропозицій відповідних ОСС. Представники ОСС мають право бути запрошеними на засідання відповідного органу адміністрації під час розгляду їхніх пропозицій та надавати пояснення. У разі неврахування або часткового врахування пропозицій ОСС, відповідний орган адміністрації надає ОСС письмове обґрунтоване пояснення причин такого рішення у 10-денний робочий строк. Детальний порядок погодження окремих рішень (наприклад, щодо відрахування, поселення/виселення з гуртожитку) може визначатися відповідними положеннями Університету та положеннями про СР КАІ та СР СМ.

    7.2.4. Спірні питання, що виникають у процесі взаємодії між ОСС та адміністрацією Університету, вирішуються шляхом переговорів, консультацій, створення узгоджувальних комісій, які формуються на паритетних засадах представниками ОСС та адміністрації, порядок роботи яких визначається за згодою сторін, або в іншому порядку, передбаченому Статутом Університету та законодавством.

    7.2.5. Рішення адміністрації з наступних питань \textbf{не можуть бути прийняті без попереднього погодження} з відповідними органами студентського самоврядування:

        \begin{enumerate}[label=\arabic*)]
            \item Відрахування студентів з Університету та їх поновлення на навчання -- погоджується із СР КАІ та Студентською радою відповідного факультету/інституту;
            \item Переведення осіб, які навчаються в Університеті за державним замовленням, на навчання за контрактом за рахунок коштів фізичних (юридичних) осіб -- погоджується із СР КАІ та Студентською радою відповідного факультету/інституту;
            \item Переведення осіб, які навчаються в Університеті за рахунок коштів фізичних (юридичних) осіб, на навчання за державним замовленням -- погоджується із СР КАІ та Студентською радою відповідного факультету/інституту;
            \item Призначення заступника декана факультету/директора інституту з виховної роботи (або аналогічної посади), заступника керівника (проректора) Університету, до сфери відповідальності якого належать питання студентського життя -- погоджується із Головою Студентської ради відповідного факультету/інституту та Головою СР КАІ відповідно;
            \item Поселення осіб, які навчаються в Університеті, у гуртожиток і виселення їх із гуртожитку -- погоджується із Головою СР СМ та Головою Студентської ради відповідного гуртожитку;
            \item Затвердження правил внутрішнього розпорядку Університету в частині, що стосується осіб, які навчаються в Університеті -- погоджується із Головою СР КАІ;
            \item Діяльність студентських містечок та гуртожитків для проживання осіб, які навчаються у закладі вищої освіти (включаючи затвердження правил проживання, розподіл місць, встановлення розміру плати за проживання) -- погоджується із Головою СР СМ.
        \end{enumerate}

\subsection*{7.3. Взаємодія зі студентськими та молодіжними організаціями}
\addcontentsline{toc}{subsection}{7.3. Взаємодія зі студентськими та молодіжними організаціями}
    7.3.1. ОСС розвивають співпрацю з міжнародними, всеукраїнськими та регіональними студентськими та молодіжними організаціями, асоціаціями з метою обміну досвідом, реалізації спільних проєктів, представлення інтересів студентства Університету.

    7.3.2. ОСС підтримують молодіжні ініціативи, спрямовані на розвиток студентства, та можуть брати участь у реалізації соціальних проєктів у партнерстві з іншими організаціями.

    7.3.3. Взаємодія ОСС із визнаними (зареєстрованими) студентськими організаціями (СО) Університету відбувається на засадах партнерства та координації діяльності і може включати: спільну організацію заходів; обмін інформацією щодо планів та результатів діяльності; створення спільних робочих груп для вирішення конкретних питань; запрошення представників СО на засідання ОСС з правом дорадчого голосу при розгляді питань, що безпосередньо стосуються діяльності відповідної СО. Можливість фінансової підтримки проєктів СО з боку ОСС визначається Розділом V цього Положення.

\subsection*{7.4. Участь у програмах та проєктах}
\addcontentsline{toc}{subsection}{7.4. Участь у програмах та проєктах}
    7.4.1. ОСС беруть участь у розробці та реалізації програм Університету, спрямованих на покращення умов навчання, побуту, дозвілля, розвитку студентів.

    7.4.2. ОСС можуть брати участь у національних та міжнародних програмах і проєктах, що стосуються студентського самоврядування, мобільності, молодіжної політики та інших актуальних для студентства питань.

\section*{Розділ VIII. Прикінцеві та перехідні положення}

\subsection*{8.1. Порядок внесення змін та доповнень}
    8.1.1. Зміни та доповнення до цього Положення вносяться у порядку, визначеному пунктом 1.3.2 цього Положення.

    8.1.2. Право ініціювати внесення змін та доповнень до цього Положення мають Конференція студентів Університету (КСУ), Студентська рада Київського авіаційного інституту (СР КАІ), Студентська рада студмістечка (СР СМ) та Студентська уповноважена делегація (СУД).

\subsection*{8.2. Набуття чинності}
    8.2.1. Це Положення набуває чинності з дати його офіційного оприлюднення на інформаційних ресурсах ОСС Університету.

    8.2.2. З моменту набуття чинності цим Положенням, Положення ``Про Тимчасовий перехідний комітет'' від 09.12.2024 та інші нормативні акти ОСС, що суперечать цьому Положенню, втрачають чинність.

\subsection*{8.3. Тлумачення Положення}
    8.3.1. Офіційне тлумачення норм цього Положення надається Студентською уповноваженою делегацією (СУД). Рішення СУД щодо тлумачення є обов'язковим для всіх ОСС.

\subsection*{8.4. Перехідні положення}
    8.4.1. Органи студентського самоврядування (включаючи Тимчасовий перехідний комітет), сформовані до набуття чинності цим Положенням, припиняють свої повноваження з моменту набуття чинності цим Положенням.

    8.4.2. Перші вибори органів студентського самоврядування відповідно до цього Положення (включаючи формування першого складу ЦВКс та скликання першої КСУ) мають бути проведені протягом одного місяця з дня набуття чинності цим Положенням у порядку, затвердженому Загальними зборами студентів Університету.

    8.4.3. Конференція студентів Університету на своєму першому засіданні після набуття чинності цим Положенням має затвердити Положення про Студентську раду Київського авіаційного інституту, Положення про Студентську раду студмістечка, Положення про Студентську уповноважену делегацію та Положення про Центральну виборчу комісію студентів протягом трьох місяців.

\end{document} 