\section*{Розділ IV. Участь в органах студентського самоврядування}
\addcontentsline{toc}{section}{Розділ IV. Участь в органах студентського самоврядування}

\subsection*{4.1. Загальні положення}
\addcontentsline{toc}{subsection}{4.1. Загальні положення}
    4.1.1. Кожен студент Університету має право брати участь у студентському самоврядуванні на умовах, визначених Законом України ``Про вищу освіту'', Статутом Університету та цим Положенням.

    4.1.2. Участь студентів у студентському самоврядуванні є добровільною. Ніхто не може бути примушений до участі або неучасті в діяльності органів студентського самоврядування.

    4.1.3. Студенти мають право вільно обирати та бути обраними до складу ОСС відповідно до порядку, встановленого цим Положенням (активне та пасивне виборче право).

    4.1.4. Обмеження щодо участі в ОСС можуть бути встановлені виключно на підставах, передбачених законодавством України та цим Положенням.

    4.1.5. Усі члени ОСС мають рівні права та обов'язки, якщо інше не передбачено цим Положенням для окремих посад чи функцій.

    4.1.6. Поряд із принципом виборності представницьких органів та керівників ОСС, положеннями про окремі виконавчі органи студентського самоврядування (зокрема, Положенням про Студентську раду Київського авіаційного інституту щодо СРФ/СРІ) можуть встановлюватися особливості формування їхнього складу для забезпечення ефективної виконавчої діяльності, за умови збереження виборності керівника відповідного органу та/або підзвітності цього органу виборному органу вищого рівня чи безпосередньо студентам.

\subsection*{4.2. Вибори до ОСС}
\addcontentsline{toc}{subsection}{4.2. Вибори до ОСС}
    4.2.1. Вибори до органів студентського самоврядування організовуються та проводяться Центральною виборчою комісією студентів (ЦВКс).

    4.2.2. Порядок організації та проведення виборів до ОСС, вимоги до кандидатів, процедури висування, голосування, встановлення результатів та інші виборчі процедури визначаються Положенням про Центральну виборчу комісію студентів Університету.

    4.2.3. ЦВКс забезпечує прозорість, чесність та демократичність виборчого процесу.

\subsection*{4.3. Набуття членства та повноважень в ОСС}
\addcontentsline{toc}{subsection}{4.3. Набуття членства та повноважень в ОСС}
    \subsubsection*{4.3.1. Набуття повноважень виборними членами ОСС}
        4.3.1.1. Повноваження обраного члена ОСС починаються з моменту офіційного оголошення результатів виборів Центральною виборчою комісією студентів, якщо інше не встановлено рішенням ЦВКс або Конференцією студентів Університету (КСУ).

        4.3.1.2. Факт набуття повноважень може бути засвідчений відповідним рішенням органу студентського самоврядування або ЦВКс.

    \subsubsection*{4.3.2. Набуття повноважень призначеними членами ОСС}
        4.3.2.1. Особи, які входять до складу ОСС за посадою (ex officio) або призначаються до складу ОСС відповідно до процедур, визначених Статутом Університету, цим Положенням або положеннями про відповідні органи ОСС, набувають повноважень з моменту прийняття відповідного рішення уповноваженим органом або особою.

        4.3.2.2. Рішення про призначення має містити дату початку повноважень.

    \subsubsection*{4.3.3. Загальні процедури набуття статусу}
        4.3.3.1. Новообрані або призначені члени ОСС можуть складати присягу, текст та порядок складання якої визначається відповідним органом ОСС.

        4.3.3.2. Орган студентського самоврядування веде реєстр своїх членів. Відповідальність за ведення та актуалізацію реєстру покладається на Секретаря відповідного ОСС, якщо інше не передбачено положенням про цей орган або рішенням його керівника (Голови).

\subsection*{4.4. Припинення повноважень члена ОСС}
\addcontentsline{toc}{subsection}{4.4. Припинення повноважень члена ОСС}
    \subsubsection*{4.4.1. Підстави для дострокового припинення повноважень}
        4.4.1.1. Власне бажання (відставка з посади).

        4.4.1.2. Неможливість виконувати обов'язки за станом здоров'я, підтверджена медичним висновком.

        4.4.1.3. Набрання законної сили обвинувальним вироком суду щодо нього.

        4.4.1.4. Порушення Статуту Університету або цього Положення.

        4.4.1.5. Систематичне невиконання без поважних причин обов'язків, покладених на нього як на члена ОСС.

        4.4.1.6. Висловлення недовіри Конференцією студентів Університету (КСУ) (застосовується до виборних посад).

        4.4.1.7. Відкликання виборцями у порядку, встановленому відповідним положенням (застосовується до виборних посад).

        4.4.1.8. Прийняття рішення про дострокове припинення повноважень органом (або в порядку), що визначений положенням про відповідний ОСС для членів, які не є обраними безпосередньо студентами Університету або факультету/інституту;

        4.4.1.9. Рішення Студентської уповноваженої делегації (СУД) про припинення повноважень у випадках та порядку, передбачених Положенням про СУД.

    \subsubsection*{4.4.2. Процедура дострокового припинення повноважень}
        4.4.2.1. Припинення повноважень за власним бажанням (відставка) відбувається на підставі письмової заяви члена ОСС, поданої до керівника відповідного ОСС або до органу, який його обрав/призначив. Повноваження припиняються з дати, зазначеної в заяві, але не раніше дня її подання, або з моменту прийняття рішення про відставку уповноваженим органом, якщо така процедура передбачена.

        4.4.2.2. У випадках, передбачених підпунктами 4.4.1.2 – 4.4.1.5, 4.4.1.8 цього Положення, питання про дострокове припинення повноважень ініціюється керівником відповідного ОСС, групою членів ОСС, або органом, що обрав/призначив члена ОСС.

        4.4.2.3. Питання про дострокове припинення повноважень розглядається:

            \begin{enumerate}[label=\alph*)]
                \item Конференцією студентів Університету (КСУ) – стосовно виборних членів ОСС рівня Університету та голів студентських рад факультетів/інститутів/гуртожитків.
                \item Засіданням відповідного органу студентського самоврядування (студентської ради факультету/інституту/гуртожитку тощо) – стосовно призначених членів цього органу або інших виборних членів цього органу, якщо інше не передбачено положенням про цей орган.
                \item Органом, що призначив члена ОСС, – якщо це передбачено процедурою призначення або положенням про відповідний орган.
            \end{enumerate}

        4.4.2.4. Член ОСС, стосовно якого розглядається питання про припинення повноважень, має бути повідомлений про дату, час та місце засідання і має право бути присутнім та надавати пояснення (окрім випадків об'єктивної неможливості, наприклад, набрання чинності вироком суду).

        4.4.2.5. Рішення про дострокове припинення повноважень приймається шляхом голосування відповідно до процедури, встановленої цим Положенням або положенням про відповідний орган ОСС. Рішення вважається прийнятим, якщо за нього проголосувала більшість голосів від загального складу відповідного органу, якщо інша квота не встановлена положенням про цей орган.

        4.4.2.6. Процедури висловлення недовіри КСУ (пп. 4.4.1.6) та відкликання виборцями (пп. 4.4.1.7) визначаються окремими положеннями або Положенням про ЦВКс.

        4.4.2.7. Факт припинення повноважень фіксується у протоколі засідання відповідного органу та доводиться до відома члена ОСС, повноваження якого припинено.

        4.4.2.8. Студентська уповноважена делегація (СУД) може відсторонити члена ОСС від виконання повноважень або прийняти рішення про припинення його повноважень у випадках та порядку, передбачених Положенням про Студентську уповноважену делегацію (пп. 4.4.1.9).

        4.4.2.9. Припинення членства в ОСС з підстав, визначених у Розділі 4.5 цього Положення, автоматично припиняє повноваження члена ОСС без необхідності ухвалення окремого рішення згідно з процедурами цього Розділу.

    \subsubsection*{4.4.3. Припинення повноважень у зв'язку із закінченням строку}
        4.4.3.1. Повноваження члена ОСС припиняються автоматично в день закінчення строку, на який його було обрано або призначено.

        4.4.3.2. Окремого рішення про припинення повноважень у зв'язку із закінченням строку не потребується, якщо інше не встановлено положенням про відповідний ОСС.

\subsection*{4.5. Припинення членства в ОСС}
\addcontentsline{toc}{subsection}{4.5. Припинення членства в ОСС}
    \subsubsection*{4.5.1. Підстави для припинення членства}
        4.5.1.1. Завершення навчання або відрахування з Університету.

        4.5.1.2. Добровільний вихід зі складу ОСС (повне припинення участі).

        4.5.1.3. Втрата статусу здобувача вищої освіти з інших причин, передбачених законодавством або Статутом Університету.

        4.5.1.4. Смерть особи.

    \subsubsection*{4.5.2. Процедура припинення членства}
        4.5.2.1. Припинення членства в ОСС у зв'язку із завершенням навчання, відрахуванням з Університету, втратою статусу здобувача вищої освіти з інших причин або смертю відбувається автоматично з моменту настання відповідної події (наприклад, дата наказу про відрахування, дата смерті, зазначена у свідоцтві про смерть).

        4.5.2.2. Секретар відповідного ОСС на підставі офіційної інформації (наприклад, наказу про відрахування, даних відділу кадрів, копії свідоцтва про смерть тощо) фіксує факт припинення членства у реєстрі членів ОСС.

        4.5.2.3. Добровільний вихід зі складу ОСС здійснюється на підставі письмової заяви студента, поданої керівнику або секретарю відповідного ОСС. Членство припиняється з дати подання заяви, якщо в заяві не вказано іншу дату.

        4.5.2.4. Особа, членство якої в ОСС припинено, втрачає всі права та обов'язки, пов'язані зі статусом члена ОСС.

\subsection*{4.6. Процедура передачі повноважень}
\addcontentsline{toc}{subsection}{4.6. Процедура передачі повноважень}
    \subsubsection*{4.6.1. Загальні положення щодо передачі повноважень}
        4.6.1.1. Передача повноважень новообраному чи новопризначеному члену ОСС здійснюється в порядку, визначеному цим розділом, та має на меті забезпечення інституційної пам'яті та безперервності діяльності ОСС.
        
        4.6.1.2. Передача повноважень може бути плановою (при закінченні строку каденції) або позаплановою (при достроковому припиненні повноважень).
        
        4.6.1.3. Передача повноважень включає:

            \begin{enumerate}[label=\alph*)]
                \item Передачу документації, що стосується діяльності ОСС;
                \item Передачу доступів до електронних ресурсів, інформаційних систем та сервісів;
                \item Ознайомлення з поточними проєктами, планами та проблемами;
                \item Надання контактів ключових стейкхолдерів та партнерів;
                \item Передачу матеріальних цінностей (за наявності).
            \end{enumerate}

    \subsubsection*{4.6.2. Планова передача повноважень}
        4.6.2.1. Процес планової передачі повноважень розпочинається не пізніше, ніж за 10 (десять) календарних днів до завершення строку каденції. Протягом цього періоду створюються умови для поступової передачі повноважень.
        
        4.6.2.2. Планова передача повноважень керівників ОСС (голів, спікерів тощо) здійснюється в такому порядку:

            \begin{enumerate}[label=\alph*)]
                \item Попередній та новообраний керівники ОСС проводять щонайменше одну спільну зустріч для обговорення процесу передачі повноважень;
                \item На спільній зустрічі обговорюються стан справ, поточні проєкти, плани, проблеми та особливості керівництва відповідним ОСС;
                \item Попередній керівник передає перелік контактів ключових стейкхолдерів та партнерів та, за можливості, представляє новообраного керівника цим особам;
                \item Попередній керівник забезпечує доступ новообраного керівника до офіційної електронної пошти, хмарних сховищ, сайтів, сторінок у соціальних мережах та інших електронних ресурсів ОСС, а також до приміщень, техніки та обладнання;
                \item Секретар або інша уповноважена особа ОСС складає акт передачі матеріальних цінностей (за наявності), який підписується попереднім та новообраним керівниками.
            \end{enumerate}
            
        4.6.2.3. Усі члени органу ОСС, строк повноважень яких закінчується, зобов'язані передати всю наявну у них документацію та матеріали, що стосуються діяльності ОСС, новообраним/новопризначеним членам відповідно до розподілу функціональних обов'язків.
        
        4.6.2.4. За результатами передачі повноважень оформлюється протокол передачі справ, який підписується:

            \begin{enumerate}[label=\alph*)]
                \item Між головами/керівниками ОСС -- обома сторонами (попереднім та новообраним головою/керівником);
                \item У разі колегіального органу -- попереднім складом та новим складом або їхніми уповноваженими представниками.
            \end{enumerate}
            
        4.6.2.5. Протокол передачі справ має містити:

            \begin{enumerate}[label=\alph*)]
                \item Опис стану справ (завершені, поточні та заплановані проєкти);
                \item Опис переданої документації та матеріалів;
                \item Перелік переданих доступів до електронних ресурсів;
                \item Перелік переданих матеріальних цінностей (за наявності);
                \item Рекомендації щодо першочергових дій та невирішених питань.
            \end{enumerate}

    \subsubsection*{4.6.3. Позапланова передача повноважень}
        4.6.3.1. У разі дострокового припинення повноважень члена ОСС, передача повноважень здійснюється у скорочені терміни, але не пізніше 5 (п'яти) робочих днів з моменту прийняття рішення про припинення повноважень.
        
        4.6.3.2. Якщо член ОСС, повноваження якого припиняються достроково, відмовляється від участі у процедурі передачі повноважень або фізично не може взяти в ній участь, процедура передачі здійснюється за участю Секретаря відповідного ОСС та керівника ОСС, які складають відповідний акт.
        
        4.6.3.3. У невідкладних випадках, коли передача повноважень не може бути здійснена безпосередньо від особи, повноваження якої припиняються, до новообраної/новопризначеної особи, керівник відповідного ОСС (або його заступник) забезпечує збереження документації, доступів та матеріальних цінностей до моменту обрання/призначення нової особи.

    \subsubsection*{4.6.4. Особливості передачі повноважень у різних ОСС}
        4.6.4.1. Особливості процедури передачі повноважень в окремих ОСС можуть визначатися положеннями про відповідні ОСС з урахуванням специфіки їх діяльності.
        
        4.6.4.2. Для забезпечення ефективної передачі повноважень та інституційної пам'яті, ОСС можуть розробляти та вести документи, що містять докладний опис процедур, контактів, доступів та іншу важливу інформацію для подальшої діяльності (т.зв. ``перехідні документи'').
        
        4.6.4.3. СУД може розробляти рекомендації та шаблони документів для забезпечення уніфікованого підходу до процедури передачі повноважень у різних ОСС.
        
    \subsubsection*{4.6.5. Відповідальність за порушення процедури передачі повноважень}
        4.6.5.1. Умисне перешкоджання процедурі передачі повноважень, знищення або приховування документації чи іншої інформації, необхідної для діяльності ОСС, вважається грубим порушенням цього Положення.
        
        4.6.5.2. У разі виявлення порушень процедури передачі повноважень, СУД може провести розслідування та вжити заходів відповідно до своїх повноважень згідно з Положенням про СУД.

