\section*{Розділ I. Загальні положення}
\addcontentsline{toc}{section}{Розділ I. Загальні положення}

\subsection*{1.1. Визначення та мета студентського самоврядування}
\addcontentsline{toc}{subsection}{1.1. Визначення та мета студентського самоврядування}
    1.1.1. Студентське самоврядування -- це право і можливість студентів вирішувати питання навчання і побуту, захисту прав та інтересів студентів, а також брати участь в управлінні закладом вищої освіти.

    1.1.2. Метою студентського самоврядування в Університеті є створення умов для самореалізації студентів та їхньої участі в управлінні Університетом шляхом забезпечення захисту їхніх прав та інтересів, сприяння їхньому гармонійному розвитку.

    1.1.3. Поряд з органами студентського самоврядування (ОСС), як складовими офіційної системи, в Університеті можуть діяти студентські організації (СО) – добровільні об'єднання студентів, створені за спільними інтересами (культурними, науковими, спортивними, соціальними тощо) для реалізації статутних цілей цих організацій. СО діють незалежно від формальної структури та ієрархії ОСС.

\subsection*{1.2. Принципи діяльності ОСС}
\addcontentsline{toc}{subsection}{1.2. Принципи діяльності ОСС}
    1.2.1. Діяльність органів студентського самоврядування (ОСС) ґрунтується на принципах: законності, добровільності, колегіальності, виборності та звітності, рівноправності студентів у можливості брати участь у студентському самоврядуванні, прозорості та відкритості, організаційної самостійності в межах повноважень, визначених законодавством, Статутом Університету та цим Положенням.

\subsection*{1.3. Законодавчі та нормативні підстави}
\addcontentsline{toc}{subsection}{1.3. Законодавчі та нормативні підстави}
    1.3.1. Правову основу діяльності ОСС становлять Конституція України, Закон України ``Про вищу освіту'', Статут Університету, це Положення та інші нормативно-правові акти України, що стосуються питань студентського самоврядування.

    1.3.2. Це Положення визначає структуру, повноваження, порядок формування та основні засади діяльності органів студентського самоврядування Університету. Проєкт змін та доповнень до цього Положення підлягає попередньому оприлюдненню та громадському обговоренню серед студентів. Зміни та доповнення до цього Положення вносяться Конференцією студентів Університету більшістю не менше двох третин голосів від загального складу делегатів КСУ.

\subsection*{1.4. Основні завдання та сфери діяльності ОСС}
\addcontentsline{toc}{subsection}{1.4. Основні завдання та сфери діяльності ОСС}
    1.4.1. Забезпечення і захист прав та інтересів студентів, зокрема стосовно організації освітнього процесу.

    1.4.2. Сприяння навчальній, науковій та творчій діяльності студентів.

    1.4.3. Сприяння створенню належних умов для проживання і відпочинку студентів у гуртожитках та соціально-побутового забезпечення.

    1.4.4. Сприяння діяльності різноманітних студентських гуртків, товариств, об\'єднань, клубів за інтересами.

    1.4.5. Організація співробітництва зі студентами інших закладів вищої освіти та молодіжними організаціями в Україні та за її межами.

    1.4.6. Сприяння працевлаштуванню студентів та випускників.

    1.4.7. Участь у вирішенні питань міжнародної мобільності студентів.

    1.4.8. Забезпечення інформаційної підтримки студентів через офіційні ресурси ОСС та інші канали комунікації.

    1.4.9. Представлення інтересів студентської спільноти Університету у взаємовідносинах з адміністрацією Університету, іншими установами та організаціями.

    1.4.10. Сприяння розвитку студентських ініціатив та співпраця з визнаними (зареєстрованими) студентськими організаціями Університету.

\subsection*{1.5. Відповідальність членів ОСС}
\addcontentsline{toc}{subsection}{1.5. Відповідальність членів ОСС}
    1.5.1. Члени ОСС зобов'язані сумлінно виконувати свої обов'язки, діяти в інтересах студентської спільноти, дотримуватися вимог законодавства, Статуту Університету та цього Положення.

    1.5.2. Відповідальність членів ОСС перед студентською спільнотою реалізується через процедури звітності, оцінки діяльності, а також через механізми дострокового припинення повноважень, визначені Розділом IV цього Положення.

\subsection*{1.6. Право участі ОСС в управлінні Університетом}
\addcontentsline{toc}{subsection}{1.6. Право участі ОСС в управлінні Університетом}
    1.6.1. Органи студентського самоврядування мають гарантоване право брати участь в управлінні Університетом у порядку, встановленому Законом України ``Про вищу освіту'' та Статутом Університету.

    1.6.2. Основні форми участі ОСС в управлінні Університетом включають: участь в обговоренні та вирішенні питань удосконалення освітнього процесу, науково-дослідної роботи, призначення стипендій, організації дозвілля, оздоровлення, побуту та харчування студентів.

    1.6.3. Представники ОСС входять до складу Вченої ради Університету. Квоти представництва студентів у Вченій раді Університету визначаються Статутом Університету. Порядок висування кандидатів та обрання представників студентства до Вченої ради Університету встановлюється Положенням про Центральну виборчу комісію студентів (ЦВКс) та/або Регламентом КСУ. Представники ОСС відповідного рівня також входять до складу вчених рад факультетів/інститутів та можуть входити до інших робочих чи дорадчих органів Університету відповідно до квот та порядку, визначених Статутом Університету та положеннями про ці органи. Порядок обрання (делегування) представників студентів до вчених рад факультетів/інститутів визначається Положенням про Центральну виборчу комісію студентів (ЦВКс) з урахуванням вимог Статуту Університету та положень про відповідні факультети/інститути.

    1.6.4. Адміністрація Університету та її структурні підрозділи не мають права втручатися у діяльність ОСС, крім випадків, передбачених законодавством України. Рішення адміністрації, що стосуються прав та інтересів студентів, приймаються з урахуванням пропозицій відповідних ОСС у порядку, визначеному Розділом VII цього Положення.