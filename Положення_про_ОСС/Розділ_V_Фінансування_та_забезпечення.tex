\section*{Розділ V. Фінансування та матеріально-технічне забезпечення}
\addcontentsline{toc}{section}{Розділ V. Фінансування та матеріально-технічне забезпечення}

\subsection*{5.1. Джерела фінансування}
\addcontentsline{toc}{subsection}{5.1. Джерела фінансування}
    5.1.1. Фінансовою основою діяльності органів студентського самоврядування (ОСС) є кошти, отримані з таких джерел:

        \begin{enumerate}[label=\alph*)]
            \item Кошти, визначені Вченою радою Університету в розмірі не менш як 0,5 відсотка власних надходжень, отриманих Університетом від основної діяльності;
            \item Членські внески студентів, розмір яких встановлюється Конференцією студентів Університету (КСУ), але не може перевищувати 1 відсотка прожиткового мінімуму, встановленого законодавством, на одну особу на місяць (рішення про запровадження та розмір внесків приймається КСУ);
            \item Благодійна допомога, гранти, дарунки, спонсорські кошти, отримані від фізичних та юридичних осіб відповідно до законодавства України;
            \item Інші надходження, не заборонені законодавством України.
        \end{enumerate}

\subsection*{5.2. Порядок використання коштів}
\addcontentsline{toc}{subsection}{5.2. Порядок використання коштів}
    5.2.1. Кошти ОСС обліковуються та зберігаються на окремому субрахунку Університету.

    5.2.2. Використання коштів ОСС здійснюється відповідно до єдиного кошторису (бюджету) ОСС, затвердженого Конференцією студентів Університету (КСУ), на виконання завдань та напрямів діяльності, передбачених цим Положенням та статутними документами відповідних ОСС.

    5.2.3. Координацію фінансової діяльності, збір та перевірку проєктів кошторисів ОСС факультетів/інститутів та СР КАІ здійснює Фінансовий комітет СР КАІ. Аналогічні функції щодо ОСС гуртожитків та СР СМ виконує Фінансовий комітет СР СМ. Порядок їхньої діяльності визначається положеннями про СР КАІ та СР СМ. Контроль за цільовим використанням коштів усіх ОСС здійснюється Студентською уповноваженою делегацією (СУД).

    5.2.4. ОСС забезпечують прозорість використання фінансових ресурсів шляхом оприлюднення кошторисів та звітів про їх виконання у порядку, визначеному Розділом VI цього Положення.

\subsection*{5.3. Матеріально-технічне забезпечення}
\addcontentsline{toc}{subsection}{5.3. Матеріально-технічне забезпечення}
    5.3.1. Адміністрація Університету надає органам студентського самоврядування необхідні для їхньої діяльності приміщення з відповідним обладнанням та інвентарем на безоплатній основі.

    5.3.2. Адміністрація Університету забезпечує ОСС доступом до мережі Інтернет, телефонного зв'язку, можливості користування оргтехнікою та іншими необхідними ресурсами на умовах, що визначаються угодою між ОСС та адміністрацією або окремими рішеннями.

    5.3.3. ОСС дбайливо ставляться до наданого Університетом майна. ОСС також мають право володіти та користуватися майном, набутим за власні кошти або отриманим як благодійна допомога, для здійснення своїх статутних завдань.

\subsection*{5.4. Грантове фінансування та спонсорство}
\addcontentsline{toc}{subsection}{5.4. Грантове фінансування та спонсорство}
    5.4.1. ОСС мають право самостійно або спільно з іншими організаціями брати участь у конкурсах на отримання грантів від українських та міжнародних фондів і організацій.

    5.4.2. ОСС можуть отримувати благодійну допомогу та спонсорські кошти від фізичних та юридичних осіб відповідно до мети своєї діяльності та вимог законодавства України.

    5.4.3. Право на укладання угод про отримання благодійної допомоги чи спонсорських коштів та управління такими коштами належить виключно Студентській раді Київського авіаційного інституту (СР КАІ) та Студентській раді студмістечка (СР СМ) через їхні уповноважені комітети або посадових осіб. Органи студентського самоврядування факультетів/інститутів та гуртожитків не мають права самостійно залучати та розпоряджатися такими коштами.

    5.4.4. Облік та використання грантових коштів, благодійної допомоги та спонсорських надходжень здійснюються відповідно до умов їх надання, вимог законодавства України та внутрішніх положень ОСС, забезпечуючи прозорість та цільове призначення.

\subsection*{5.5. Фінансова взаємодія ОСС зі студентськими організаціями}
\addcontentsline{toc}{subsection}{5.5. Фінансова взаємодія ОСС зі студентськими організаціями}
    5.5.1. Визнані (зареєстровані) студентські організації (СО) мають право подавати проєкти на конкурсній основі для отримання цільового фінансування з коштів єдиного бюджету ОСС на реалізацію конкретних заходів, що відповідають цілям та завданням студентського самоврядування, визначеним у Розділі I цього Положення.

    5.5.2. Порядок проведення конкурсу проєктів СО, критерії відбору, обсяги фінансування та порядок звітності визначаються відповідним органом ОСС (СР КАІ, СР СМ) та/або Положенням про студентські організації Університету.

    5.5.3. Надання коштів з бюджету ОСС для покриття операційних витрат (поточна діяльність, утримання тощо) студентських організацій не передбачається. СО діють на засадах самофінансування або залучення зовнішніх ресурсів.