\section*{Розділ VIII. Прикінцеві та перехідні положення}

\subsection*{8.1. Порядок внесення змін та доповнень}
    8.1.1. Зміни та доповнення до цього Положення вносяться у порядку, визначеному пунктом 1.3.2 цього Положення.

    8.1.2. Право ініціювати внесення змін та доповнень до цього Положення мають Конференція студентів Університету (КСУ), Студентська рада Київського авіаційного інституту (СР КАІ), Студентська рада студмістечка (СР СМ) та Студентська уповноважена делегація (СУД).

\subsection*{8.2. Набуття чинності}
    8.2.1. Це Положення набуває чинності з дати його офіційного оприлюднення на інформаційних ресурсах ОСС Університету.

    8.2.2. З моменту набуття чинності цим Положенням, Положення ``Про Тимчасовий перехідний комітет'' від 09.12.2024 та інші нормативні акти ОСС, що суперечать цьому Положенню, втрачають чинність.

\subsection*{8.3. Тлумачення Положення}
    8.3.1. Офіційне тлумачення норм цього Положення надається Студентською уповноваженою делегацією (СУД). Рішення СУД щодо тлумачення є обов'язковим для всіх ОСС.

\subsection*{8.4. Перехідні положення}
    8.4.1. Органи студентського самоврядування (включаючи Тимчасовий перехідний комітет), сформовані до набуття чинності цим Положенням, припиняють свої повноваження з моменту набуття чинності цим Положенням.

    8.4.2. Перші вибори органів студентського самоврядування відповідно до цього Положення (включаючи формування першого складу ЦВКс та скликання першої КСУ) мають бути проведені протягом одного місяця з дня набуття чинності цим Положенням у порядку, затвердженому Загальними зборами студентів Університету.

    8.4.3. Конференція студентів Університету на своєму першому засіданні після набуття чинності цим Положенням має затвердити Положення про Студентську раду Київського авіаційного інституту, Положення про Студентську раду студмістечка, Положення про Студентську уповноважену делегацію та Положення про Центральну виборчу комісію студентів протягом трьох місяців.