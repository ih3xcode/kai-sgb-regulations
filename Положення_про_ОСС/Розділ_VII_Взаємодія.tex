\section*{Розділ VII. Взаємодія з іншими структурами}

\subsection*{7.1. Загальні засади взаємодії}
    7.1.1. Органи студентського самоврядування (ОСС) взаємодіють з адміністрацією Університету, іншими структурними підрозділами Університету, студентськими та молодіжними організаціями в Україні та за кордоном з метою ефективного виконання своїх завдань, представництва та захисту прав і інтересів студентів, спільного розвитку студентського середовища, залучення додаткових ресурсів та формування позитивного іміджу Університету.

    7.1.2. Взаємодія ОСС з іншими структурами здійснюється на принципах партнерства, взаємної поваги, конструктивного діалогу, законності, прозорості та відповідальності.

\subsection*{7.2. Взаємодія з адміністрацією Університету}
    7.2.1. Взаємодія ОСС з адміністрацією Університету здійснюється шляхом: проведення регулярних зустрічей та консультацій; участі представників ОСС у засіданнях робочих та дорадчих органів адміністрації; створення спільних робочих груп для вирішення конкретних питань; взаємного обміну інформацією та документами; узгодження планів та заходів, що стосуються студентського життя.

    7.2.2. ОСС мають право звертатися до адміністрації Університету та її структурних підрозділів з пропозиціями, запитами, заявами з питань, що належать до їхньої компетенції. Порядок подання таких звернень визначається загальним порядком діловодства Університету з урахуванням вимог, встановлених регламентами відповідних ОСС.

    7.2.3. Адміністрація Університету розглядає звернення та пропозиції ОСС у встановлені законодавством або Статутом Університету строки. Рішення адміністрації, що стосуються прав та інтересів студентів, приймаються після обов'язкового розгляду пропозицій відповідних ОСС. Представники ОСС мають право бути запрошеними на засідання відповідного органу адміністрації під час розгляду їхніх пропозицій та надавати пояснення. У разі неврахування або часткового врахування пропозицій ОСС, відповідний орган адміністрації надає ОСС письмове обґрунтоване пояснення причин такого рішення у 10-денний робочий строк. Детальний порядок погодження окремих рішень (наприклад, щодо відрахування, поселення/виселення з гуртожитку) може визначатися відповідними положеннями Університету та положеннями про СР КАІ та СР СМ.

    7.2.4. Спірні питання, що виникають у процесі взаємодії між ОСС та адміністрацією Університету, вирішуються шляхом переговорів, консультацій, створення узгоджувальних комісій, які формуються на паритетних засадах представниками ОСС та адміністрації, порядок роботи яких визначається за згодою сторін, або в іншому порядку, передбаченому Статутом Університету та законодавством.

    7.2.5. Рішення адміністрації з наступних питань \textbf{не можуть бути прийняті без попереднього погодження} з відповідними органами студентського самоврядування:
        \begin{enumerate}[label=\arabic*)]
            \item Відрахування студентів з Університету та їх поновлення на навчання -- погоджується із СР КАІ та Студентською радою відповідного факультету/інституту;
            \item Переведення осіб, які навчаються в Університеті за державним замовленням, на навчання за контрактом за рахунок коштів фізичних (юридичних) осіб -- погоджується із СР КАІ та Студентською радою відповідного факультету/інституту;
            \item Переведення осіб, які навчаються в Університеті за рахунок коштів фізичних (юридичних) осіб, на навчання за державним замовленням -- погоджується із СР КАІ та Студентською радою відповідного факультету/інституту;
            \item Призначення заступника декана факультету/директора інституту з виховної роботи (або аналогічної посади), заступника керівника (проректора) Університету, до сфери відповідальності якого належать питання студентського життя -- погоджується із Головою Студентської ради відповідного факультету/інституту та Головою СР КАІ відповідно;
            \item Поселення осіб, які навчаються в Університеті, у гуртожиток і виселення їх із гуртожитку -- погоджується із Головою СР СМ та Головою Студентської ради відповідного гуртожитку;
            \item Затвердження правил внутрішнього розпорядку Університету в частині, що стосується осіб, які навчаються в Університеті -- погоджується із Головою СР КАІ;
            \item Діяльність студентських містечок та гуртожитків для проживання осіб, які навчаються у закладі вищої освіти (включаючи затвердження правил проживання, розподіл місць, встановлення розміру плати за проживання) -- погоджується із Головою СР СМ.
        \end{enumerate}

\subsection*{7.3. Взаємодія зі студентськими та молодіжними організаціями}
    7.3.1. ОСС розвивають співпрацю з міжнародними, всеукраїнськими та регіональними студентськими та молодіжними організаціями, асоціаціями з метою обміну досвідом, реалізації спільних проєктів, представлення інтересів студентства Університету.

    7.3.2. ОСС підтримують молодіжні ініціативи, спрямовані на розвиток студентства, та можуть брати участь у реалізації соціальних проєктів у партнерстві з іншими організаціями.

    7.3.3. Взаємодія ОСС із визнаними (зареєстрованими) студентськими організаціями (СО) Університету відбувається на засадах партнерства та координації діяльності і може включати: спільну організацію заходів; обмін інформацією щодо планів та результатів діяльності; створення спільних робочих груп для вирішення конкретних питань; запрошення представників СО на засідання ОСС з правом дорадчого голосу при розгляді питань, що безпосередньо стосуються діяльності відповідної СО. Можливість фінансової підтримки проєктів СО з боку ОСС визначається Розділом V цього Положення.

\subsection*{7.4. Участь у програмах та проєктах}
    7.4.1. ОСС беруть участь у розробці та реалізації програм Університету, спрямованих на покращення умов навчання, побуту, дозвілля, розвитку студентів.

    7.4.2. ОСС можуть брати участь у національних та міжнародних програмах і проєктах, що стосуються студентського самоврядування, мобільності, молодіжної політики та інших актуальних для студентства питань.