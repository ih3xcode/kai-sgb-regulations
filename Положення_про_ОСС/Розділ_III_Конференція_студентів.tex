\section*{Розділ III. Конференція студентів Університету}
\addcontentsline{toc}{section}{Розділ III. Конференція студентів Університету}

\subsection*{3.1. Статус та загальні засади діяльності КСУ}
\addcontentsline{toc}{subsection}{3.1. Статус та загальні засади діяльності КСУ}
    3.1.1. Конференція студентів Університету (КСУ) є вищим представницьким органом студентського самоврядування Університету.

    3.1.2. Чергова КСУ скликається Спікером КСУ не рідше одного разу на рік. Позачергова КСУ може бути скликана за ініціативою Студентської ради Київського авіаційного інституту (СР КАІ), Студентської ради студмістечка (СР СМ), Студентської уповноваженої делегації (СУД), Центральної виборчої комісії студентів (ЦВКс), Спікера КСУ, не менше третини від загального складу делегатів КСУ або за ініціативою не менше 5\% студентів від загальної кількості студентів Університету.

    3.1.3. Діяльність КСУ здійснюється на принципах колегіальності, виборності делегатів, рівності прав делегатів, гласності та відкритості.

    3.1.4. За рішенням КСУ, прийнятим більшістю голосів від загального складу делегатів, окремі питання можуть розглядатися на закритому засіданні. Закриті засідання можуть проводитися при розгляді персональних питань, питань, що містять конфіденційну інформацію, або в інших випадках, коли цього вимагає захист прав та інтересів осіб чи забезпечення ефективного обговорення. Рішення про проведення закритого засідання та його результати фіксуються в протоколі КСУ.

    3.1.5. Представники адміністрації Університету мають право брати участь у засіданнях КСУ з правом дорадчого голосу.

\subsection*{3.2. Повноваження КСУ}
\addcontentsline{toc}{subsection}{3.2. Повноваження КСУ}
    До виключних повноважень КСУ належить:
        \begin{enumerate}[label=\alph*)]
            \item Затвердження Положення про студентське самоврядування Університету та внесення змін до нього;
            \item Затвердження положень, що регламентують створення та діяльність органів студентського самоврядування та інших структур, передбачених цим Положенням (включаючи, але не обмежуючись, положеннями про СР КАІ, СР СМ, СУД, ЦВКс, студентські організації), та внесення змін до них;
            \item Заслуховування та затвердження звітів СР КАІ, СР СМ, СУД, ЦВКс та інших органів, підзвітних КСУ;
            \item Затвердження результатів виборів Голови СР КАІ та Голови СР СМ, проведених у встановленому порядку;
            \item Обрання та припинення повноважень членів СУД та ЦВКс;
            \item Визначення структури та стратегічних напрямів діяльності ОСС Університету;
            \item Затвердження єдиного кошторису (бюджету) ОСС Університету та звіту про його виконання. У разі, якщо проєкт кошторису не був погоджений Студентською уповноваженою делегацією (СУД), КСУ може затвердити його лише за умови, якщо за таке рішення проголосувало не менше двох третин (2/3) від загального складу делегатів КСУ;
            \item Розгляд питання про висловлення недовіри керівникам та членам СР КАІ, СР СМ, СУД, ЦВКс, а також іншим членам ОСС. Питання про висловлення недовіри за підставами, розгляд яких належить до компетенції СУД згідно з Положенням про СУД, розглядається КСУ, як правило, за поданням СУД або в порядку апеляції на рішення СУД. В інших випадках питання ініціюється делегатом КСУ або СУД. Рішення про висловлення недовіри приймається не менше ніж двома третинами голосів від загального складу делегатів КСУ та має наслідком автоматичне припинення повноважень відповідної особи;
            \item Призначення виконувачів обов'язків (в.о.) посадових осіб ОСС на термін більше 1 (одного) місяця;
            \item Затвердження або відхилення призначення виконувачів обов'язків (в.о.) посадових осіб ОСС, призначених Студентською уповноваженою делегацією (СУД) у випадках, передбачених Положенням про СУД. Рішення про затвердження приймається простою більшістю голосів від загального складу делегатів КСУ;
            \item Визначення розміру членських внесків студентів (якщо вони запроваджуються);
            \item Вирішення інших питань, віднесених до компетенції КСУ цим Положенням, Статутом Університету або законодавством України.
        \end{enumerate}

\subsection*{3.3. Порядок формування делегатів КСУ}
\addcontentsline{toc}{subsection}{3.3. Порядок формування делегатів КСУ}
    3.3.1. Делегати КСУ обираються (делегуються) від студентів факультетів, інститутів та гуртожитків на основі пропорційного представництва та рівних можливостей.

    3.3.2. Квота представництва для обрання делегатів КСУ встановлюється Центральною виборчою комісією студентів (ЦВКс) перед кожними виборами і становить: 1 (один) делегат від 500 студентів факультету/інституту; 0,5\% від кількості мешканців гуртожитку, але не більше 5 (п'яти) делегатів від одного гуртожитку.

    3.3.3. Порядок обрання (делегування) представників на КСУ від студентів факультетів, інститутів, гуртожитків визначається Положенням про Центральну виборчу комісію студентів (ЦВКс).

    3.3.4. Строк повноважень делегатів КСУ становить 1 (один) рік.

    3.3.5. Голова Студентської ради Київського авіаційного інституту (СР КАІ) та Голова Студентської ради студмістечка (СР СМ) є членами КСУ за посадою (ex officio). Вони мають право голосу на засіданнях КСУ лише у випадку, якщо вони були обрані на свої посади у встановленому порядку відповідно до положень про СР КАІ та СР СМ, і не є тимчасово виконуючими обов'язки. Члени КСУ за посадою, які мають право голосу, враховуються при визначенні кворуму та підрахунку голосів як частина загального складу КСУ.

\subsection*{3.4. Порядок скликання та проведення КСУ}
\addcontentsline{toc}{subsection}{3.4. Порядок скликання та проведення КСУ}
    3.4.1. Порядок скликання КСУ визначено у пункті 3.1.2 цього Положення.

    3.4.2. Повідомлення про дату, час, місце проведення та проєкт порядку денного КСУ оприлюднюється Спікером КСУ (або органом, що скликає позачергову КСУ, окрім СУД у невідкладних випадках) на офіційних інформаційних ресурсах ОСС не пізніше ніж за 10 (десять) календарних днів до її проведення. У невідкладних випадках, що потребують термінового розгляду, Студентська уповноважена делегація (СУД) має право скликати позачергову КСУ, повідомивши про це не пізніше ніж за 3 (три) календарні дні до дати проведення, з обов'язковим обґрунтуванням терміновості.

    3.4.3. КСУ є правомочною, якщо на її засіданні зареєстровано більше половини від загальної кількості обраних (затверджених) делегатів.

    3.4.4. На першому засіданні новообраної КСУ зі свого складу обираються Спікер КСУ та Секретар КСУ, які діють на постійній основі протягом усього строку повноважень КСУ даного скликання. Рішення про обрання приймається більшістю голосів від загального складу делегатів КСУ.

    3.4.5. Рішення КСУ приймаються відкритим поіменним голосуванням більшістю голосів від загального складу делегатів КСУ, якщо інша процедура або інша кількість голосів не передбачена цим Положенням або Регламентом КСУ (зокрема, для питань, що потребують 2/3 голосів).

    3.4.6. Хід засідання КСУ протоколюється Секретарем КСУ. Рішення КСУ оформлюються протоколом, який підписується Спікером КСУ та Секретарем КСУ, та підлягають оприлюдненню на офіційних інформаційних ресурсах ОСС протягом 3 (трьох) робочих днів з дня проведення засідання.

    3.4.7. Контроль за виконанням рішень КСУ покладається на Спікера КСУ та на органи студентського самоврядування, до компетенції яких належить виконання відповідних рішень.

\subsection*{3.5. Робочі органи КСУ}
\addcontentsline{toc}{subsection}{3.5. Робочі органи КСУ}
    3.5.1. КСУ має право створювати зі складу своїх делегатів постійні та тимчасові комісії для попереднього розгляду та підготовки питань, що належать до її компетенції.

    3.5.2. Перелік постійних комісій, порядок їх формування, повноваження та організація роботи визначаються Регламентом КСУ або окремими рішеннями КСУ.