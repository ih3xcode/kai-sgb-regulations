\section*{Розділ II. Структура органів студентського самоврядування}
\addcontentsline{toc}{section}{Розділ II. Структура органів студентського самоврядування}

\subsection*{2.1. Система та рівні ОСС}
\addcontentsline{toc}{subsection}{2.1. Система та рівні ОСС}
    2.1.1. Студентське самоврядування в Університеті здійснюється через єдину систему органів студентського самоврядування (ОСС), що діє на принципах законності, добровільності, колегіальності, виборності, підзвітності, прозорості та незалежності від адміністрації Університету, з метою забезпечення виконання завдань студентського самоврядування та захисту прав та інтересів студентів.

    2.1.2. Система ОСС має таку структуру рівнів:
    
        \begin{enumerate}[label=\alph*)]
            \item Загальноуніверситетський рівень;
            \item Локальний рівень (органи студентського самоврядування факультетів/інститутів та гуртожитків).
        \end{enumerate}

    2.1.3. Органи студентського самоврядування різних рівнів діють на засадах єдності мети, координації діяльності, субсидіарності (прийняття рішень на найнижчому можливому ефективному рівні), розмежування та делегування повноважень, а також ієрархічної підпорядкованості ОСС нижчого рівня ОСС вищого рівня в межах їхньої компетенції.

\subsection*{2.2. Вищий представницький орган ОСС}
\addcontentsline{toc}{subsection}{2.2. Вищий представницький орган ОСС}
    2.2.1. Вищим органом студентського самоврядування в Університеті є Конференція студентів Університету (КСУ).

    2.2.2. Статус, повноваження, порядок формування делегатів, скликання та процедури прийняття рішень КСУ визначаються Розділом ІІІ цього Положення.

\subsection*{2.3. Вищий виконавчий орган ОСС}
\addcontentsline{toc}{subsection}{2.3. Вищий виконавчий орган ОСС}
    2.3.1. Вищим виконавчим органом студентського самоврядування Університету є Студентська рада Київського авіаційного інституту (СР КАІ).

    2.3.2. Основні повноваження, завдання та напрями діяльності СР КАІ визначаються Положенням про Студентську раду Київського авіаційного інституту, яке затверджується КСУ.

    2.3.3. Порядок формування, структура та порядок роботи СР КАІ визначаються Положенням про Студентську раду Київського авіаційного інституту.

\subsection*{2.4. Інші загальноуніверситетські ОСС}
\addcontentsline{toc}{subsection}{2.4. Інші загальноуніверситетські ОСС}
    2.4.1. Центральна виборча комісія студентів (ЦВКс) є постійно діючим органом, відповідальним за організацію та проведення виборів до ОСС. Статус та повноваження ЦВКс визначаються Положенням про ЦВКс, яке затверджується КСУ.

    2.4.2. Студентська уповноважена делегація (СУД) є постійно діючим органом, відповідальним за контрольну діяльність в системі ОСС. Статус та повноваження СУД визначаються Положенням про СУД, яке затверджується КСУ.

    2.4.3. Студентська рада студмістечка (СР СМ) є органом, що координує діяльність ОСС гуртожитків та представляє інтереси мешканців студмістечка на загальноуніверситетському рівні. Статус та повноваження СР СМ визначаються окремим Положенням про Студентську раду студмістечка, яке затверджується КСУ.

\subsection*{2.5. Органи студентського самоврядування факультетів/інститутів}
\addcontentsline{toc}{subsection}{2.5. Органи студентського самоврядування факультетів/інститутів}
    2.5.1. Основним органом студентського самоврядування на рівні факультету/інституту є Студентська рада факультету/інституту.

    2.5.2. Повноваження, порядок формування та діяльність ОСС факультету/інституту визначаються Положенням про Студентську раду Київського авіаційного інституту, яке затверджується КСУ, та не повинні суперечити цьому Положенню.

\subsection*{2.6. Органи студентського самоврядування гуртожитків}
\addcontentsline{toc}{subsection}{2.6. Органи студентського самоврядування гуртожитків}
    2.6.1. Основним органом студентського самоврядування на рівні гуртожитку є Студентська рада гуртожитку.

    2.6.2. Повноваження, порядок формування та діяльність ОСС гуртожитку визначаються Положенням про Студентську раду студмістечка, яке затверджується КСУ, та не повинні суперечити цьому Положенню.

\subsection*{2.7. Взаємодія між ОСС}
\addcontentsline{toc}{subsection}{2.7. Взаємодія між ОСС}
    2.7.1. Взаємодія між ОСС різних рівнів та одного рівня будується на принципах координації, співпраці, взаємної поваги, підзвітності відповідно до ієрархії, визначеної цим Положенням, та розмежування повноважень. Рішення ОСС вищого рівня є обов'язковими для ОСС нижчого рівня в межах їх компетенції.

    2.7.2. Механізмами взаємодії ОСС є, зокрема: проведення спільних засідань керівних органів; створення спільних робочих груп та комісій за напрямками діяльності; регулярний обмін інформацією, планами та звітами; взаємне делегування представників до складу ОСС іншого рівня або органу; проведення координаційних нарад голів ОСС.

    2.7.3. Спори між ОСС різних рівнів або одного рівня щодо компетенції, виконання рішень або інших питань їхньої діяльності вирішуються шляхом переговорів. У разі недосягнення згоди, спір може бути передано на розгляд Студентської уповноваженої делегації (СУД) для надання висновку або медіації. Порядок розгляду спорів СУД визначається Положенням про СУД.

\subsection*{2.8. Студентські організації (СО)}
\addcontentsline{toc}{subsection}{2.8. Студентські організації (СО)}
    2.8.1. Студентські організації (СО) є добровільними об'єднаннями студентів за інтересами, що діють незалежно від системи та ієрархії органів студентського самоврядування (ОСС).

    2.8.2. Порядок створення, реєстрації (визнання), діяльності та припинення діяльності СО, їх права та обов'язки визначаються окремим Положенням про студентські організації Університету, яке затверджується Конференцією студентів Університету (КСУ).

    2.8.3. Органи студентського самоврядування взаємодіють зі студентськими організаціями, що визнані (зареєстровані) у порядку, встановленому Положенням про студентські організації Університету.