\section*{Розділ IV. Участь в органах студентського самоврядування}

\subsection*{4.1. Загальні положення}
    4.1.1. Кожен студент Університету має право брати участь у студентському самоврядуванні на умовах, визначених Законом України ``Про вищу освіту'', Статутом Університету та цим Положенням.

    4.1.2. Участь студентів у студентському самоврядуванні є добровільною. Ніхто не може бути примушений до участі або неучасті в діяльності органів студентського самоврядування.

    4.1.3. Студенти мають право вільно обирати та бути обраними до складу ОСС відповідно до порядку, встановленого цим Положенням (активне та пасивне виборче право).

    4.1.4. Обмеження щодо участі в ОСС можуть бути встановлені виключно на підставах, передбачених законодавством України та цим Положенням.

    4.1.5. Усі члени ОСС мають рівні права та обов'язки, якщо інше не передбачено цим Положенням для окремих посад чи функцій.

\subsection*{4.2. Вибори до ОСС}
    4.2.1. Вибори до органів студентського самоврядування організовуються та проводяться Центральною виборчою комісією студентів (ЦВКс).

    4.2.2. Порядок організації та проведення виборів до ОСС, вимоги до кандидатів, процедури висування, голосування, встановлення результатів та інші виборчі процедури визначаються Положенням про Центральну виборчу комісію студентів Університету.

    4.2.3. ЦВКс забезпечує прозорість, чесність та демократичність виборчого процесу.

\subsection*{4.3. Набуття членства та повноважень в ОСС}
    \subsubsection*{4.3.1. Набуття повноважень виборними членами ОСС}
        4.3.1.1. Повноваження обраного члена ОСС починаються з моменту офіційного оголошення результатів виборів Центральною виборчою комісією студентів, якщо інше не встановлено рішенням ЦВКс або Конференцією студентів Університету (КСУ).

        4.3.1.2. Факт набуття повноважень може бути засвідчений відповідним рішенням органу студентського самоврядування або ЦВКс.
    \subsubsection*{4.3.2. Набуття повноважень призначеними членами ОСС}
        4.3.2.1. Особи, які входять до складу ОСС за посадою (ex officio) або призначаються до складу ОСС відповідно до процедур, визначених Статутом Університету, цим Положенням або положеннями про відповідні органи ОСС, набувають повноважень з моменту прийняття відповідного рішення уповноваженим органом або особою.

        4.3.2.2. Рішення про призначення має містити дату початку повноважень.
    \subsubsection*{4.3.3. Загальні процедури набуття статусу}
        4.3.3.1. Новообрані або призначені члени ОСС можуть складати присягу, текст та порядок складання якої визначається відповідним органом ОСС.

        4.3.3.2. Орган студентського самоврядування веде реєстр своїх членів. Відповідальність за ведення та актуалізацію реєстру покладається на Секретаря відповідного ОСС, якщо інше не передбачено положенням про цей орган або рішенням його керівника (Голови).

\subsection*{4.4. Припинення повноважень члена ОСС}
    \subsubsection*{4.4.1. Підстави для дострокового припинення повноважень}
        4.4.1.1. Власне бажання (відставка з посади).

        4.4.1.2. Неможливість виконувати обов'язки за станом здоров'я, підтверджена медичним висновком.

        4.4.1.3. Набрання законної сили обвинувальним вироком суду щодо нього.

        4.4.1.4. Порушення Статуту Університету або цього Положення.

        4.4.1.5. Систематичне невиконання без поважних причин обов'язків, покладених на нього як на члена ОСС.

        4.4.1.6. Висловлення недовіри Конференцією студентів Університету (КСУ) (застосовується до виборних посад).

        4.4.1.7. Відкликання виборцями у порядку, встановленому відповідним положенням (застосовується до виборних посад).

        4.4.1.8. Прийняття рішення про звільнення органом, що призначив (застосовується до призначених посад).

        4.4.1.9. Рішення Студентської уповноваженої делегації (СУД) про припинення повноважень у випадках та порядку, передбачених Положенням про СУД.
    \subsubsection*{4.4.2. Процедура дострокового припинення повноважень}
        4.4.2.1. Припинення повноважень за власним бажанням (відставка) відбувається на підставі письмової заяви члена ОСС, поданої до керівника відповідного ОСС або до органу, який його обрав/призначив. Повноваження припиняються з дати, зазначеної в заяві, але не раніше дня її подання, або з моменту прийняття рішення про відставку уповноваженим органом, якщо така процедура передбачена.

        4.4.2.2. У випадках, передбачених підпунктами 4.4.1.2 – 4.4.1.5, 4.4.1.8 цього Положення, питання про дострокове припинення повноважень ініціюється керівником відповідного ОСС, групою членів ОСС, або органом, що обрав/призначив члена ОСС.

        4.4.2.3. Питання про дострокове припинення повноважень розглядається:
            \begin{enumerate}[label=\alph*)]
                \item Конференцією студентів Університету (КСУ) – стосовно виборних членів ОСС рівня Університету та голів студентських рад факультетів/інститутів/гуртожитків.
                \item Засіданням відповідного органу студентського самоврядування (студентської ради факультету/інституту/гуртожитку тощо) – стосовно призначених членів цього органу або інших виборних членів цього органу, якщо інше не передбачено положенням про цей орган.
                \item Органом, що призначив члена ОСС, – якщо це передбачено процедурою призначення або положенням про відповідний орган.
            \end{enumerate}

        4.4.2.4. Член ОСС, стосовно якого розглядається питання про припинення повноважень, має бути повідомлений про дату, час та місце засідання і має право бути присутнім та надавати пояснення (окрім випадків об'єктивної неможливості, наприклад, набрання чинності вироком суду).

        4.4.2.5. Рішення про дострокове припинення повноважень приймається шляхом голосування відповідно до процедури, встановленої цим Положенням або положенням про відповідний орган ОСС. Рішення вважається прийнятим, якщо за нього проголосувала більшість голосів від загального складу відповідного органу, якщо інша квота не встановлена положенням про цей орган.

        4.4.2.6. Процедури висловлення недовіри КСУ (пп. 4.4.1.6) та відкликання виборцями (пп. 4.4.1.7) визначаються окремими положеннями або Положенням про ЦВКс.

        4.4.2.7. Факт припинення повноважень фіксується у протоколі засідання відповідного органу та доводиться до відома члена ОСС, повноваження якого припинено.

        4.4.2.8. Студентська уповноважена делегація (СУД) може відсторонити члена ОСС від виконання повноважень або прийняти рішення про припинення його повноважень у випадках та порядку, передбачених Положенням про Студентську уповноважену делегацію (пп. 4.4.1.9).

        4.4.2.9. Припинення членства в ОСС з підстав, визначених у Розділі 4.5 цього Положення, автоматично припиняє повноваження члена ОСС без необхідності ухвалення окремого рішення згідно з процедурами цього Розділу.
    \subsubsection*{4.4.3. Припинення повноважень у зв'язку із закінченням строку}
        4.4.3.1. Повноваження члена ОСС припиняються автоматично в день закінчення строку, на який його було обрано або призначено.

        4.4.3.2. Окремого рішення про припинення повноважень у зв'язку із закінченням строку не потребується, якщо інше не встановлено положенням про відповідний ОСС.

\subsection*{4.5. Припинення членства в ОСС}
    \subsubsection*{4.5.1. Підстави для припинення членства}
        4.5.1.1. Завершення навчання або відрахування з Університету.

        4.5.1.2. Добровільний вихід зі складу ОСС (повне припинення участі).

        4.5.1.3. Втрата статусу здобувача вищої освіти з інших причин, передбачених законодавством або Статутом Університету.

        4.5.1.4. Смерть особи.
    \subsubsection*{4.5.2. Процедура припинення членства}
        4.5.2.1. Припинення членства в ОСС у зв'язку із завершенням навчання, відрахуванням з Університету, втратою статусу здобувача вищої освіти з інших причин або смертю відбувається автоматично з моменту настання відповідної події (наприклад, дата наказу про відрахування, дата смерті, зазначена у свідоцтві про смерть).

        4.5.2.2. Секретар відповідного ОСС на підставі офіційної інформації (наприклад, наказу про відрахування, даних відділу кадрів, копії свідоцтва про смерть тощо) фіксує факт припинення членства у реєстрі членів ОСС.

        4.5.2.3. Добровільний вихід зі складу ОСС здійснюється на підставі письмової заяви студента, поданої керівнику або секретарю відповідного ОСС. Членство припиняється з дати подання заяви, якщо в заяві не вказано іншу дату.

        4.5.2.4. Особа, членство якої в ОСС припинено, втрачає всі права та обов'язки, пов'язані зі статусом члена ОСС.
