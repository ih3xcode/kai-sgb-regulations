\section*{Розділ VI. Контроль та звітність}
\addcontentsline{toc}{section}{Розділ VI. Контроль та звітність}

\subsection*{6.1. Контроль за діяльністю ОСС}
\addcontentsline{toc}{subsection}{6.1. Контроль за діяльністю ОСС}
    6.1.1. Контроль за діяльністю ОСС здійснюється з метою забезпечення дотримання ними законодавства України, Статуту Університету, цього Положення та інших нормативних актів, що регулюють діяльність студентського самоврядування.

    6.1.2. Студентська уповноважена делегація (СУД) є основним органом контролю в системі студентського самоврядування. Повноваження СУД щодо контролю за дотриманням нормативних документів ОСС, фінансовою діяльністю (у випадках, передбачених Положенням про СУД), розгляду скарг та спорів, а також відповідні інструменти контролю визначаються Положенням про СУД.

    6.1.3. Здійснення ОСС звітності у порядку, передбаченому пунктом 6.2 цього Положення, та забезпечення прозорості діяльності згідно з пунктом 6.3 цього Положення є формами громадського контролю за діяльністю органів студентського самоврядування.

\subsection*{6.2. Звітність ОСС}
\addcontentsline{toc}{subsection}{6.2. Звітність ОСС}
    6.2.1. Органи студентського самоврядування є підзвітними студентській спільноті Університету та Конференції студентів Університету (КСУ).

    6.2.2. Усі органи студентського самоврядування зобов'язані звітувати про свою діяльність перед Конференцією студентів Університету (КСУ) не рідше одного разу на рік. Звітування відбувається шляхом усної доповіді керівника (або уповноваженої особи) ОСС та подання письмового звіту під час засідання КСУ.

    6.2.3. Кожен ОСС зобов'язаний створити та підтримувати офіційні інформаційні ресурси (канал в месенджері Telegram, сховище документів Google Drive в межах університетського робочого простору та/або інші ресурси, визначені відповідним ОСС) для інформування студентів про свою діяльність та оприлюднення документів відповідно до пункту 6.3 цього Положення.

    6.2.4. Органи студентського самоврядування факультетів, інститутів, гуртожитків тощо, окрім звітування перед КСУ, звітують перед студентською спільнотою відповідного факультету, інституту, гуртожитку у порядку та строки, визначені положеннями про ці ОСС, але не рідше одного разу на рік.

\subsection*{6.3. Прозорість та відкритість діяльності ОСС}
\addcontentsline{toc}{subsection}{6.3. Прозорість та відкритість діяльності ОСС}
    6.3.1. Статут Університету в частині студентського самоврядування, це Положення, а також зміни до них, підлягають обов'язковому оприлюдненню на офіційних інформаційних ресурсах відповідних ОСС, таких як Google Drive та визначений інформаційний канал, протягом 3 робочих днів з моменту їх затвердження.

    6.3.2. Рішення органів студентського самоврядування (протоколи засідань, рішення, накази тощо) підлягають оприлюдненню на офіційних інформаційних ресурсах відповідного ОСС протягом 3 робочих днів з моменту їх прийняття/підписання, за винятком інформації з обмеженим доступом, перелік та порядок поводження з якою визначається Положенням про СУД та законодавством України.

    6.3.3. Звіти ОСС про свою діяльність підлягають обов'язковому розміщенню на офіційних інформаційних ресурсах відповідного ОСС протягом 3 робочих днів з моменту їх затвердження або представлення.

    6.3.4. Кожен студент має право на вільний доступ до інформації про діяльність ОСС, оприлюдненої відповідно до пунктів 6.3.1 – 6.3.3 цього Положення, через офіційні інформаційні ресурси ОСС.

\subsection*{6.4. Стандарти звітування ОСС}
\addcontentsline{toc}{subsection}{6.4. Стандарти звітування ОСС}
    \subsubsection*{6.4.1. Типи звітів}
        6.4.1.1. Річний звіт -- комплексний документ, що підсумовує діяльність ОСС за календарний рік або навчальний рік (на вибір ОСС) та подається до КСУ відповідно до п. 6.2.2 цього Положення.
        
        6.4.1.2. Проміжний (семестровий) звіт -- документ, що відображає діяльність ОСС за півріччя або семестр, використовується для внутрішнього моніторингу та оцінки прогресу.
        
        6.4.1.3. Тематичний звіт -- документ, що висвітлює окремий напрямок діяльності, проєкт або захід, складається за потреби або на вимогу.
        
        6.4.1.4. Фінансовий звіт -- документ про використання коштів за певний період, який додається до річного або проміжного звіту, якщо ОСС здійснював фінансові операції.

    \subsubsection*{6.4.2. Структура річного звіту}
        6.4.2.1. Річний звіт ОСС повинен містити такі обов'язкові розділи:

            \begin{enumerate}[label=\alph*)]
                \item \textbf{Загальна інформація}: назва ОСС, звітний період, дата складання, склад органу;
                \item \textbf{Виконання плану роботи}: порівняння запланованих і реалізованих заходів та проєктів;
                \item \textbf{Опис основних досягнень та результатів діяльності}: кількісні та якісні показники;
                \item \textbf{Фінансовий звіт} (за наявності фінансування): доходи, витрати, залишок коштів;
                \item \textbf{Взаємодія з іншими органами та організаціями}: опис співпраці;
                \item \textbf{Проблеми та виклики}: аналіз труднощів та способи їх подолання;
                \item \textbf{Плани та пропозиції на наступний період}.
            \end{enumerate}
            
        6.4.2.2. За потреби, структура річного звіту може бути розширена додатковими розділами.
        
        6.4.2.3. Річний звіт має включати як кількісні показники (кількість проведених заходів, охоплення студентів, витрачені кошти тощо), так і якісні показники (зміни в середовищі, задоволеність студентів, вирішені проблеми тощо).

    \subsubsection*{6.4.3. Форма та обсяг звіту}
        6.4.3.1. Річний звіт подається в електронній та паперовій (для КСУ) формах.
        
        6.4.3.2. Обсяг звіту визначається ОСС самостійно, виходячи з потреби повного висвітлення результатів діяльності за звітний період.
        
        6.4.3.3. До звіту можуть додаватися фотоматеріали, інфографіка, копії документів, публікації та інші додатки, що підтверджують та ілюструють діяльність ОСС.
        
        6.4.3.4. Для забезпечення уніфікованого підходу до звітування, СУД розробляє та оновлює шаблони звітів для всіх типів ОСС, які розміщуються у відповідному розділі електронного сховища документів.

    \subsubsection*{6.4.4. Порядок підготовки та затвердження звіту}
        6.4.4.1. Підготовка звіту розпочинається не пізніше, ніж за 20 календарних днів до дати подання.
        
        6.4.4.2. Секретар ОСС або інша уповноважена особа відповідає за збір інформації від всіх структурних підрозділів та посадових осіб ОСС.
        
        6.4.4.3. Проєкт звіту обговорюється на засіданні відповідного ОСС та затверджується колегіально:

            \begin{enumerate}[label=\alph*)]
                \item Річний звіт затверджується більшістю від загального складу відповідного ОСС;
                \item Інші види звітів можуть затверджуватись більшістю від присутніх членів, якщо інше не передбачено внутрішніми документами ОСС.
            \end{enumerate}
            
        6.4.4.4. Затверджений звіт підписується керівником (головою) та секретарем відповідного ОСС.

    \subsubsection*{6.4.5. Оцінка та моніторинг звітності}
        6.4.5.1. СУД може проводити аналіз та оцінку звітів ОСС за такими критеріями:

            \begin{enumerate}[label=\alph*)]
                \item Повнота та достовірність інформації;
                \item Відповідність діяльності цілям та завданням ОСС;
                \item Ефективність використання ресурсів;
                \item Рівень досягнення запланованих результатів;
                \item Дотримання структури та вимог до звітності.
            \end{enumerate}
            
        6.4.5.2. За результатами оцінки, СУД може надавати рекомендації щодо покращення діяльності та звітування ОСС.
        
        6.4.5.3. КСУ та інші уповноважені органи можуть приймати рішення про затвердження або незатвердження річного звіту ОСС. Незатвердження звіту може бути підставою для внесення змін до складу та/або діяльності відповідного ОСС у порядку, визначеному цим Положенням або положеннями про відповідні ОСС.