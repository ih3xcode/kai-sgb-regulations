\section*{Розділ VI. Контроль та звітність}

\subsection*{6.1. Контроль за діяльністю ОСС}
    6.1.1. Контроль за діяльністю ОСС здійснюється з метою забезпечення дотримання ними законодавства України, Статуту Університету, цього Положення та інших нормативних актів, що регулюють діяльність студентського самоврядування.

    6.1.2. Студентська уповноважена делегація (СУД) є основним органом контролю в системі студентського самоврядування. Повноваження СУД щодо контролю за дотриманням нормативних документів ОСС, фінансовою діяльністю (у випадках, передбачених Положенням про СУД), розгляду скарг та спорів, а також відповідні інструменти контролю визначаються Положенням про СУД.

    6.1.3. Здійснення ОСС звітності у порядку, передбаченому пунктом 6.2 цього Положення, та забезпечення прозорості діяльності згідно з пунктом 6.3 цього Положення є формами громадського контролю за діяльністю органів студентського самоврядування.

\subsection*{6.2. Звітність ОСС}
    6.2.1. Органи студентського самоврядування є підзвітними студентській спільноті Університету та Конференції студентів Університету (КСУ).

    6.2.2. Усі органи студентського самоврядування зобов'язані звітувати про свою діяльність перед Конференцією студентів Університету (КСУ) не рідше одного разу на рік. Звітування відбувається шляхом усної доповіді керівника (або уповноваженої особи) ОСС та подання письмового звіту під час засідання КСУ.

    6.2.3. Кожен ОСС зобов'язаний створити та підтримувати офіційні інформаційні ресурси (канал в месенджері Telegram, сховище документів Google Drive в межах університетського робочого простору та/або інші ресурси, визначені відповідним ОСС) для інформування студентів про свою діяльність та оприлюднення документів відповідно до пункту 6.3 цього Положення.

    6.2.4. Органи студентського самоврядування факультетів, інститутів, гуртожитків тощо, окрім звітування перед КСУ, звітують перед студентською спільнотою відповідного факультету, інституту, гуртожитку у порядку та строки, визначені положеннями про ці ОСС, але не рідше одного разу на рік.

\subsection*{6.3. Прозорість та відкритість діяльності ОСС} 
    6.3.1. Статут Університету в частині студентського самоврядування, це Положення, а також зміни до них, підлягають обов'язковому оприлюдненню на офіційних інформаційних ресурсах відповідних ОСС, таких як Google Drive та визначений інформаційний канал, протягом 3 робочих днів з моменту їх затвердження.

    6.3.2. Рішення органів студентського самоврядування (протоколи засідань, рішення, накази тощо) підлягають оприлюдненню на офіційних інформаційних ресурсах відповідного ОСС протягом 3 робочих днів з моменту їх прийняття/підписання, за винятком інформації з обмеженим доступом, перелік та порядок поводження з якою визначається Положенням про СУД та законодавством України.

    6.3.3. Звіти ОСС про свою діяльність підлягають обов'язковому розміщенню на офіційних інформаційних ресурсах відповідного ОСС протягом 3 робочих днів з моменту їх затвердження або представлення.

    6.3.4. Кожен студент має право на вільний доступ до інформації про діяльність ОСС, оприлюдненої відповідно до пунктів 6.3.1 – 6.3.3 цього Положення, через офіційні інформаційні ресурси ОСС.